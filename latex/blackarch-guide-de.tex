%%%%%%%%%%%%%%%%%%%%%%%%%%%%%%%%%%%%%%%%%%%%%%%%%%%%%%%%%%%%%%%%%%%%%%%%%%%%%%%%
%                                                                              %
% BlackArch Linux Handbuch                                                     %
%                                                                              %
%%%%%%%%%%%%%%%%%%%%%%%%%%%%%%%%%%%%%%%%%%%%%%%%%%%%%%%%%%%%%%%%%%%%%%%%%%%%%%%%

\documentclass[a4paper, oneside, 11pt]{book}

%%% INCLUDES %%%
\renewcommand{\familydefault}{\sfdefault}

\usepackage{array}
\usepackage{color}
\usepackage{enumerate}
\usepackage{fancyhdr}
\usepackage{fancyvrb}
\usepackage{geometry}
\usepackage{graphicx}
\usepackage{hyperref}
\usepackage{ifpdf}
\usepackage{listings}
\usepackage{pstricks}
\usepackage{supertabular}
\usepackage{tocloft}
\usepackage[utf8]{inputenc}
\usepackage[T1]{fontenc}
\usepackage{fullpage}
\usepackage[ngerman]{babel}
\usepackage{graphicx}
\clubpenalty10000
\widowpenalty10000
\displaywidowpenalty=10000
\usepackage{ltablex}

\keepXColumns

%%% LAYOUT %%%
\setlength{\parindent}{0em}
\setlength{\parskip}{1.5ex plus0.5ex minus0.5ex}
\geometry{left=2.5cm, textwidth=16cm, top=3cm, textheight=25cm, bottom=3cm}
\widowpenalty=2000
\clubpenalty=1000
\frenchspacing
\sloppy
\pagecolor[HTML]{FFFFFF}
\color[HTML]{333333}
\setcounter{tocdepth}{10}
\setcounter{secnumdepth}{10}

\hypersetup{
  pdftitle={BlackArch Linux, Das BlackArch Linux Handbuch},
  pdfsubject={BlackArch Linux, Das BlackArch Linux Handbuch},
  pdfauthor={BlackArch Linux, BlackArch Linux},
  pdfkeywords={BlackArch Linux, Penetration Testing, Security, Hacking, Linux},
  pdfcenterwindow=true,
  colorlinks=true,
  breaklinks=true,
  linkcolor=blue,
  menucolor=blue,
  urlcolor=blue
}

% syntax highlighting
% all options should be set here document wide
\lstset{
backgroundcolor=\color[HTML]{EEEEEE},
frame=single,
basicstyle=\footnotesize\ttfamily,
float,
deletekeywords={return},
otherkeywords={mkdir, curl, sudo, sha1sum, grep, cut, sort, wget, makepkg,
pacman, blackman, chmod},
keywordstyle=\color{orange},
commentstyle=\color{blue},
stringstyle=\color{red},
language=bash,
showspaces=false,
showtabs=false,
tabsize=2,
texcl=true
breaklines=true
}

%%% HEADER / FOOTER %%%
\setlength{\headheight}{33pt}
\setlength{\headsep}{33pt}
\lhead{{\includegraphics[width=1cm,height=1cm]{images/logo.png}}}
\rhead{Das BlackArch Linux Handbuch}

%%% CUSTOM MACROS %%%
% for html links
\ifpdf\else
\def\href#1#2{\htmladdnormallink{#2}{#1}}
\fi

%------------------%
%  TITLE PAGE      %
%------------------%
\begin{document}
\pagestyle{empty}
\begin{center}
\begin{figure}[htbp]
\centering
\vspace{0.5cm}
\includegraphics[width=8cm]{images/logo.png}
\label{fig:logo}
\end{figure}
\vspace{0.5cm}
\Huge{\textbf{The BlackArch Linux Guide}}\\
\vspace{1cm}
\Large{\color{blue}https://www.blackarch.org/}\\
\vspace{0.5cm}
\end{center}
\newpage
\tableofcontents
\newpage
\pagestyle{fancy}

%------------------%
%  Chapter 1       %
%------------------%

\chapter{Einführung}

\section{Übersicht}
Das BlackArch Linux Handbuch ist in verschiedene Teile aufgeteilt:
\begin{itemize}
\item Einführung - Gibt einen breiten Überblick, eine Einführung, und weitere hilfreiche Projektinformationen
\item Nutzerhandbuch - Alles was ein typischer Nutzer wissen muss um BlackArch zu benutzen
\item Entwicklerhandbuch - Wie kann man zu BlackArch beitragen und entwickeln
\item Tool Guide - Tiefgehende Details zu Tools und Beispiele zur Benutzung (WIP)
\end{itemize}

\section{Was ist BlackArch Linux?}
BlackArch ist eine vollwertige Linux Distribution für Penetration Tester und Security Researcher.
Es basiert auf \href{https://www.archlinux.org/}{ArchLinux} und Nutzer können BlackArch Komponenten einzeln oder in Gruppen installieren.

Das Toolset wird mittel eines inoffiziellem Benutzer Repositories verteilt, so dass man BlackArch auf einer existierenden Arch Linux Installation installieren kann. Pakete können individuell oder über Kategorien installiert werden.
\href{https://wiki.archlinux.org/index.php/Unofficial\_User\_Repositories}
{Inoffizielles Nutzerrepository}

Das konstant wachsende Repository beinhaltet aktuell über \href{https://www.blackarch.org/tools.html}{2600} tools.
Alle tools werden intensiv getestet bevor sie zur Codebasis hinzugefügt werden, um die Qualität des Repositories zu gewährleisten.
% should quickly describe the testing methods/code review procedures etc

\section{Geschichte von BlackArch Linux}
Coming soon...

\section{Unterstützte Plattformen}
Coming soon...

\section{Mitmachen}
Man kann über folgende Wege mit dem BlackArch Team in Kontakt treten:

Website: \url{https://www.blackarch.org/}

Mail: \href{mailto:team@blackarch.org}{team@blackarch.org}

IRC: \url{irc://irc.freenode.net/blackarch}

Twitter: \url{https://twitter.com/blackarchlinux}

Github: \url{https://github.com/Blackarch/}

Matrix: \url{https://matrix.to/#/#BlackArch:matrix.org}

%------------------%
%  Kapitel 2       %
%------------------%


\chapter{Benutzerhandbuch}

\section{Installation}
Der folgende Abschnitt zeigt, wie man das BlackArch Repository einrichtet und Pakete installiert. BlackArch unterstützt sowohl die Installation von Binärpaketen als auch die Installation über selbstkompilierten Quellcode. 

BlackArch ist kompatibel mit regulären Arch installationen. Es verhält sich wie ein inoffizelles Nutzerreporisotry.
Wenn stattdessen ein ISO benötigt wird, siehe den Abschnitt
\href{https://www.blackarch.org/downloads.html#iso}{ISOs}.

\subsection{Installation basierend auf einer vorhandenen ArchLinux Installation}
Führe \href{https://blackarch.org/strap.sh}{strap.sh} als root aus und folge den Anweisungen. 

Hier ein Beispiel.
\begin{lstlisting}
   curl -O https://blackarch.org/strap.sh
   sha1sum strap.sh # should match: 5ea40d49ecd14c2e024deecf90605426db97ea0c
   sudo chmod +x strap.sh
   sudo ./strap.sh
\end{lstlisting}

Jetzt lade eine frische Kopie der Master Paket Liste und synchronisiere die Pakete:
\begin{lstlisting}
  sudo pacman -Syyu
\end{lstlisting}


\subsection{Paketinstallattion}
Jetzt können Tools aus dem BlackArch Repository installiert werden.
\begin{enumerate}
\item Um alle verfügbaren Tools aufzulisten:
\begin{lstlisting}
  pacman -Sgg | grep blackarch | cut -d' ' -f2 | sort -u
\end{lstlisting}

\item Um alle Tools zu installieren:
\begin{lstlisting}
  pacman -S blackarch
\end{lstlisting}

\item Um eine Toolkategorie zu installieren:
\begin{lstlisting}
  pacman -S blackarch-<category>
\end{lstlisting}

\item Um die BlackArch Kategorien zu sehen:
\begin{lstlisting}
  pacman -Sg | grep blackarch
\end{lstlisting}

\end{enumerate}

\subsection{Paketinstallation auf Quellcodebasis}
Alternativ können BlackArch-Pakete auch aus Quellcode gebaut werden. Die PKGBUILDS können auf \href{https://github.com/BlackArch/blackarch/tree/master/packages}{github} gefunden werden. Um das gesamte Repository zu bauen, kann das \href{https://github.com/BlackArch/blackman}{Blackman} tool genutzt werden.
\begin{itemize}
\item Als erste muss Blackman installiert werden. Wenn das BlackArch Reposity auf ihrer Maschine eingerichtet ist, kann Blackman installiert werden:
\begin{lstlisting}
  pacman -S blackman
\end{lstlisting}

\item Blackman kann von Quellcode gebaut und installiert werden: 
\begin{lstlisting}
  mkdir blackman
  cd blackman
  wget https://raw.github.com/BlackArch/blackarch/master/packages/blackman/PKGBUILD
 # Sicherstellen, dass die PKGBUILD nicht bösartig verändert worden sind.
  makepkg -s
\end{lstlisting}

\item Blackman kann auch aus dem AUR installiert werden:
\begin{lstlisting}
  <Verwendeter AUR Helfer> -S blackman
\end{lstlisting}

\end{itemize}

\subsection{Grundlegende Verwendung von Blackman} Blackman ist sehr einfach zu nutzen, auch wenn sich die flags von dem unterscheiden, was man typischerweise von pacman erwarten würde. 
Die Grundlegende Benutzung wird im folgenden gezeigt.
\begin{itemize}
\item Herunterladen, kompilieren and installieren von Paketen:
\begin{lstlisting}
  sudo blackman -i package
\end{lstlisting}

\item Herunterladen, kompilieren und installieren einer ganzen Kategorie:
\begin{lstlisting}
  sudo blackman -g group
\end{lstlisting}

\item Herunterladen, kompilieren und installieren aller BlackArch Tools:
\begin{lstlisting}
  sudo blackman -a
\end{lstlisting}

\item Auflistung aller BlackArch Kategorien:
\begin{lstlisting}
  blackman -l
\end{lstlisting}

\item Auflistung der Tools einer Kategorie:
\begin{lstlisting}
  blackman -p category
\end{lstlisting}

\end{itemize}

\subsection{Installing from full-, netinstall- ISO or ArchLinux}
BlackArch Linux kann von unseren full- oder netinstall-ISOs intalliert werden. \\Siehe
\url{https://www.blackarch.org/download.html#iso}. Die folgenden Schritte sind nötig wenn die ISO gebootet ist.
\begin{itemize}
\item Installieren des blackarch-installer Pakets:
\begin{lstlisting}
  sudo pacman -S blackarch-installer
\end{lstlisting}

\item Run
\begin{lstlisting}
  sudo blackarch-install
\end{lstlisting}

\end{itemize}

%------------------%
%  Chapter 3       %
%------------------%

\chapter{Entwicklerhandbuch}

\section{Das Arch Build System und Repositories}

PKGBUILD Dateien sind Build Skripte. Jedes beschreibt makepkg(1) wie ein Paket gebaut wird. PKGBUILD Dateien werden in Bash geschrieben.

Für weitere Informationen, lese (oder überfliege) folgende Seiten:
\begin{itemize}
\item \href{https://wiki.archlinux.org/index.php/Creating_Packages}{Arch Wiki: Erzeuge Packages}
\item \href{https://wiki.archlinux.org/index.php/Makepkg}{Arch Wiki: makepkg}
\item \href{https://wiki.archlinux.org/index.php/PKGBUILD}{Arch Wiki: PKGBUILD}
\item \href{https://wiki.archlinux.org/index.php/Arch_Packaging_Standards}{Arch Wiki: Arch Packetierungs Standards}
\end{itemize}

\section{Blackarch PKGBUILD Standards}
Der Einfachkeit halber sind unsere PKGBUILDs dem des AUR sehr ähnlich, die kleinen Unterschiede werden im weiteren Text beschrieben. 
Jedes Paket muss mindestens zu blackarch gehören, es wird aber auch viele beziehungen über mehrere pakete die zu mehreren Gruppen gehören geben.

\subsection{Gruppen}
Um es Nutzern zu ermöglichen eine ganze Reihe von Paketen schnell und einfach zu installieren,
wurden Pakete in Gruppen eingeteilt.
Gruppen ermöglichen es den benutzern mit einem einfachen "pacman -S <group name>" eine Menge von Paketen zu bekommen.

\subsubsection{blackarch}
Die blackarch gruppe ist die basis-Gruppe zu der alle Pakete gehören müssen. 
Das ermöglicht es den Nutzern einfach alle Pakete zu installieren.

Was sollte hier drin sein: Alles.

\subsubsection{blackarch-anti-forensic}
Pakete die dazu benutzt werden, forensische Aktivitäten zu umgehen. Das beinhaltet Verschlüsselung, Steganographie 
und alles was es ermöglicht Datei/Ordner Attribute zu manipulieren.
Das alles beinhaltet Tools die allgemein veränderungen an einem System durchführen mit dem Zweck,
Information zu verstecken. 

Beispiele: luks, TrueCrypt, Timestomp, dd, ropeadope, secure-delete

\subsubsection{blackarch-automation}
Pakete zur tool oder workflow Automatisierung.

Beispiele: blueranger, tiger, wiffy

\subsubsection{blackarch-backdoor}
Pakete zur Ausnutzung oder Öffnen von backdoors auf bereits verwundbaren
Systemen.

Beispiele: backdoor-factory, rrs, weevely

\subsubsection{blackarch-binary}
Pakete die auf irgendwelchen Binärdateien arbeiten.

Beispiele: binwally, packerid

\subsubsection{blackarch-bluetooth}
Pakete die alles exploiten was mit dem Bluetooth Standard (802.15.1) zu tun hat.

Beispiele: ubertooth, tbear, redfang

\subsubsection{blackarch-code-audit}
Pakete die bestehenden Code analysieren um Sicherheitslücken zu finden.

Beispiele: flawfinder, pscan

\subsubsection{blackarch-cracker}
Pakete die zum cracken von kryptographischen Funktionen, zum Beispiel Hashes.

Beispiele: hashcat, john, crunch

\subsubsection{blackarch-crypto}
Pakete die mit kryptographie arbeiten, mit der Ausnahme vom cracken.

Beispiele: ciphertest, xortool, sbd

\subsubsection{blackarch-database}
Pakete die Datenbank-Exploits auf jedem Level betreffen.

Beispiele: metacoretex, blindsql

\subsubsection{blackarch-debugger}
Pakete die es dem Nutzer erlauben in Echtzeit zu sehen, was ein bestimmtes Programm tut.

Beispiele: radare2, shellnoob

\subsubsection{blackarch-decompiler}
Pakete die versuchen kompilierte Programm in Quellcode zu konvertieren.

Beispiele: flasm, jd-gui

\subsubsection{blackarch-defensive}
Pakete die Versuchen den Nutzer vor Malware und Attacken anderer Nutzer zu schützen.

Beispiele: arpon, chkrootkit, sniffjoke

\subsubsection{blackarch-disassembler}
Ähnlich zu blackarch-decompiler; Hier gibt es vermutlich einige Programme die
in beide Kategorien fallen, mit dem Unterschied das diese Pakete Assembler ausgeben
statt den puren Quellcode.

Beispiele: inguma, radare2

\subsubsection{blackarch-dos}
Pakete die DoS (Denial of Service) Angriffe nutzen.

Beispiele: 42zip, nkiller2

\subsubsection{blackarch-drone}
Pakete die zur Verwaltung von echten Drohnen verwendet werden.

Beispiele: meshdeck, skyjack

\subsubsection{blackarch-exploitation}
Pakete die exploits anderer Programme oder Dienste nutzen.

Beispiele: armitage, metasploit, zarp

\subsubsection{blackarch-fingerprint}
Pakete die Fingerabdrücke biometrischer Systeme exploiten.

Beispiele: dns-map, p0f, httprint

\subsubsection{blackarch-firmware}
Pakete die Schwachstellen in Firmware ausnutzen.

Beispiele: Noch keine, asap hinzufügen.

\subsubsection{blackarch-forensic}
Pakete die benutzt werden um Daten auf physischen Festplatten oder Speicher zu finden.

Beispiele: aesfix, nfex, wyd

\subsubsection{blackarch-fuzzer}
Pakete die die Fuzzy Testprinzipien nutzen, zum Beispiel
zufälligen Input "reinzuwerfen" und zu sehen was passiert.

Beispiele: msf, mdk3, wfuzz

\subsubsection{blackarch-hardware}
Pakete die alles verwalten oder ausnutzen was mit
physischer Hardware zu tun hat.

Beispiele: arduino, smali

\subsubsection{blackarch-honeypot}
Pakete die als "honeypots" fungieren. Zum Beispiel Programme
die sich als verwundbare Dienste ausgeben und Hacker in eine Falle locken
sollen.

Beispiele: artillery, bluepot, wifi-honey

\subsubsection{blackarch-keylogger}
Pakete die Tastendrücke auf anderen Systemen aufnehmen und speichern.

Beispiele: None yet, amend asap.

\subsubsection{blackarch-malware}
Pakete die zu Malware zählen oder Malware erkennung.

Beispiele: malwaredetect, peepdf, yara

\subsubsection{blackarch-misc}
Pakete die nicht unbedingt in eine spezielle Kategorie passen.

Beispiele: oh-my-zsh-git, winexe, stompy

\subsubsection{blackarch-mobile}
Pakete die Mobile Plattformen manipulieren.

Beispiele: android-sdk-platform-tools, android-udev-rules

\subsubsection{blackarch-networking}
Pakete die IP Netzerke betreffen.

Beispiele: arptools, dnsdiag, impacket

\subsubsection{blackarch-nfc}
Pakete die NFC (near-field communication) nutzen.

Beispiele: nfcutils

\subsubsection{blackarch-packer}
Pakete die Packer bedienen oder beinhalten.

\textit{Packer sind Programme die malware in anderen Executables einbetten. }

Beispiele: packerid

\subsubsection{blackarch-proxy}
Pakete die als Proxy fungieren, also zum Beispiel Netzwerkverkehr durch einen
anderen Knoten im Internet umleiten.

Beispiele: burpsuite, ratproxy, sslnuke

\subsubsection{blackarch-recon}
Pakete die aktiv verwundbare exploits suchen.
Eine Obergruppe für ähnliche Pakete.

Beispiele: canri, dnsrecon, netmask

\subsubsection{blackarch-reversing}
Ûbergruppe für jegliche decompiler, 
disassembler oder ähnliche Programme.

Beispiele: capstone, radare2, zerowine

\subsubsection{blackarch-scanner}
Pakete die ausgewählte Systeme auf Schwachstellen scannen.

Beispiele: scanssh, tiger, zmap

\subsubsection{blackarch-sniffer}
Pakete die mit dem analysieren von Netzwerkverkehr
zu tun haben.

Beispiele: hexinject, pytactle, xspy

\subsubsection{blackarch-social}
Pakete die hauptsächlich soziale netzwerke angreifen.

Beispiele: jigsaw, websploit

\subsubsection{blackarch-spoof}
Pakete die versuchen den Angreifer zu spoofen, 
sodass der Angreifer nicht als Angreifer fure das Opfer zu 
erkennen ist.

Beispiele: arpoison, lans, netcommander

\subsubsection{blackarch-threat-model}
Pakete die zum Reporten/Aufnehmen des Threat-Models
in einem speziellen Szenario benutzt werden.

Beispiele: magictree

\subsubsection{blackarch-tunnel}
Pakete die dazu genutzt werden, Netzwerkverkehr zu einem
gegebenen Netzwerk zu tunneln.

Beispiele: ctunnel, iodine, ptunnel

\subsubsection{blackarch-unpacker}
Pakete die dazu genutzt werden, vorgepackten Schadcode von 
einer executable auszupacken.

Beispiele: js-beautify

\subsubsection{blackarch-voip}
Pakete die auf VOIP Programmen und Protokollen arbeiten.

Beispiele: iaxflood, rtp-flood, teardown

\subsubsection{blackarch-webapp}
Pakete die auf internet-zugewandten Anwendungen arbeiten.

Beispiele: metoscan, whatweb, zaproxy

\subsubsection{blackarch-windows}
Diese Gruppe ist für native Windows Pakete die unter wine laufen.

Beispiele: 3proxy-win32, pwdump, winexe

\subsubsection{blackarch-wireless}
Pakete die auf drahtlosen Netzwerken arbeiten.

Beispiele: airpwn, mdk3, wiffy

\section{Repository Struktur}
Das primäre git repo für BlackArch befindet sich hier:
\href{https://github.com/BlackArch/blackarch}{https://github.com/BlackArch/blackarch}.
Es gibt ausserdem verschiedene Sekundäre Repositories hier:
\href{https://github.com/BlackArch}{https://github.com/BlackArch}.

Innerhalb des Hauptrepos gibt es drei wichtige Verzeichnisse:

\begin{itemize}
\item docs - Dokumentation.
\item packages - PKGBUILD Dateien.
\item scripts - Nützliche kleine Skripte.
\end{itemize}

\subsection{Scripts}
Hier eine Referenz für Skripte im \verb|scripts/| Verzeichnis:

\begin{itemize}
\item baaur - Coming soon: Wird Pakete in das AUR hochladen.
\item babuild - Baut ein Paket.
\item bachroot - Managen eines chroot zum testen.
\item baclean - Räumt alte .pkg.tar.xz Dateien aus dem Paket Repository.
\item baconflict - Wird bald \verb|scripts/conflicts| ersetzen.
\item bad-files - Findet schlechte Dateien in gebauten Paketen.
\item balock - Anlegen oder lösen des Repository locks.
\item banotify - IRC benachrichtigen über Paket pushes.
\item barelease - Veröffentlicht Pakete in das Repository.
\item baright - Gibt die BlackArch Copyright Informationen aus.
\item basign - Signiert Packete.
\item basign-key - Signiert einen Schlüssel.
\item blackman - Verhält sich ähnlich wie pacman, baut aber aus git. 
	(Nicht zu verwechseln mit nrz's Blackman)
\item check-groups - Überprüft groups.
\item checkpkgs - Überprüft Pakete auf Fehler.
\item conflicts - Sucht nach Dateikonflikten.
\item dbmod - Modifiziert eine Paketdatenbank.
\item depth-list - Erzeugt eine Liste sortiert nach Abhängigkeitspfad.
\item deptree - Erzeugt einen Abhängigkeitsbaum, der nur blackarch Pakete enthält.
\item get-blackarch-deps - Liefert eine List von blackarch Abhängigkeiten für ein Paket.
\item get-official - Liefert offizielle Pakete zum Release.
\item list-loose-packages - Listet Pakete die weder in Gruppen noch Abhängigkeiten anderer Pakete sind.
\item list-needed - Liste fehlender Abhängigkeiten.
\item list-removed - Liste von Pakete die im Paketrepository sind aber nicht im git.
\item list-tools - Liste der Tools.
\item outdated - Sucht nach veralteten Paketen im Repository im Vergleich zum git Repository.
\item pkgmod - Modifiziert ein Buildpaket.
\item pkgrel - Zählt die pkgrel in einem Paket hoch.
\item prep - Aufräumen des PKGBUILD Datei-Styles und Fehlersuche.
\item sitesync - Synchronisiert zwischein einer lokalen Kopie des Paketrepositories und der Remote.
\item size-hunt - Sucht nach grossen Paketen.
\item source-backup - Backup von package source Dateien.
\end{itemize}

\section{Beitragen zum BlackArch Repository}
Dieser Abschnitt zeigt, wie Beiträge im BlackArch Linux Projekt gemacht werden.
wir akzeptieren Pull Requests jeglicher Grösse, von kleinen Tippfehler-Korrekturen 
bis zu neuen Paketen. \\Für Hilfe, Vorschlage oder Fragen Kontaktiere uns. 
\\\\
Jeder ist willkommen. Alle Beiträge werden geschätzt.

\subsection{Benötigte Tutorials}
Bitte lies folgende Tutorials bevor du mitmachst:
\begin{itemize}
\item
\href{https://wiki.archlinux.org/index.php/Arch\_Packaging\_Standards)}{Arch
Packaging Standards}
\item \href{https://wiki.archlinux.org/index.php/Creating\_Packages}{Paketerzeugung}
\item \href{https://wiki.archlinux.org/index.php/PKGBUILD}{PKGBUILD}
\item \href{https://wiki.archlinux.org/index.php/Makepkg}{Makepkg}
\end{itemize}

\subsection{Schritte zum Mitmachen}
Um Änderungen zum BlackArchLinux Projekt zu submitten, folge diesen Schritten:
steps:
\begin{enumerate}
\item Fork das Repository von
\url{https://github.com/BlackArch/blackarch}
\item Hacke die benötigten Dateien (z.B. PKGBUILD, .patch files, usw).
\item Committe deine Änderungen.
\item Pushe deine Äderungen.
\item Bitte uns darum deine changes zu mergen, am liebsten durch einen Pull Request.
\end{enumerate}

\subsection{Beispiel}
Das folgende Beispiel zeigt, wie ein neues Paket zum BlackArch Projekt
submitted wird.
Wir benutzen \href{https://github.com/Jguer/yay}{yay}
(pacaur kann auch benutzt werden) um eine bereits existierende PKGBUILD Datei für
\textbf{nfsshell} aus dem \href{https://aur.archlinux.org/}{AUR} herunter zu laden und nach unseren
Bedürfnissen anzupassen.

\subsubsection{Fetch PKGBUILD}
Die \textit{PKGBUILD} Datei mit yay oder pacaur holen:
\begin{lstlisting}
  user@blackarchlinux $ yay -G nfsshell
  ==> Download nfsshell sources
  x LICENSE
  x PKGBUILD
  x gcc.patch
  user@blackarchlinux $ cd nfsshell/
\end{lstlisting}

\subsubsection{Aufräumen der PKGBUILD}
Aufräumen der \textit{PKGBUILD} Datei und ein bisschen Zeit sparen:
\begin{lstlisting}
  user@blackarchlinux nfsshell $ ./blackarch/scripts/prep PKGBUILD
  cleaning 'PKGBUILD'...
  expanding tabs...
  removing vim modeline...
  removing id comment...
  removing contributor and maintainer comments...
  squeezing extra blank lines...
  removing '|| return'...
  removing leading blank line...
  removing $pkgname...
  removing trailing whitespace...
\end{lstlisting}

\subsubsection{PKGBUILD anpassen}
Anpassen der \textit{PKGBUILD} Datei:
\begin{lstlisting}
  user@blackarchlinux nfsshell $ vi PKGBUILD
\end{lstlisting}

\subsubsection{Das Paket bauen}
Bau das Paket:
\begin{lstlisting}user@blackarchlinux nfsshell $ makepkg -sf
==> Making package: nfsshell 19980519-1 (Mon Dec  2 17:23:51 CET 2013)
==> Checking runtime dependencies...
==> Checking buildtime dependencies...
==> Retrieving sources...
-> Downloading nfsshell.tar.gz...
% Total    % Received % Xferd  Average Speed   Time    Time     Time
CurrentDload  Upload   Total   Spent    Left  Speed100 29213  100 29213    0
0  48150      0 --:--:-- --:--:-- --:--:-- 48206
-> Found gcc.patch
-> Found LICENSE
...
<lots of build process and compiler output here>
...
==> Leaving fakeroot environment.
==> Finished making: nfsshell 19980519-1 (Mon Dec  2 17:23:53 CET 2013)
\end{lstlisting}

\subsubsection{Installieren und testen des Pakets}
Installiere und teste das Paket:
\begin{lstlisting}
  user@blackarchlinux nfsshell $ pacman -U nfsshell-19980519-1-x86_64.pkg.tar.xz
  user@blackarchlinux nfsshell $ nfsshell # test it
\end{lstlisting}

\subsubsection{Adde, commite and pushe das Paket}
Füge das Paket hinzu, mach den Commit und Pushe.
\begin{lstlisting}user@blackarchlinux nfsshell $ cd /blackarchlinux/packages
user@blackarchlinux ~/blackarchlinux/packages $ mv ~/nfsshell .
user@blackarchlinux ~/blackarchlinux/packages $ git commit -am nfsshell && git push
\end{lstlisting}

\subsubsection{Erzeuge einen Pull Request}
Erzeuge einen Pull Request auf \href{https://github.com/}{github.com}
\begin{lstlisting}
  firefox https://github.com/<contributor>/blackarchlinux
\end{lstlisting}

\subsubsection{Füge eine upstream remote hinzu.}
Es ist eine gute Idee wenn man upstream auf einem Fork arbeitet, den eigenen Fork zu pullen und das Haupt-BlackArch repository 
als eine Remote hinzuzufügen.
\begin{lstlisting}
  user@blackarchlinux ~/blackarchlinux $ git remote -v
  origin <the url of your fork> (fetch)
  origin <the url of your fork> (push)
  user@blackarchlinux ~/blackarchlinux $ git remote add upstream https://github.com/blackarch/blackarch
  user@blackarchlinux ~/blackarchlinux $ git remote -v
  origin <the url of your fork> (fetch)
  origin <the url of your fork> (push)
  upstream https://github.com/blackarch/blackarch (fetch)
  upstream https://github.com/blackarch/blackarch (push)
\end{lstlisting}

Standardmäßig sollte git direkt auf origin pushen, aber stelle sicher das deine git konfiguration
richtig konfiguriert ist. Das sollte kein Problem sein, solange du commit rechte hast, da du ohne diese
nicht upstream pushen kannst.

Wenn du nicht committen kannst, könntest du mehr Erfolg mit
\textit{git@github.com:blackarch/blackarch.git} haben.

\subsection{Requests}
\begin{enumerate}
\item Füge keine \textbf{Maintainer} oder \textbf{Contributor} Kommentare zu
\textit{PKGBUILD} Dateien hinzu. Füge maintainer und contributor Namen zu der
AUTHORS sektion im BlackArch guide hinzu.
\item Der Konsistenz willen, bitte folge dem generellen Stil anderer
\textit{PKGBUILD} Dateien im repo und nutze doppel-space Einrückungen.
\end{enumerate}

\subsection{Generelle tips}
\href{http://wiki.archlinux.org/index.php/Namcap}{namcap} kann Pakete auf Fehler überprüfen.

%------------------%
%  Kapitel 4       %
%------------------%

\chapter{Tool Handbuch}
Coming soon...

\section{Coming Soon}
Coming soon...

%%% APPENDIX %%%
\appendix
\include{latex/appendix-de}

\end{document}

%%% EOF %%%
