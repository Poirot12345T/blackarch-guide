%%%%%%%%%%%%%%%%%%%%%%%%%%%%%%%%%%%%%%%%%%%%%%%%%%%%%%%%%%%%%%%%%%%%%%%%%%%%%%%%
%                                                                              %
% BlackArch Linux Guide                                                        %
%                                                                              %
%%%%%%%%%%%%%%%%%%%%%%%%%%%%%%%%%%%%%%%%%%%%%%%%%%%%%%%%%%%%%%%%%%%%%%%%%%%%%%%%

\documentclass[a4paper, oneside, 11pt]{book}

%%% INCLUDES %%%
\renewcommand{\familydefault}{\sfdefault}

\usepackage{array}
\usepackage{color}
\usepackage{enumerate}
\usepackage{fancyhdr}
\usepackage{fancyvrb}
\usepackage{geometry}
\usepackage{graphicx}
\usepackage{html}
\usepackage{hyperref}
\usepackage{ifpdf}
\usepackage{listings}
\usepackage{pstricks}
\usepackage{supertabular}
\usepackage{tocloft}
\usepackage[utf8]{inputenc}

%%% LAYOUT %%%
\setlength{\parindent}{0em}
\setlength{\parskip}{1.5ex plus0.5ex minus0.5ex}
\geometry{left=2.5cm, textwidth=16cm, top=3cm, textheight=25cm, bottom=3cm}
\widowpenalty=2000
\clubpenalty=1000
\frenchspacing
\sloppy
\pagecolor[HTML]{FFFFFF}
\color[HTML]{333333}
\setcounter{tocdepth}{10}
\setcounter{secnumdepth}{10}

\hypersetup{
  pdftitle={BlackArch Linux, The BlackArch Linux Guide},
  pdfsubject={BlackArch Linux, The BlackArch Linux Guide},
  pdfauthor={BlackArch Linux, BlackArch Linux},
  pdfkeywords={BlackArch Linux, Penetration Testing, Security, Hacking, Linux},
  pdfcenterwindow=true,
  colorlinks=true,
  breaklinks=true,
  linkcolor=blue,
  menucolor=blue,
  urlcolor=blue
}

% syntax highlighting
% all options should be set here document wide
\lstset{
backgroundcolor=\color[HTML]{EEEEEE},
frame=single,
basicstyle=\footnotesize\ttfamily,
float,
deletekeywords={return},
otherkeywords={mkdir, curl, sudo, sha1sum, grep, cut, sort, wget, makepkg,
pacman, blackman, chmod},
keywordstyle=\color{orange},
commentstyle=\color{blue},
stringstyle=\color{red},
language=bash,
showspaces=false,
showtabs=false,
tabsize=2
}

%%% HEADER / FOOTER %%%
\setlength{\headheight}{33pt}
\setlength{\headsep}{33pt}
\lhead{{\includegraphics[width=1cm,height=1cm]{images/logo.png}}}
\rhead{The BlackArch Linux Guide}

%%% CUSTOM MACROS %%%
% for html links
\ifpdf\else
\def\href#1#2{\htmladdnormallink{#2}{#1}}
\fi

%------------------%
%  TITLE PAGE      %
%------------------%
\begin{document}
\pagestyle{empty}
\begin{center}
\begin{figure}[htbp]
\centering
\vspace{0.5cm}
\includegraphics[width=8cm]{images/logo.png}
\label{fig:logo}
\end{figure}
\vspace{0.5cm}
\Huge{\textbf{The BlackArch Linux Guide}}\\
\vspace{1cm}
\Large{\color{blue}https://www.blackarch.org/}\\
\vspace{0.5cm}
\end{center}
\newpage
\tableofcontents
\newpage
\pagestyle{fancy}

%------------------%
%  Chapter 1       %
%------------------%

\chapter{Introducere}

\section{Overview}
Manualul pentru BlackArch Linux este impartit in cateva parti:
\begin{itemize}

\item Introducere-Ofera o previzualizare, o introducere, niste informatii in plus
\item Manualul utilizatorului-Totul ce un utilizator trebuie sa stie pentru a folosi BlackArch eficient
\item Manualul Dezvoltatorului-Cum sa contribui in proiect
\item Manualul pentru ustensile-Explicatii clare si precise despre folosirea lor (WIP)
\end{itemize}

\section{Ce este BlackArch Linux?}
BlackArch este o distributie Linux completa pentru securitate cibernetica

Este derivata de la  \href{https://www.archlinux.org/}{ArchLinux} iar utilizatorii pot instala componnentele BlackArch separat

Intregul sistem este distribuit ca Arch Linux \href{https://wiki.archlinux.org/index.php/Unofficial\_User\_Repositories}
{unofficial user repository} deci, poti instala BlackArch intr-o instalatie ArchLinux obisnuita.Pachetele pot fi instalate separat sau pe categorii

Este continuu actualizat continand peste
\href{https://www.blackarch.org/tools.html}{2600} unelte.
Toate uneltele sunt testate riguros  si adaugate intr-o baza de date pentru a fi folosite la capacitate maxima
% should quickly describe the testing methods/code review procedures etc

\section{Istoria BlackArch Linux}
Incurand...

\section{Platforme Suportate}
Incurand...

\section{Ajuta-ne}
Poti contacta echipa BlackArch print urmatoarele metode:

Website: \url{https://www.blackarch.org/}

Mail: \href{mailto:team@blackarch.org}{team@blackarch.org}

IRC: \url{irc://irc.freenode.net/blackarch}

Twitter: \url{https://twitter.com/blackarchlinux}

Github: \url{https://github.com/Blackarch/}

Matrix: \url{https://matrix.to/#/#BlackArch:matrix.org}

%------------------%
%  Chapter 2       %
%------------------%


\chapter{Manualul Utilizatorului}

\section{Instalare}

Urmatoarea sectiune iti va arata cum sa instalezi  BlackArch repository si cum sa instalezi pachete. BlackArch suporta ambele variante, instalarea din repository folosind
pachete binare cat si compiland si instaland din sursa.

BlackArch este compatibil cu normala instalare Arch.Este exact ca un neoficial 'pachet'. Daca vrei un ISO, mai jos
\href{https://www.blackarch.org/downloads.html#iso}{Live ISO} section.

\subsection{Instalarea sa pe ArchLinux}
Ruleaza \href{https://blackarch.org/strap.sh}{strap.sh} ca root urmand urmatoarele instructiuni. Urmatorul exemplu
\begin{lstlisting}
   curl -O https://blackarch.org/strap.sh
   sha1sum strap.sh # should match: 5ea40d49ecd14c2e024deecf90605426db97ea0c
   sudo chmod +x strap.sh
   sudo ./strap.sh
\end{lstlisting}

Acum istaleaza o copie noua a pachetului master listeaza si sincronizeaza pachetele:
\begin{lstlisting}
  sudo pacman -Syyu
\end{lstlisting}



\subsection{Instalarea pachetelor}
Acum poti instala uneltele din blackarch repository.
\begin{enumerate}
\item Pentru a lista uneltele valabile, ruleaza
\begin{lstlisting}
  pacman -Sgg | grep blackarch | cut -d' ' -f2 | sort -u
\end{lstlisting}

\item Petru a instala toate uneltele, ruleaza
\begin{lstlisting}
  pacman -S blackarch
\end{lstlisting}

\item Pentru a instala o anumita categorie, ruleaza
\begin{lstlisting}
  pacman -S blackarch-<category>
\end{lstlisting}

\item Pentru a instala categoriile BlackArch, ruleaza
\begin{lstlisting}
  pacman -Sg | grep blackarch
\end{lstlisting}

\end{enumerate}


\subsection{Instalarea pachetelor din sursa}
Ca parte din o, poti construi pachete  BlackArch din sursa. Poti gasi PKGBUILDs in
\href{https://github.com/BlackArch/blackarch/tree/master/packages}{github}. To
creeaza intregul repo, poti folosi
\href{https://github.com/BlackArch/blackman}{Blackman} tool.
\begin{itemize}
\item Prima data, trebuie sa instalezi Blackman. Daca repo-ul pentru pachetul BlackArch
este configurat, poti instala Blackman:
\begin{lstlisting}
  pacman -S blackman
\end{lstlisting}

\item Poti instala si configura Blackman din sursa:
\begin{lstlisting}
  mkdir blackman
  cd blackman
  wget https://raw.github.com/BlackArch/blackarch/master/packages/blackman/PKGBUILD
  # Make sure the PKGBUILD has not been maliciously tampered with.
  makepkg -s
\end{lstlisting}

\item Sau poti instala BlackArch din AUR:
\begin{lstlisting}
  <whatever AUR helper you use> -S blackman
\end{lstlisting}

\end{itemize}
\subsection{Folosirea simpla a lui Blackman} BlackMan e foarte usor de folosit, desi e diferent de ce te-ai fi asteptat, e diferit de pacman.
\begin{itemize}
\item Downloadeaza, compileaza si instaleaza pachete:
\begin{lstlisting}
  sudo blackman -i package
\end{lstlisting}

\item Descarca, compileaza si instaleaza intreaga categorie:
\begin{lstlisting}
  sudo blackman -g group
\end{lstlisting}

\item Descarca, compileaza si instaleaza toate uneltele BlackArch:
\begin{lstlisting}
  sudo blackman -a
\end{lstlisting}

\item Pentru a lista categoriile BlackArch:
\begin{lstlisting}
  blackman -l
\end{lstlisting}

\item Pentru a lista categoriile uneltelor:
\begin{lstlisting}
  blackman -p category
\end{lstlisting}

\end{itemize}

\subsection{Instalarea din  live-, netinstall- ISO sau ArchLinux}
Poti instala BlackArch Linux din unul dintre live- neinstall-.ISO.\\See
\url{https://www.blackarch.org/download.html#iso}. Urmatorii pasi sunt necesari dupa boot up-ul ISO-ului
\begin{itemize}
\item Instalarea instalarului-blackarch:
\begin{lstlisting}
  sudo pacman -S blackarch-installer
\end{lstlisting}

\item Run
\begin{lstlisting}
  sudo blackarch-install
\end{lstlisting}

\end{itemize}

%------------------%
%  Chapter 3       %
%------------------%

\chapter{Manualul creatorului}

\section{Sistemul de construire al Arch si pachete separate}


PKGBUILD sunt fisiere. Fiecare ii  comunica  makepkg(1) cum sa creeze un
pachet. Fisierele PKGBUILD  sunt scrise in Bash.

Pentru mai multe informatii, citeste (sau rasfoieste) urmatoarele:
\begin{itemize}
\item \href{https://wiki.archlinux.org/index.php/Creating_Packages}{Arch Wiki: Crearea pachetelor}
\item \href{https://wiki.archlinux.org/index.php/Makepkg}{Arch Wiki: makepkg}
\item \href{https://wiki.archlinux.org/index.php/PKGBUILD}{Arch Wiki: PKGBUILD}
\item \href{https://wiki.archlinux.org/index.php/Arch_Packaging_Standards}{Arch Wiki:  inpacetarea standard a Arch }
\end{itemize}

\section{PKGBUILD pentru standardele BlackArch}
Pentru o mai buna intelegere,  PKGBUILDs sunt similare cu cele AUR,
cu mici diferente enumerate mai jos. Fiecare pachet trebuie sa apartina BlackArch catusi de putin.

\subsection{Grupuri}
Pentru a instala o specifica ramura de pachete rapid si usor,pachetele au fost
separate in grupuri. Acestea se pot instala prin simplul 'pacman -S <group name>' pentru a instala multiple pachete

\subsubsection{blackarch}
Grupul BlackArch este baza tuturor grupurilor is este locul in care pachetele thebuie sa vie stocate. Acest lucru da voie utilizatorilor sa instaleze pachete usor
Ce ar trebuie sa contina: TOTUL.

\subsubsection{blackarch-anti-forensic}
Pachetele ce sunt necesare pentru forensic,
incluzand encryption, steganography, si orice  ar trebuie sa modifice fisiere.
Aceste unelte ar trebui sa functioneze in cazul in care doriti sa ascundeti informatii.

Exemple: luks, TrueCrypt, Timestomp, dd, ropeadope, secure-delete

\subsubsection{blackarch-automation}
Pachete ce ar trebui folosite pentru workflow automation

Exemple: blueranger, tiger, wiffy

\subsubsection{blackarch-backdoor}
Pachete ce  exploateaza sau folosesc backdoors sunt  deja  in sisteme vulnerabile.

Exemple: backdoor-factory, rrs, weevely

\subsubsection{blackarch-binary}
Pachete ce opereaza pe fisiere binare.

Exemple: binwally, packerid

\subsubsection{blackarch-bluetooth}
Pachete ce ataca orice are legatura cu Bluetooth standard (802.15.1).

Exemple: ubertooth, tbear, redfang

\subsubsection{blackarch-code-audit}
Pachete ce  scaneaza o sursa existenta de cod in cautare de vulnerabilitati.

Exemple: flawfinder, pscan

\subsubsection{blackarch-cracker}
Pachete folosite pentru cracking cryptographic functions, ie hashes.

Exemple: hashcat, john, crunch

\subsubsection{blackarch-crypto}
Pachete ce folosesc  cryptography,cu exeptia decriptarii.

Exemple: ciphertest, xortool, sbd

\subsubsection{blackarch-database}
Packages that involve database exploitations on any level.

Examples: metacoretex, blindsql

\subsubsection{blackarch-debugger}
Pachete ce dau voie utilizatorului sa vizioneza ce ''face' un program intr-un anumit timp.

Exemple: radare2, shellnoob

\subsubsection{blackarch-decompiler}
Pachete ce incearca sa face  reverse la un program pentru ai vedea codul sursa.

Examples: flasm, jd-gui

\subsubsection{blackarch-defensive}
Pachete ce sunt folosite pentru a apara utilizatorul de virusi si atacuri.

Exemple: arpon, chkrootkit, sniffjoke

\subsubsection{blackarch-disassembler}
Este similar cu blackarch-decompiler, si vor fi probabil multe programe ce vor ajunge in mai multe catregorii, desi aceste pachete produc assembly output

Exemple: inguma, radare2

\subsubsection{blackarch-dos}
Pachete folosite pentru D0S  (Denial of Service).

Exemple: 42zip, nkiller2

\subsubsection{blackarch-drone}
Pachete ce sunt folosite pentru a folosi drone

Exemple: meshdeck, skyjack

\subsubsection{blackarch-exploitation}
Pachete ce se folosesc de vulnerabilitatiile altor programe sau servicii

Exemple: armitage, metasploit, zarp

\subsubsection{blackarch-fingerprint}
Pachete ce exploateaza  echipament fingerprint biometric.

Exemple: dns-map, p0f, httprint

\subsubsection{blackarch-firmware}
Pachete ce expluateaza vulnerabilitati in firmware:

Exemple: None yet, amend asap.

\subsubsection{blackarch-forensic}
Pachhete ce sunt folosite pentru a descoperi data pe disk-uri si memorie interna.

Exemple: aesfix, nfex, wyd

\subsubsection{blackarch-fuzzer}
Pachete ce sunt folosite pentru fuzz, ie
'aruncand' valori "la nimereala" in campuri de input pentru a vedea ce se intampla.

Exemple: msf, mdk3, wfuzz

\subsubsection{blackarch-hardware}
Pachete ce exploateaza sau folosesc orice are legatura cu hardware.

Exemple: arduino, smali

\subsubsection{blackarch-honeypot}
Pachete ce au legatura cu "honeypots", ie, programe ce au tendinta sa para vulnerabile pentru a fii exploatate.

Exemple: artillery, bluepot, wifi-honey

\subsubsection{blackarch-keylogger}
Pachetele ce tin cont de functionalitatea sistemului tau.

Exemple: None yet, amend asap.

\subsubsection{blackarch-malware}
Pachete ce tin cont de malware.

Exemple: malwaredetect, peepdf, yara

\subsubsection{blackarch-misc}
Pachte ce nu se incadreaza in vreo categorie.

Exemple: oh-my-zsh-git, winexe, stompy

\subsubsection{blackarch-mobile}
Pachete ce manipuleaza platforme mobile.

Exemple: android-sdk-platform-tools, android-udev-rules

\subsubsection{blackarch-networking}
Pachetel ce au legatura cu ip-ul.

Exemple: Cam toate 

\subsubsection{blackarch-nfc}
Pachete  ce se folosesc ce nfc(near-field communications).

Exemple: nfcutils

\subsubsection{blackarch-packer}
Pachete ce se folesesc de alte pachete.

\textit{packers sunt programe ce contin malware.}

Exemple: packerid

\subsubsection{blackarch-proxy}
Pachete ce sunt considerate ca proxy, ie redirectionand date /trafic
print internet.

Exemple: burpsuite, ratproxy, sslnuke

\subsubsection{blackarch-recon}
Pachete ce cauta vulnerabilitati. Mai mult o adunare decat un grup.

Exemple: canri, dnsrecon, netmask

\subsubsection{blackarch-reversing}
O adunare pentru dezasamblare

Exemple: capstone, radare2, zerowine

\subsubsection{blackarch-scanner}
Pachete ce cauta vulnerabilitati in sisteme specifice .

Exemple: scanssh, tiger, zmap

\subsubsection{blackarch-sniffer}
Pachete ce se concentreaza pe analizat traficul de date.

Exemple: hexinject, pytactle, xspy

\subsubsection{blackarch-social}
Pachete pentru social networking sites.

Exemple: jigsaw, websploit

\subsubsection{blackarch-spoof}
Pachete ce incearca sa faca spoof atacatorul, unde atacatorul devine o victima

Exemple: arpoison, lans, netcommander

\subsubsection{blackarch-threat-model}
Pachete  ce se folosesc pentru reporting/recording o amenintare.

Exemple: magictree

\subsubsection{blackarch-tunnel}
Pachete ce sunt folosite pentru a directiona network traffic la un anumit network.

Exemple : ctunnel, iodine, ptunnel

\subsubsection{blackarch-unpacker}
Pachete ce sunt  folosite pentru a extracta pre-facut malware din un
executabil.

Exemple: js-beautify

\subsubsection{blackarch-voip}
Programe ce opereaza pe voip programs si protocoale.

Exemple: iaxflood, rtp-flood, teardown

\subsubsection{blackarch-webapp}
Programe ce opereaza pe internet-facing applications.

Exemple: metoscan, whatweb, zaproxy

\subsubsection{blackarch-windows}
Aceasta categorie este alcatuita de programe ce ruleaza nativ pe windows,  iar noi folosim wine pentru a le folosi pe BlackArch.

Exemple: 3proxy-win32, pwdump, winexe

\subsubsection{blackarch-wireless}
Programe ce opereaza pe wireless networks la orice nivel.

Exemple: airpwn, mdk3, wiffy

\section{Repository structure}
Poti gasi intregul BlackArch repo aici:
\href{https://github.com/BlackArch/blackarch}{https://github.com/BlackArch/blackarch}.
Avem is cateva repos secundare aici:
\href{https://github.com/BlackArch}{https://github.com/BlackArch}.

In repo-ul principal de pe git, exista trei categorii imprtante:

\begin{itemize}
\item docs - Documentatie.
\item packages - fisiere PKGBUILD.
\item scripts - Scripturi folositoare.
\end{itemize}

\subsection{Scripts}
O referinta pentru scripturi se gaseste aici  \verb|scripts/| directory:

\begin{itemize}
\item baaur-Incurand, acesta va uploada fisiere pe  AUR.
\item babuild-Va construi un pachet.
\item bachroot-Va alege un chroot pentru folosire.
\item baclean-Va curata pachete vechi ca.pkg.tar.xz din  repo.
\item baconflict-Incurand va inlocui \verb|scripts/conflicts|.
\item bad-files-Va gasi fisiere corupte.
\item balock-Va detine sau dona pachete.
\item banotify-Va notifica IRC despre package pushes.
\item barelease-Va da pachete pe repo.
\item baright-Ofera  BlackArch copyright info.
\item basign-Pachete de logare.
\item basign-key-Va asocia o cheie.
\item blackman-Are acelasi principiu ca pacman dar pentru git(a nu se confunda cu nrz's Blackman).
\item check-groups-Verifica grupurile.
\item checkpkgs-Verifica pachete pentru erori.
\item conflicts-Verifica pentru conflicturi de pachete.
\item dbmod-Modifica baza de date pentru pachete.
\item depth-list-Creaza o lista pentru dependency depth.
\item deptree-Creeaza un dependency tree,listeaza doar pachete blackarch.
\item get-blackarch-deps-Ia o lista pentru blackarch dependencies pentru un pachet.
\item get-official-Ia pachete oficiale.
\item list-loose-packages-Listeaza pachete ce nu sunt in grupurisi nu au dependente de alte pachete.
\item list-needed-Listeaza dependente lipsa.
\item list-removed-Listeaza pachete ce sunt in  repo dar nu in git.
\item list-tools-Listeaza  unelte.
\item outdated-Verifica pachete out-dated in pachetul repo ca cel de pe git.
\item pkgmod-Modifica un pachet creat.
\item pkgrel-Include pkgrel intr-un pachet.
\item prep-Curata un PKGBUILD al unui fisier si gaseste erori.
\item sitesync-Sync o copie locala a unui pachet repo si o copie remote.
\item size-hunt-Cauta pachete mari.
\item source-backup-Face backup la pachete.
\end{itemize}

\section{Contribuie la proiect}
In aceasta sectiune va fi prezentat cum sa participi la proiectul BlackArch Linux. Aceptam cereri pull de toate dimensiunile, de la unele minore la pachete intregi.\\Pentru ajutor sau intrebari ne puteti contacta.

\\\\
Toate lumea poate participa. Orice contributie e aceptata.

\subsection{Tutoriale recomandate}
Va rugam sa cititi urmatoarele inainte de a ne ajuta:
\begin{itemize}
\item
\href{https://wiki.archlinux.org/index.php/Arch\_Packaging\_Standards)}{Arch
Packaging Standards}
\item \href{https://wiki.archlinux.org/index.php/Creating\_Packages}{Creating
Packages}
\item \href{https://wiki.archlinux.org/index.php/PKGBUILD}{PKGBUILD}
\item \href{https://wiki.archlinux.org/index.php/Makepkg}{Makepkg}
\end{itemize}

\subsection{Pasii pentru a contribui}
 Pentru a ne trimite schimbariile pentru proiectul BlackArchLinux, urmati urmatorii pasi:

\begin{enumerate}
\item Fork the repository from
\url{https://github.com/BlackArch/blackarch}
\item Modificati fisierele de care aveti nevoie, (e.g. PKGBUILD, .patch files, etc).
\item Pregatiti schimbariile.
\item Da-ti push la schimbari.
\item Spuneti-ne sa va aceptam schimbarile, preferabil printr-un pull.
\end{enumerate}

\subsection{Exemple}
Urmatorul exemplu demonstreaza cum se va trimite un nou pachet catre BlackArch. Folosim \href{https://wiki.archlinux.org/index.php/yaourt}{yaourt}
(poti folosi pacaur deasemenea) pentru a lua un pachet pre-existent fisierul PKGBUILD pentru \textbf{nfsshell} din \href{https://aur.archlinux.org/}{AUR} si modifical cum vrei

\subsubsection{Foloseste PKGBUILD}
Ia \textit{PKGBUILD} fisierul folosind yaourt sau pacaur:
\begin{lstlisting}
  user@blackarchlinux $ yaourt -G nfsshell
  ==> Download nfsshell sources
  x LICENSE
  x PKGBUILD
  x gcc.patch
  user@blackarchlinux $ cd nfsshell/
\end{lstlisting}

\subsubsection{Curata PKGBUILD}
Curata \textit{PKGBUILD} fisierul si salveaza:
\begin{lstlisting}
  user@blackarchlinux nfsshell $ ./blackarch/scripts/prep PKGBUILD
  cleaning 'PKGBUILD'...
  expanding tabs...
  removing vim modeline...
  removing id comment...
  removing contributor and maintainer comments...
  squeezing extra blank lines...
  removing '|| return'...
  removing leading blank line...
  removing $pkgname...
  removing trailing whitespace...
\end{lstlisting}

\subsubsection{Adjust PKGBUILD}
Ajunstaza \textit{PKGBUILD} fisierul:
\begin{lstlisting}
  user@blackarchlinux nfsshell $ vi PKGBUILD
\end{lstlisting}

\subsubsection{Build the package}
Construieste pachetul:
\begin{lstlisting}user@blackarchlinux nfsshell $ makepkg -sf
==> Making package: nfsshell 19980519-1 (Mon Dec  2 17:23:51 CET 2013)
==> Checking runtime dependencies...
==> Checking buildtime dependencies...
==> Retrieving sources...
-> Downloading nfsshell.tar.gz...
% Total    % Received % Xferd  Average Speed   Time    Time     Time
CurrentDload  Upload   Total   Spent    Left  Speed100 29213  100 29213    0
0  48150      0 --:--:-- --:--:-- --:--:-- 48206
-> Found gcc.patch
-> Found LICENSE
...
<lots of build process and compiler output here>
...
==> Leaving fakeroot environment.
==> Finished making: nfsshell 19980519-1 (Mon Dec  2 17:23:53 CET 2013)
\end{lstlisting}

\subsubsection{Instaleaza si testeaza pachetul}
Instaleaza si testeaza pachetul:
\begin{lstlisting}
  user@blackarchlinux nfsshell $ pacman -U nfsshell-19980519-1-x86_64.pkg.tar.xz
  user@blackarchlinux nfsshell $ nfsshell # test it
\end{lstlisting}

\subsubsection{Adauga, modifica si trimite pachetul}
Adauga, confirma si trimite pachetul
\begin{lstlisting}user@blackarchlinux nfsshell $ cd /blackarchlinux/packages
user@blackarchlinux ~/blackarchlinux/packages $ mv ~/nfsshell .
user@blackarchlinux ~/blackarchlinux/packages $ git commit -am nfsshell && git push
\end{lstlisting}

\subsubsection{Creeaza o cerere pull }
Creeaza o cerere pe \href{https://github.com/}{github.com}
\begin{lstlisting}
  firefox https://github.com/<contributor>/blackarchlinux
\end{lstlisting}

\subsubsection{Adauga un remote pentru upstream}
Un lucru istet de facut este sa dai fork si sa lucrezi remote .
\begin{lstlisting}
  user@blackarchlinux ~/blackarchlinux $ git remote -v
  origin <the url of your fork> (fetch)
  origin <the url of your fork> (push)
  user@blackarchlinux ~/blackarchlinux $ git remote add upstream https://github.com/blackarch/blackarch
  user@blackarchlinux ~/blackarchlinux $ git remote -v
  origin <the url of your fork> (fetch)
  origin <the url of your fork> (push)
  upstream https://github.com/blackarch/blackarch (fetch)
  upstream https://github.com/blackarch/blackarch (push)
\end{lstlisting}

In mod implicit, git ar trebui sa trimita direct la origine, dar ca sa te asiguri ca git config este configurat corect.Acest lucru nu ar trebui sa fie o problema decat daca trebuie sa trimiti drepturi si nu poti trimite upstream fara ele

Daca poti trimite, ai putea aveam mai mult success cu
\textit{git@github.com:blackarch/blackarch.git} dar e alegerea ta.

\subsection{Requests}
\begin{enumerate}
\item Nu adauga \textbf{Maintainer} sau \textbf{Contributor} comentari pentru fisierele
\textit{PKGBUILD} . Adauga utilizator pentru sectiune AUTHORS  BlackArch guide.
\item Pentru o mai buna intelegere, va rugam sa respectati  stilul tuturor
\textit{PKGBUILD} fisierelor din repo.
\end{enumerate}

\subsection{Recomandari}
\href{http://wiki.archlinux.org/index.php/Namcap}{namcap} poate verifica pentru erori.

%------------------%
%  Chapter 4       %
%------------------%

\chapter{Tools Guide}
In curand...

\section{Coming Soon}
In curand...

%%% APPENDIX %%%
\appendix
\include{latex/appendix-ro}

\end{document}

%%% EOF %%%
