%%%%%%%%%%%%%%%%%%%%%%%%%%%%%%%%%%%%%%%%%%%%%%%%%%%%%%%%%%%%%%%%%%%%%%%%%%%%%%%%
%                                                                              %
%                            BlackArch Linux Guide                             %
%                                 Spanish version                              %
%                                                   by Ubidragon               %
%                                                                              %
%%%%%%%%%%%%%%%%%%%%%%%%%%%%%%%%%%%%%%%%%%%%%%%%%%%%%%%%%%%%%%%%%%%%%%%%%%%%%%%%
%
%
%
%
%Traductors:
%Ubaldo Hidalgo Arriaga <ubidragon@protonmail.com>, 2016
%
%
%
%

\documentclass[a4paper, oneside, 11pt]{book}

%%% INCLUDES %%%
\renewcommand{\familydefault}{\sfdefault}

\usepackage{array}
\usepackage{color}
\usepackage{enumerate}
\usepackage{fancyhdr}
\usepackage{fancyvrb}
\usepackage{geometry}
\usepackage{graphicx}
\usepackage{html}
\usepackage{hyperref}
\usepackage{ifpdf}
\usepackage{listings}
\usepackage{pstricks}
\usepackage{supertabular}
\usepackage{tocloft}
\usepackage[utf8]{inputenc}

%%% LAYOUT %%%
\setlength{\parindent}{0em}
\setlength{\parskip}{1.5ex plus0.5ex minus0.5ex}
\geometry{left=2.5cm, textwidth=16cm, top=3cm, textheight=25cm, bottom=3cm}
\widowpenalty=2000
\clubpenalty=1000
\frenchspacing
\sloppy
\pagecolor[HTML]{FFFFFF}
\color[HTML]{333333}
\setcounter{tocdepth}{10}
\setcounter{secnumdepth}{10}

\hypersetup{
  pdftitle={BlackArch Linux, The BlackArch Linux Guide},
  pdfsubject={BlackArch Linux, The BlackArch Linux Guide},
  pdfauthor={BlackArch Linux, BlackArch Linux},
  pdfkeywords={BlackArch Linux, Penetration Testing, Security, Hacking, Linux},
  pdfcenterwindow=true,
  colorlinks=true,
  breaklinks=true,
  linkcolor=blue,
  menucolor=blue,
  urlcolor=blue
}

% syntax highlighting
% all options should be set here document wide
\lstset{
backgroundcolor=\color[HTML]{EEEEEE},
frame=single,
basicstyle=\footnotesize\ttfamily,
float,
deletekeywords={return},
otherkeywords={mkdir, curl, sudo, sha1sum, grep, cut, sort, wget, makepkg,
pacman, blackman, chmod},
keywordstyle=\color{orange},
commentstyle=\color{blue},
stringstyle=\color{red},
language=bash,
showspaces=false,
showtabs=false,
tabsize=2
}

%%% HEADER / FOOTER %%%
\setlength{\headheight}{33pt}
\setlength{\headsep}{33pt}
\lhead{{\includegraphics[width=1cm,height=1cm]{images/logo.png}}}
\rhead{The BlackArch Linux Guide}

%%% CUSTOM MACROS %%%
% for html links
\ifpdf\else
\def\href#1#2{\htmladdnormallink{#2}{#1}}
\fi

%------------------%
%  TITLE PAGE      %
%------------------%
\begin{document}
\pagestyle{empty}
\begin{center}
\begin{figure}[htbp]
\centering
\vspace{0.5cm}
\includegraphics[width=8cm]{images/logo.png}
\label{fig:logo}
\end{figure}
\vspace{0.5cm}
\Huge{\textbf{The BlackArch Linux Guide}}\\
\vspace{0.5cm}
\Large{\textbf{Spanish Version}}\\
\vspace{1cm}
\Large{\color{blue}https://www.blackarch.org/}\\
\vspace{0.5cm}
\end{center}
\newpage
\tableofcontents
\newpage
\pagestyle{fancy}

%------------------%
%  Capitulo 1      %
%------------------%

\chapter{Introducci\'on}

\section{Resumen}
The BlackArch Linux Guide se divide en varias partes:
\begin{itemize}
\item Introducci\'on - Proporciona una visi\'on general, introducci\'on, y una informaci\'on adicional sobre el proyecto
\item Gu\'ia del Usuario - Todo lo que un usuario necesita saber para poder utilizar con eficacia BlackArch
\item Gu\'ia del Desarrollador - ¿C\'omo empezar a desarrollar y contribuir para BlackArch?
\item Gu\'ia de Herramientas - Detalles en profundidad de las herramientas con ejemplos de uso (WIP)
\end{itemize}

\section{¿Qu\'e es BlackArch Linux?}
BlackArch es una completa distribuci\'on de Linux para penetration testers e investigadores de seguridad.
Es una distribuci\'on derivada de \href{https://www.archlinux.org/}{ArchLinux} por lo que los
usuarios podr\'an instalar componentes individuales o grupos de componentes directamente sobre
esta distribuci\'on.

La herramienta se distribuye como un Arch Linux
\href{https://wiki.archlinux.org/index.php/Unofficial\_User\_Repositories}
{repositorio no oficial} por lo que podemos instalar BlackArch sobre una instalaci\'on
existente de Arch Linux. Los paquetese se pueden instalar individualmente o por categorias.


El repositorio en constante expansi\'on incluye \href{https://www.blackarch.org/tools.html}{2600} herramientas.
Todas las herramientas son probadas antes de ser agregadas al c\'odigo base para mantener la calidad del repositorio.
% should quickly describe the testing methods/code review procedures etc

\section{Historia de BlackArch Linux}
Pr\'oximamente...

\section{Plataformas Soportadas}
Pr\'oximamente...

\section{Participaci\'on}
Usted puede ponerse en contacto con el equipo de BlackArch por los siguientes canales:

Website: \url{https://www.blackarch.org/}

Email: \href{mailto:team@blackarch.org}{team@blackarch.org}

IRC: \url{irc://irc.freenode.net/blackarch}

Twitter: \url{https://twitter.com/blackarchlinux}

Github: \url{https://github.com/Blackarch/}

Matrix: \url{https://matrix.to/#/#BlackArch:matrix.org}

%------------------%
%  Capitulo 2      %
%------------------%


\chapter{Gu\'ia del Usuario}

\section{Instalaci\'on}
Las siguientes secciones le mostrar\'an c\'omo configurar el repositorio de BlackArch e
instalar paquetes. BlackArch admite instalar desde el repositorio usando paquetes
binarios, as\'i como compilar e instalar de las fuentes.

BlackArch es compatible con una instalaci\'on normal de Arch. Su funcionamiento es
como un repositorio de usuario no oficial. En cambio si usted desea una ISO, mire 
en la secci\'on \href{https://www.blackarch.org/downloads.html#iso}{Live ISO}.

\subsection{Instalaci\'on sobre ArchLinux}
Ejecute \href{https://blackarch.org/strap.sh}{strap.sh} como administrador y sigue
las instrucciones. Mire el ejemplo siguiente.
\begin{lstlisting}
   curl -O https://blackarch.org/strap.sh
   sha1sum strap.sh # should match: 5ea40d49ecd14c2e024deecf90605426db97ea0c
   sudo chmod +x strap.sh
   sudo ./strap.sh
\end{lstlisting}
Ahora descarga una copia nueva de la lista principal de paquetes y paquetes
de sincronizaci\'on:
\begin{lstlisting}
  sudo pacman -Syyu
\end{lstlisting}


\subsection{Instalando Paquetes}
Ahora puede instalar las herramientas del repositorio de BlackArch.
\begin{enumerate}
\item Para ver una lista con todas las herramientas disponibles, ejecute 
\begin{lstlisting}
  pacman -Sgg | grep blackarch | cut -d' ' -f2 | sort -u
\end{lstlisting}

\item Para instalar todas las herramientas, ejecute
\begin{lstlisting}
  pacman -S blackarch
\end{lstlisting}

\item Para instalar una categor\'ia, ejecute
\begin{lstlisting}
  pacman -S blackarch-<category>
\end{lstlisting}

\item Para ver las categor\'ias de BlackArch, ejecute
\begin{lstlisting}
  pacman -Sg | grep blackarch
\end{lstlisting}

\end{enumerate}

\subsection{Instalando paquetes desde fuentes}
Como parte de un m\'etodo alternativo de instalaci\'on, usted puede construir los paquetes
BlackArchpackages desde las fuentes. Usted deber\'a buscar los PKGBUILDs en
\href{https://github.com/BlackArch/blackarch/tree/master/packages}{github}. Para construir
el repositorio entero, usted debr\'a usar la herramienta
\href{https://github.com/BlackArch/blackman}{Blackman}.
\begin{itemize}
\item En primer lugar, usted tiene que instalar Blackman. Si el repositorio de paquetes de BlackArch
est\'a configurado en su equipo, puede instalar Blackman:
\begin{lstlisting}
  pacman -S blackman
\end{lstlisting}

\item Usted puede compilar e instalar desde las fuentes:
\begin{lstlisting}
  mkdir blackman
  cd blackman
  wget https://raw.github.com/BlackArch/blackarch/master/packages/blackman/PKGBUILD
  # Aseg\'urese que PKGBUILD no ha sido manipulado maliciosamente.
  makepkg -s
\end{lstlisting}

\item Tambi\'en puede instalar Blackman desde AUR:
\begin{lstlisting}
  <cualquier ayudante de AUR que use> -S blackman
\end{lstlisting}

\end{itemize}

\subsection{Uso b\'asico de Blackman} Blackman es muy f\'acil de usar, aunque los 
comandos son diferentes de lo que normalmente podr\'ia esperar de algo parecido
a pacman. El uso b\'asico ser\'a delineado a continuaci\'on.
\begin{itemize}
\item Descargue, compile e instale paquetes:
\begin{lstlisting}
  sudo blackman -i package
\end{lstlisting}

\item Descargue, compile e instale alguna categor\'ia:
\begin{lstlisting}
  sudo blackman -g group
\end{lstlisting}

\item Descargue, compile e instale todas las herramientas de BlackArch:
\begin{lstlisting}
  sudo blackman -a
\end{lstlisting}

\item Para ver una lista de las categor\'ias de blackarch:
\begin{lstlisting}
  blackman -l
\end{lstlisting}

\item Para una lista de herramientas por categor\'ias:
\begin{lstlisting}
  blackman -p category
\end{lstlisting}

\end{itemize}

\subsection{Instalando desde live-, netinstall- ISO o ArchLinux}
Usted puede instalar BlackArch Linux desde uno de nuestros live- o netinstall-ISOs.\\Mire
\url{https://www.blackarch.org/download.html#iso}. Los pasos siguientes son necesarios
tras el arranque de la ISO.
\begin{itemize}
\item Instale el paquete blackarch-installer:
\begin{lstlisting}
  sudo pacman -S blackarch-installer
\end{lstlisting}

\item Ejecute
\begin{lstlisting}
  sudo blackarch-install
\end{lstlisting}

\end{itemize}

%------------------%
%  Capitulo 3      %
%------------------%

\chapter{Gu\'ia del Desarrollador}

\section{Sistema de compilaci\'on y repositorios de Arch}

Los archivos PKGBUILD son scripts de construcci\'on. Cada uno makepkg(1) dice c\'omo crear un
paquete. Los archivos PKGBUILD est\'an escritos en Bash.

Para obtener m\'as informaci\'on, lea (u hoje\'e) lo siguiente:
\begin{itemize}
\item \href{https://wiki.archlinux.org/index.php/Creating_Packages}{Arch Wiki: Creando Paquetes}
\item \href{https://wiki.archlinux.org/index.php/Makepkg}{Arch Wiki: makepkg}
\item \href{https://wiki.archlinux.org/index.php/PKGBUILD}{Arch Wiki: PKGBUILD}
\item \href{https://wiki.archlinux.org/index.php/Arch_Packaging_Standards}{Arch Wiki: Arch Packaging Standards}
\end{itemize}

\section{Blackarch PKGBUILD standards}
En favor de la simplicidad, nuestros PKGBUILDs son similares a los de los AUR,
con algunas pequeñas diferencias que se detallan a continuaci\'on. Cada paquete debe
pertenecer a Blackarch como m\'inimo, tambi\'en habr\'a mucho cruce con
m\'ultiples paquetes pertenecientes a m\'ultiples grupos.

\subsection{Grupos}
Para permitir a los usuarios instalar una gama espec\'ifica de paquetes de forma r\'apida y sencilla,
se han separado en grupos. Los grupos permiten a los usuarios simplemente
ir "pacman -S <nombre de grupo>" para sacar un mont\'on de paquetes.

\subsubsection{blackarch}
El grupo de blackarch es el grupo base donde todos los paquetes al que deben pertenecer tambi\'en. Esto permite que
los usuarios puedan instalar cada paquete con facilidad.

Qu\'e deber\'ia estar aqu\'i: Todo.

\subsubsection{blackarch-anti-forensic}
Son paquetes que se utilizan para contrarrestar las actividades forenses,
incluyendo cifrado, esteganograf\'ia y todo lo que modifica los atributos de archivos/archivos.
Todo esto incluye herramientas para trabajar con cualquier cosa en general que realice cambios a un sistema con
el fin de ocultar informaci\'on.

Ejemplos: luks, TrueCrypt, Timestomp, dd, ropeadope, secure-delete

\subsubsection{blackarch-automation}
Paquetes que se utilizan para la automatizaci\'on de herramientas o flujos de trabajo.

Ejemplos: blueranger, tiger, wiffy

\subsubsection{blackarch-backdoor}
Paquetes que explotan o abren puertas traseras en sistemas ya vulnerables.

Ejemplos: backdoor-factory, rrs, weevely

\subsubsection{blackarch-binary}
Paquetes que operan con archivos binarios de alguna forma.

Ejemplos: binwally, packerid

\subsubsection{blackarch-bluetooth}
Paquetes que explotan cualquier cosa relacionada con el est\'andar Bluetooth (802.15.1).

Ejemplos: ubertooth, tbear, redfang

\subsubsection{blackarch-code-audit}
Paquetes que auditan el c\'odigo fuente existente para el an\'alisis de vulnerabilidades.

Ejemplos: flawfinder, pscan

\subsubsection{blackarch-cracker}
Paquetes usados para crackear funciones criptogr\'aficas, i.e.: hashes.

Ejemplos: hashcat, john, crunch

\subsubsection{blackarch-crypto}
Paquetes que trabajan con la criptograf\'ia, a excepci\'on del cracking.

Ejemplos: ciphertest, xortool, sbd

\subsubsection{blackarch-database}
Paquetes que impliquen explotaci\'on de base de datos a cualquier nivel.

Ejemplos: metacoretex, blindsql

\subsubsection{blackarch-debugger}
Paquetes que permiten al usuario ver lo que un programa en particular est\'a "haciendo" en tiempo real.

Ejemplos: radare2, shellnoob

\subsubsection{blackarch-decompiler}
Paquetes que intentan revertir un programa compilado en c\'odigo fuente.

Ejemplos: flasm, jd-gui

\subsubsection{blackarch-defensive}
Paquetes que se utilizan para proteger a un usuario de malware y ataques de otros usuarios.

Ejemplos: arpon, chkrootkit, sniffjoke

\subsubsection{blackarch-disassembler}
Esto es similar a blackarch-decompiler, y probablemente habr\'a mucho
de programas que se encuentran en ambos, sin embargo, estos paquetes producen salida en ensamblador en lugar del c\'odigo fuente en bruto.

Ejemplos: inguma, radare2

\subsubsection{blackarch-dos}
Paquetes que utilizan ataques de denegaci\'on de servicio (DoS).

Ejemplos: 42zip, nkiller2

\subsubsection{blackarch-drone}
Los paquetes que se utilizan para la gesti\'on de la ingenier\'ia f\'isica de drones.

Ejemplos: meshdeck, skyjack

\subsubsection{blackarch-exploitation}
Paquetes que aprovechan las ventajas de los exploits en otros programas o servicios.

Ejemplos: armitage, metasploit, zarp

\subsubsection{blackarch-fingerprint}
Paquetes que explotan los equipos biom\'etricos de huellas dactilares.

Ejemplos: dns-map, p0f, httprint

\subsubsection{blackarch-firmware}
Paquetes que aprovechan las vulnerabilidades del firmware

Ejemplos: None yet, amend asap.

\subsubsection{blackarch-forensic}
Paquetes que se utilizan para encontrar datos en discos f\'isicos o memoria incrustada.

Ejemplos: aesfix, nfex, wyd

\subsubsection{blackarch-fuzzer}
Paquetes que utilizan el principio de prueba de la polic\'ia, es decir
"lanzar" entradas aleatorias al sujeto para ver qu\'e pasa.

Ejemplos: msf, mdk3, wfuzz

\subsubsection{blackarch-hardware}
Paquetes que explotan o gestionan cualquier cosa que tenga que ver con
hardware f\'isico.

Ejemplos: arduino, smali

\subsubsection{blackarch-honeypot}
Los paquetes que act\'uan como "honeypots", es decir, programas que parecen
ser servicios vulnerables utilizados para atraer a los hackers a una trampa.

Ejemplos: artillery, bluepot, wifi-honey

\subsubsection{blackarch-keylogger}
Paquetes que registran y conservan las pulsaciones de teclas en otro sistema.

Ejemplos: None yet, amend asap.

\subsubsection{blackarch-malware}
Paquetes que cuentan como cualquier tipo de software malicioso o
detecci\'on de malware.

Ejemplos: malwaredetect, peepdf, yara

\subsubsection{blackarch-misc}
Paquetes que no encajan particularmente en ninguna categor\'ia.

Ejemplos: oh-my-zsh-git, winexe, stompy

\subsubsection{blackarch-mobile}
Paquetes que manipulan plataformas m\'oviles.

Ejemplos: android-sdk-platform-tools, android-udev-rules

\subsubsection{blackarch-networking}
Paquete que involucra redes IP.

Ejemplos: Anything pretty much

\subsubsection{blackarch-nfc}
Paquetes que utilizan nfc (comunicaciones de campo cercano).

Ejemplos: nfcutils

\subsubsection{blackarch-packer}
Paquetes que operan o invaden a los empaquetadores.

\textit{son programas que incrustan malware dentro de otros ejecutables.}

Ejemplos: packerid

\subsubsection{blackarch-proxy}
Paquetes que act\'uan como proxy, es decir, redirigiendo el tr\'afico
a trav\'es de otro nodo en Internet.

Ejemplos: burpsuite, ratproxy, sslnuke

\subsubsection{blackarch-recon}
Paquetes que buscan activamente vulnerabilidades a
lo salvaje. M\'as bien un grupo de cobertura para paquetes similares.

Ejemplos: canri, dnsrecon, netmask

\subsubsection{blackarch-reversing}
Este es un grupo de cobertura para cualquier descompilador,
desensamblador o cualquier otro programa similar.

Ejemplos: capstone, radare2, zerowine

\subsubsection{blackarch-scanner}
Paquetes que analizan los sistemas seleccionados en busca de vulnerabilidades.

Ejemplos: scanssh, tiger, zmap

\subsubsection{blackarch-sniffer}
Paquetes que involucran el an\'alisis del tr\'afico de la red.

Ejemplos: hexinject, pytactle, xspy

\subsubsection{blackarch-social}
Paquetes que atacan principalmente los sitios de redes sociales.

Ejemplos: jigsaw, websploit

\subsubsection{blackarch-spoof}
Paquetes que intentan suplantar al atacante, tales como
el atacante no parecer como atacante a la v\'ictima.

Ejemplos: arpoison, lans, netcommander

\subsubsection{blackarch-threat-model}
Paquetes que se utilizar\'ian para informar/grabar el
modelo de amenaza descrito en un escenario particular.

Ejemplos: magictree

\subsubsection{blackarch-tunnel}
Los paquetes que se utilizan para tunelizar el tr\'afico de red en una red determinada
red.
Ejemplos: ctunnel, iodine, ptunnel

\subsubsection{blackarch-unpacker}
Los paquetes que se utilizan para extraer malware preempaquetado de un archivo
ejecutable.

Ejemplos: js-beautify

\subsubsection{blackarch-voip}
Paquetes que operan con programas y protocolos voip.

Ejemplos: iaxflood, rtp-flood, teardown

\subsubsection{blackarch-webapp}
Paquetes que funcionan con aplicaciones orientadas a Internet.

Ejemplos: metoscan, whatweb, zaproxy

\subsubsection{blackarch-windows}
Este grupo es para cualquier paquete nativo de Windows que se ejecute v\'ia wine.

Ejemplos: 3proxy-win32, pwdump, winexe

\subsubsection{blackarch-wireless}
Paquetes que operan en redes inal\'ambricas en cualquier nivel.

Ejemplos: airpwn, mdk3, wiffy

\section{Estructura del repositorio}
Usted puede encontrar el repositorio git principal de  BlackArch aqu\'i:
\href{https://github.com/BlackArch/blackarch}{https://github.com/BlackArch/blackarch}.
Tambi\'en hay varios repositorios secundarios aqu\'i:
\href{https://github.com/BlackArch}{https://github.com/BlackArch}.

En el repositorio git principal, hay tres directorios importantes:

\begin{itemize}
\item docs - Documentaci\'on.
\item packages - archivos PKGBUILD.
\item scripts - Scripts poco \'utiles.
\end{itemize}

\subsection{Scripts}
Aqu\'i hay una referencia para los scripts en el directorio \verb|scripts/| :

\begin{itemize}
\item baaur -  Pronto, esto subir\'a paquetes al AUR.
\item babuild - Construye un paquete.
\item bachroot - Administrar un chroot para pruebas.
\item baclean - Limpie los archivos.pkg.tar.xz antiguos del paquete repo.
\item baconflict - Pronto esto reemplazar\'a a \verb|scripts/conflicts|.
\item bad-files - Encuentra archivos defectuosos en los paquetes construidos.
\item balock - Obtenga o libere el bloqueo de reposici\'on de paquetes.
\item banotify - Notificar a IRC sobre el envio de paquetes.
\item barelease - Liberar paquetes al repositorio de paquetes.
\item baright - Imprimir la informaci\'on de copyright de BlackArch.
\item basign -  Firmar paquetes.
\item basign-key - Firmar una clave.
\item blackman - Esto se comporta como pacman pero construye a partir de git (no para de ser
    confundido con el Blackman de nrz).
\item check-groups - Verifica grupos.
\item checkpkgs - Verifica paquetes con errores.
\item conflicts - Compruebe si hay conflictos de archivos.
\item dbmod - Modificar una base de datos de paquetes.
\item depth-list - Cree una lista ordenada por la profundidad de la relaci\'on.
\item deptree - Cree un \'arbol de dependencias con una lista de solo los paquetes proporcionados por Blackarch.
\item get-blackarch-deps - Obtenga una lista de dependencias de Blackarch para un paquete.
\item get-official - Obtenga paquetes oficiales para su publicaci\'on.
\item list-loose-packages - Lista de paquetes que no est\'an en grupos y que no est\'an
    para otros paquetes.
\item list-needed - Enumerar las dependencias que faltan.
\item list-removed - Lista los paquetes que est\'an en el repositorio de paquetes pero no en git.
\item list-tools - Lista de herramientas.
\item outdated - Busque paquetes que est\'en desactualizados en el repositorio de paquetes.
    en relaci\'on con el git repo.
\item pkgmod - Modificar un paquete de compilaci\'on.
\item pkgrel - Incrementar pkgrel en un paquete.
\item prep - Limpia el estilo de un archivo PKGBUILD y encuentra errores.
\item sitesync - Sincronizaci\'on entre una copia local del repositorio de paquetes y una copia remota.
\item size-hunt - B\'usqueda de paquetes grandes.
\item source-backup - Haga copias de seguridad de los archivos de origen de los paquetes.
\end{itemize}

\section{Contributing to repository}
Esta secci\'on muestra c\'omo contribuir al proyecto BlackArch Linux. Nosotros aceptamos pull request de todos los tamaños, desde pequeñas correcciones de errores tipogr\'aficos hasta nuevos paquetes.
\\Para ayudar, sugerencias o preguntas no dude en ponerse en contacto con nosotros.
\\\\
Todos son Bienvenidos a contribuir. Todas las contribuciones son apreciadas.

\subsection{Tutoriales necesarios}
Por favor lea los siguientes tutorial antes de empezar a colaborar:
\begin{itemize}
\item
\href{https://wiki.archlinux.org/index.php/Arch\_Packaging\_Standards)}{Normas
de empaquetado de Arch}
\item \href{https://wiki.archlinux.org/index.php/Creating\_Packages}{Creaci\'on 
de Paquetes}
\item \href{https://wiki.archlinux.org/index.php/PKGBUILD}{PKGBUILD}
\item \href{https://wiki.archlinux.org/index.php/Makepkg}{Makepkg}
\end{itemize}

\subsection{Steps for contributing}
Para enviar sus cambios al proyecto BlackArchLinux, siga estos pasos
pasos:
\begin{enumerate}
\item Fork al repositorio origen
\url{https://github.com/BlackArch/blackarch}
\item Modifique los archivos necesarios (por ejemplo, PKGBUILD, archivos.patch, etc.).
\item Commit a tus cambios.
\item Sube tus cambios.
\item P\'idanos que fusionemos sus cambios, preferiblemente a trav\'es de una solicitud de pull request
\end{enumerate}

\subsection{Ejemplo}
El siguiente ejemplo muestra c\'omo enviar un nuevo paquete a BlackArch
proyecto. Utilizamos \href{https://wiki.archlinux.org/index.php/yaourt}{yaourt}
(tambi\'en puede usar pacaur) para obtener un archivo PKGBUILD preexistente 
\textbf{nfsshell} de \href{https://aur.archlinux.org/}{AUR} y ajustarlo de acuerdo a nuestras necesidades.

\subsubsection{Sincronizar PKGBUILD}
Sincronizar el fichero \textit{PKGBUILD} usando yaourt o pacaur:
\begin{lstlisting}
  user@blackarchlinux $ yaourt -G nfsshell
  ==> Download nfsshell sources
  x LICENSE
  x PKGBUILD
  x gcc.patch
  user@blackarchlinux $ cd nfsshell/
\end{lstlisting}

\subsubsection{Limpiar PKGBUILD}
Limpia el archivo \textit{PKGBUILD} y ahorra tiempo:
\begin{lstlisting}
  user@blackarchlinux nfsshell $ ./blackarch/scripts/prep PKGBUILD
  cleaning 'PKGBUILD'...
  expanding tabs...
  removing vim modeline...
  removing id comment...
  removing contributor and maintainer comments...
  squeezing extra blank lines...
  removing '|| return'...
  removing leading blank line...
  removing $pkgname...
  removing trailing whitespace...
\end{lstlisting}

\subsubsection{Ajustar PKGBUILD}
Ajustar el archivo \textit{PKGBUILD} :
\begin{lstlisting}
  user@blackarchlinux nfsshell $ vi PKGBUILD
\end{lstlisting}

\subsubsection{Construcci\'on del paquete}
Construcci\'on del paquete:
\begin{lstlisting}user@blackarchlinux nfsshell $ makepkg -sf
==> Making package: nfsshell 19980519-1 (Mon Dec  2 17:23:51 CET 2013)
==> Checking runtime dependencies...
==> Checking buildtime dependencies...
==> Retrieving sources...
-> Downloading nfsshell.tar.gz...
% Total    % Received % Xferd  Average Speed   Time    Time     Time
CurrentDload  Upload   Total   Spent    Left  Speed100 29213  100 29213    0
0  48150      0 --:--:-- --:--:-- --:--:-- 48206
-> Found gcc.patch
-> Found LICENSE
...
<lots of build process and compiler output here>
...
==> Leaving fakeroot environment.
==> Finished making: nfsshell 19980519-1 (Mon Dec  2 17:23:53 CET 2013)
\end{lstlisting}

\subsubsection{Instalar y probar el paquete}
Instalar y probar el paquete:
\begin{lstlisting}
  user@blackarchlinux nfsshell $ pacman -U nfsshell-19980519-1-x86_64.pkg.tar.xz
  user@blackarchlinux nfsshell $ nfsshell # test it
\end{lstlisting}

\subsubsection{dAd, commit and push package}
Add, commit y push el paquete
\begin{lstlisting}user@blackarchlinux nfsshell $ cd /blackarchlinux/packages
user@blackarchlinux ~/blackarchlinux/packages $ mv ~/nfsshell .
user@blackarchlinux ~/blackarchlinux/packages $ git commit -am nfsshell && git push
\end{lstlisting}

\subsubsection{Crea un pull request}
Crea una pull request en \href{https://github.com/}{github.com}
\begin{lstlisting}
  firefox https://github.com/<contributor>/blackarchlinux
\end{lstlisting}

\subsubsection{Adding a remote for upstream}
Una cosa inteligente a hacer si usted est\'a trabajando upstream y en un fork es tirar de su propio fork y añadir el repositorio principal como remoto
\begin{lstlisting}
  user@blackarchlinux ~/blackarchlinux $ git remote -v
  origin <the url of your fork> (fetch)
  origin <the url of your fork> (push)
  user@blackarchlinux ~/blackarchlinux $ git remote add upstream https://github.com/blackarch/blackarch
  user@blackarchlinux ~/blackarchlinux $ git remote -v
  origin <the url of your fork> (fetch)
  origin <the url of your fork> (push)
  upstream https://github.com/blackarch/blackarch (fetch)
  upstream https://github.com/blackarch/blackarch (push)
\end{lstlisting}

Por defecto, git deber\'ia empujar directamente al origen, pero aseg\'urate de que tu configuraci\'on de git est\'a correctamente. Esto no ser\'a un problema a menos que tengas derechos como
no ser\'as capaz de empujar upstream sin ellos.

Si tienes la capacidad de comprometerte, puedes tener m\'as \'exito usando
\textit{git@github.com:blackarch/blackarch.git} pero depende de ti.

\subsection{Solicitudes}
\begin{enumerate}
\item \textbf{Responsable} o \textbf{Colaborador} no añada comentarios en los
archivos \textit{PKGBUILD}. Añada sus nombres de responsable y colaborador
en la seccion de AUTORES de la BlackArch guide.
\item En favor de la coherencia, por favor, siga el estilo general de la otra
\textit{PKGBUILD} en el repositorio y usar la sangr\'ia de dos espacios.
\end{enumerate}


\subsection{Consejos generales}
\href{http://wiki.archlinux.org/index.php/Namcap}{namcap} puede comprobar si 
los paquetes contienen errores.

%------------------%
%  Capitulo 4      %
%------------------%

\chapter{Gu\'ia de Herramientas}
Pr\'oximamente...

\section{Pr\'oximamente...}
Pr\'oximamente...

%%% Apendice %%%
\appendix
\include{latex/appendix-es}

\end{document}

%%% EOF %%%
