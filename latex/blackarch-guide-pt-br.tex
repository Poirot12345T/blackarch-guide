%%%%%%%%%%%%%%%%%%%%%%%%%%%%%%%%%%%%%%%%%%%%%%%%%%%%%%%%%%%%%%%%%%%%%%%%%%%%%%%%
%                                                                              %
% BlackArch Linux Guide                                                        %
%                                                                              %
%%%%%%%%%%%%%%%%%%%%%%%%%%%%%%%%%%%%%%%%%%%%%%%%%%%%%%%%%%%%%%%%%%%%%%%%%%%%%%%%

\documentclass[a4paper, oneside, 11pt]{book}

%%% INCLUDES %%%
\renewcommand{\familydefault}{\sfdefault}

\usepackage{array}
\usepackage{color}
\usepackage{enumerate}
\usepackage{fancyhdr}
\usepackage{fancyvrb}
\usepackage{geometry}
\usepackage{graphicx}
\usepackage{html}
\usepackage{hyperref}
\usepackage{ifpdf}
\usepackage{listings}
\usepackage{pstricks}
\usepackage{supertabular}
\usepackage{tocloft}


\usepackage[brazilian]{babel}
\usepackage[utf8]{inputenc}
\usepackage[T1]{fontenc}


%%% LAYOUT %%%
\setlength{\parindent}{0em}
\setlength{\parskip}{1.5ex plus0.5ex minus0.5ex}
\geometry{left=2.5cm, textwidth=16cm, top=3cm, textheight=25cm, bottom=3cm}
\widowpenalty=2000
\clubpenalty=1000
\frenchspacing
\sloppy
\pagecolor[HTML]{FFFFFF}
\color[HTML]{333333}
\setcounter{tocdepth}{10}
\setcounter{secnumdepth}{10}

\hypersetup{
  pdftitle={BlackArch Linux, Guia BlackArch Linux},
  pdfsubject={BlackArch Linux, Guia BlackArch Linux},
  pdfauthor={BlackArch Linux, BlackArch Linux},
  pdfkeywords={BlackArch Linux, Penetration Testing, Security, Hacking, Linux},
  pdfcenterwindow=true,
  colorlinks=true,
  breaklinks=true,
  linkcolor=blue,
  menucolor=blue,
  urlcolor=blue
}

% syntax highlighting
% all options should be set here document wide
\lstset{
backgroundcolor=\color[HTML]{EEEEEE},
frame=single,
basicstyle=\footnotesize\ttfamily,
float,
deletekeywords={return},
otherkeywords={mkdir, curl, sudo, sha1sum, grep, cut, sort, wget, makepkg,
pacman, blackman, chmod},
keywordstyle=\color{orange},
commentstyle=\color{blue},
stringstyle=\color{red},
language=bash,
showspaces=false,
showtabs=false,
tabsize=2
}

%%% HEADER / FOOTER %%%
\setlength{\headheight}{33pt}
\setlength{\headsep}{33pt}
\lhead{{\includegraphics[width=1cm,height=1cm]{images/logo.png}}}
\rhead{Guia BlackArch Linux}

%%% CUSTOM MACROS %%%
% for html links
\ifpdf\else
\def\href#1#2{\htmladdnormallink{#2}{#1}}
\fi

%------------------%
%  TITLE PAGE      %
%------------------%
\begin{document}
\pagestyle{empty}
\begin{center}
\begin{figure}[htbp]
\centering
\vspace{0.5cm}
\includegraphics[width=8cm]{images/logo.png}
\label{fig:logo}
\end{figure}
\vspace{0.5cm}
\Huge{\textbf{Guia BlackArch Linux}}\\
\vspace{1cm}
\Large{\color{red}https://www.blackarch.org/}\\
\vspace{0.5cm}
\end{center}
\newpage
\tableofcontents
\newpage
\pagestyle{fancy}

%------------------%
%  Chapter 1       %
%------------------%

\chapter{Introdução}

\section{Resumo}
O guia do BlackArch linux é dividido em algumas partes:
\begin{itemize}
\item Introdução - Apresenta uma visão geral e informações adicionais sobre o projeto
\item Guia do Usuário - Tudo que o usuário precisa saber para usar o BlackArch de forma efetiva
\item Guia do Desenvolvedor - Como começar a desenvolver e contribuir para o BlackArch
\item Guia das Ferramentas - Detalhes e exemplos de uso das ferramentas
\end{itemize}

\section{O que é o BlackArch Linux?}
BlackArch é uma distribuição Linux para testes de penetração e pesquisadores de segurança.É uma derivação do \href{https://www.archlinux.org/}{ArchLinux} e os usuários podem instalar componentes do BlackArch individualmente ou por grupos em cima da distribuição.
As ferramentas são distribuídas através do Arch Linux
\href{https://wiki.archlinux.org/index.php/Unofficial\_User\_Repositories}
{unofficial user repository} então você pode instalar o BlackArch em cima de uma instalação Arch Linux já existente.Os pacotes pode ser instalados individualmente ou por categorias.

O repositório continua aumentando e já possui  \href{https://www.blackarch.org/tools.html}{2600} ferramentas.
Todas as ferramentas são testadas depois da adição de algum codigo mantendo a qualidade do repositório.
% should quickly describe the testing methods/code review procedures etc

\section{A historia do BlackArch Linux}
Em breve...

\section{Plataformas suportadas}
Em breve...

\section{Junte-se ao BlackArch}
Você pode se comunicar com a equipe BlackArch através das formas abaixo:

Website: \url{https://www.blackarch.org/}

Mail: \href{mailto:team@blackarch.org}{team@blackarch.org}

IRC: \url{irc://irc.freenode.net/blackarch}

Twitter: \url{https://twitter.com/blackarchlinux}

Github: \url{https://github.com/Blackarch/}

Matrix: \url{https://matrix.to/#/#BlackArch:matrix.org}

%------------------%
%  Chapter 2       %
%------------------%


\chapter{Guia do Usuário}

\section{Instalação} 
A seção seguinte mostra como configurar o repositório BlackArch e instalar os pacotes.
O BlackArch suporta a instalação direta pelo repositório usando pacotes binários ou compilando, como também através de outras fontes.
O BlackArch é compatível como toda instalação do Arch. A instalação é feita através de um repositório não oficial, se você quiser usar uma ISO olhe na seção \href{https://www.blackarch.org/downloads.html#iso}{Live ISO}.

\subsection{Instalando em cima do ArchLinux}
Execute \href{https://blackarch.org/strap.sh}{strap.sh} como root e siga as instruções abaixo.
\begin{lstlisting}
   curl -O https://blackarch.org/strap.sh
   sha1sum strap.sh # Deve ser igual a: 5ea40d49ecd14c2e024deecf90605426db97ea0c
   sudo chmod +x strap.sh
   sudo ./strap.sh
\end{lstlisting}
Agora baixe uma copia atualizada da lista de pacotes e os sincronize.
\begin{lstlisting}
  sudo pacman -Syyu
\end{lstlisting}


\subsection{Instalando os pacotes}
Agora você pode instalar os pacotes do repositório do blackarch.
\begin{enumerate} 
\item Para listar todas as ferramentas disponíveis execute
\begin{lstlisting}
  pacman -Sgg | grep blackarch | cut -d' ' -f2 | sort -u
\end{lstlisting}

\item Para instalar todas a ferramentas execute
\begin{lstlisting}
  pacman -S blackarch
\end{lstlisting}

\item Para instalar toda uma categoria execute
\begin{lstlisting}
  pacman -S blackarch-<category>
\end{lstlisting}

\item Para ver as categorias execute
\begin{lstlisting}
  pacman -Sg | grep blackarch
\end{lstlisting}

\end{enumerate}

\subsection{Instalando os pacotes pelo codigo fonte}
De maneira alternativa você pode criar os pacotes do BlackArch diretamente do codigo fonte. Você pode achar os PKGBUILDs no
\href{https://github.com/BlackArch/blackarch/tree/master/packages}{github}. Para criar todo o repositório você pode usar a ferramenta
\href{https://github.com/BlackArch/blackman}{Blackman}.
\begin{itemize}
\item Primeramente, você tem que instalar o Blackman. Se o repositório do BlackArch está configurado na sua máquina, você pode instalar o Blackman:
\begin{lstlisting}
  pacman -S blackman
\end{lstlisting}

\item Você pode construir e instalar o Blackman pelo código fonte:
\begin{lstlisting}
  mkdir blackman
  cd blackman
  wget https://raw.github.com/BlackArch/blackarch/master/packages/blackman/PKGBUILD
  # Tenha certeza que o PKGBUILD nao possui codigo malicioso.
  makepkg -s
\end{lstlisting}

\item Ou você pode instalar o Blackman pelo AUR::
\begin{lstlisting}
  <quaquer ajudante do AUR> -S blackman
\end{lstlisting} 

\end{itemize}

\subsection{Uso básico do Blackman} Blackman possui um uso simples, seus parâmetros são um pouco diferentes do que você esperaria de algo como o pacman. O básico será mostrado abaixo.
\begin{itemize}
\item Baixa, compila e instala o pacote:
\begin{lstlisting}
  sudo blackman -i pacote
\end{lstlisting}

\item Baixa, compila e instala toda uma categoria:
\begin{lstlisting}
  sudo blackman -g grupo
\end{lstlisting}

\item Baixa, compila e instala todas as ferramentas do BlackArch:
\begin{lstlisting}
  sudo blackman -a
\end{lstlisting}

\item Lista todas as categorias de ferramentas:
\begin{lstlisting}
  blackman -l
\end{lstlisting}

\item Lista uma categoria de ferramentas:
\begin{lstlisting}
  blackman -p categoria 
\end{lstlisting}

\end{itemize}

\subsection{Instalação pelo live-, netinstall- ISO ou ArchLinux}
Você pode instalar o BlackArch através de um live- ou netinstall-ISOs.\\Veja em
\url{https://www.blackarch.org/download.html#iso}.Os passos a seguir são executados depois da inicialização da ISO.
\begin{itemize}
\item Instalando o pacote blackarch-installer:
\begin{lstlisting}
  sudo pacman -S blackarch-installer
\end{lstlisting}

\item Execute
\begin{lstlisting}
  sudo blackarch-install
\end{lstlisting}

\end{itemize}

%------------------%
%  Chapter 3       %
%------------------%

\chapter{Guia do Desenvolvedor}

\section{Construir sistemas e repositórios Arch}

Os PKGBUILD são scripts de "construção". Cada um diz ao makepkg(1) como construir o pacote. Os arquivos PKGBUILD são escritos em Bash.

Para mais informações, ler abaixo:
\begin{itemize}
\item \href{https://wiki.archlinux.org/index.php/Creating_Packages}{Arch Wiki: Creating Packages}
\item \href{https://wiki.archlinux.org/index.php/Makepkg}{Arch Wiki: makepkg}
\item \href{https://wiki.archlinux.org/index.php/PKGBUILD}{Arch Wiki: PKGBUILD}
\item \href{https://wiki.archlinux.org/index.php/Arch_Packaging_Standards}{Arch Wiki: Arch Packaging Standards}
\end{itemize}

\section{Padrão PKGBUILD do Blackarch}
Por sua simplicidade, os PKGBUILDs são similares aos do AUR, como algumas pequenas diferenças como mostrado a baixo. Cada pacote deve pertencer ao blackarch o minimo possível, e deve haver o máximo de crossover com vários pacotes percentes a vários grupos.
\subsection{Grupos}
Permite que os usuários instalem um tipo específico de pacote de forma rápida e fácil, os pacotes são separados por grupos. Os grupos permitem que o usuário digite "pacman -S <group name>" e instale vários pacotes de uma vez.

\subsubsection{blackarch}
O grupo blackarch é o grupo base onde todos os pacotes se encontram. Isto permite que o usuário instale todos os pacotes de uma só vez com facilidade.

O que deve estar aqui: Tudo.

\subsubsection{blackarch-anti-forensic}
Pacotes relacionados ao uso de anti-forense, incluindo encriptação, estenografia, e tudo para modificar os atributos de arquivo/arquivos.
Inclui todas as ferramentas para trabalhar com tudo em geral para modificar um sistema com o propósito de esconder uma informação.

Exemplo: luks, TrueCrypt, Timestomp, dd, ropeadope, secure-delete

\subsubsection{blackarch-automation}
Pacotes para o uso de ferramentas ou automatização de tarefas.

Exemplo: blueranger, tiger, wiffy

\subsubsection{blackarch-backdoor}
Pacotes para explorar ou abrir blackdoors em sistemas vulneráveis.

Exemplo: backdoor-factory, rrs, weevely

\subsubsection{blackarch-binary}
Pacotes para modificar arquivos binários de algum modo.

Exemplo: binwally, packerid

\subsubsection{blackarch-bluetooth}
Pacotes para explorar tudo relacionado ao Bluetooth padrão (802.15.1).

Exemplo: ubertooth, tbear, redfang

\subsubsection{blackarch-code-audit}
Pacotes para analisar um código em busca de vulnerabilidades.

Exemplo: flawfinder, pscan

\subsubsection{blackarch-cracker}
Pacotes usados para quebrar criptografia.

Exemplo: hashcat, john, crunch

\subsubsection{blackarch-crypto}
Pacotes para trabalhar com criptografia sem ser em sua quebra.

Exemplo: ciphertest, xortool, sbd

\subsubsection{blackarch-database}
Pacotes que envolvam exploração de banco de dados em algum nível.

Exemplo: metacoretex, blindsql

\subsubsection{blackarch-debugger}
Pacotes para permitir que o usuário veja o que um programa em particular esta "fazendo" em tempo real.

Exemplo: radare2, shellnoob

\subsubsection{blackarch-decompiler}
Pacotes voltados para converter um programa compilado em código fonte.

Exemplo: flasm, jd-gui

\subsubsection{blackarch-defensive}
Pacotes para proteger o usuário de malware \& ataques de outros usuários.

Exemplos: arpon, chkrootkit, sniffjoke

\subsubsection{blackarch-disassembler}
Similar ao blackarch-decompiler, possui semelhança em alguns pacotes mas esses são mais voltados a produzir uma saída em assembly, não em código fonte.

Exemplo: inguma, radare2

\subsubsection{blackarch-dos}
Pacotes para o uso de ataques DoS (Denial of Service).

Exemplo: 42zip, nkiller2

\subsubsection{blackarch-drone}
Pacotes usados para o manejo físico de engenharia de drones.

Exemplo: meshdeck, skyjack

\subsubsection{blackarch-exploitation}
Pacotes para explorar outros programas ou serviços.

Exemplo: armitage, metasploit, zarp

\subsubsection{blackarch-fingerprint}
Pacotes para explorar equipamentos de biometria digital.

Exemplo: dns-map, p0f, httprint

\subsubsection{blackarch-firmware}
Pacotes para explorar vulnerabilidades no firmware

Exemplo: None yet, amend asap.

\subsubsection{blackarch-forensic}
Pacotes usados para procurar informação em discos físicos ou memória.

Exemplo: aesfix, nfex, wyd

\subsubsection{blackarch-fuzzer}
Pacotes que usam o princípio do teste fuzz, "jogando" números aleatórios com o objetivo de ver o que acontece.

Exemplo: msf, mdk3, wfuzz

\subsubsection{blackarch-hardware}
Pacotes para explorar ou manejar o hardware.

Exemplo: arduino, smali

\subsubsection{blackarch-honeypot}
Pacotes que agem como "honeypots" (pote de mel), programas para simular uma vulnerabilidade e cria uma armadilha para os hackers.

Exemplo: artillery, bluepot, wifi-honey

\subsubsection{blackarch-keylogger}
Pacotes para gravar o digito em outros sistemas.

Exemplo: None yet, amend asap.

\subsubsection{blackarch-malware}
Pacotes para qualquer tipo de software malicioso ou detecção de
malware.

Exemplo: malwaredetect, peepdf, yara

\subsubsection{blackarch-misc}
Pacotes com nada em particular para encaixar em uma categoria.

Exemplo: oh-my-zsh-git, winexe, stompy

\subsubsection{blackarch-mobile}
Pacotes para manipulação de sistemas móveis.

Exemplo: android-sdk-platform-tools, android-udev-rules

\subsubsection{blackarch-networking}
Pacotes que envolvam redes IP.

Exemplo: Praticamente tudo

\subsubsection{blackarch-nfc}
Pacotes que usam nfc (near-field communications).

Exemplo: nfcutils

\subsubsection{blackarch-packer}
Pacotes que operam ou envolvem packers.

\textit{packers são programas que juntam malware com outros executáveis.}

Exemplo: packerid

\subsubsection{blackarch-proxy}
Pacotes que agem com proxy, como redirecionar o tráfego para outro nó na internet.

Exemplo: burpsuite, ratproxy, sslnuke

\subsubsection{blackarch-recon}
Pacotes que procuram exploits. Mais no grupo umbrella para pacotes similares.

Exemplo: canri, dnsrecon, netmask

\subsubsection{blackarch-reversing}
Este é um grupo umbrella para todo decompilador, disassembler ou programas similares.

Exemplo: capstone, radare2, zerowine

\subsubsection{blackarch-scanner}
Pacotes que scaneiam um sistema em busca de vulnerabilidades.

Exemplo: scanssh, tiger, zmap

\subsubsection{blackarch-sniffer}
Pacotes que envolvem analise de tráfego de rede.

Exemplo: hexinject, pytactle, xspy

\subsubsection{blackarch-social}
Pacotes para ataque em redes sociais.

Exemplo: jigsaw, websploit

\subsubsection{blackarch-spoof}
Pacotes para falsificar um ataque, onde um ataque não é mostrado como uma ameaça a vítima.

Exemplo: arpoison, lans, netcommander

\subsubsection{blackarch-threat-model}
Pacotes que são usados para reportar/gravar um modelo de ameaça em um cenário particular.

Exemplo: magictree

\subsubsection{blackarch-tunnel}
Pacotes que são usados para criar túneis no tráfego de rede.

Exemplo: ctunnel, iodine, ptunnel

\subsubsection{blackarch-unpacker}
Pacotes que são usados para extrair um malware de um executável.

Exemplo: js-beautify

\subsubsection{blackarch-voip}
Pacotes que operam em programas e protocolos VoIP.

Exemplo: iaxflood, rtp-flood, teardown

\subsubsection{blackarch-webapp}
Pacotes que operam em aplicações na internet.

Exemplo: metoscan, whatweb, zaproxy

\subsubsection{blackarch-windows}
Este grupo é para todo pacote nativo para Windows que é executado através do wine.

Exemplo: 3proxy-win32, pwdump, winexe

\subsubsection{blackarch-wireless}
Pacotes que operam em rede wireless em algum nível.

Exemplo: airpwn, mdk3, wiffy

\section{Estrutura do repositório}
Você pode achar o repositório git BlackArch principal aqui:
\href{https://github.com/BlackArch/blackarch}{https://github.com/BlackArch/blackarch}.
Temos também um secundário repositório aqui:
\href{https://github.com/BlackArch}{https://github.com/BlackArch}.

O repositório git principal, contém três diretórios importantes:

\begin{itemize}
\item docs - Documentação.
\item packages - Arquivo PKGBUILD.
\item scripts - Pequenos scripts.
\end{itemize}

\subsection{Scripts}
Aqui há uma referencia para o diretório de \verb|scripts/| scripts:

\begin{itemize}
\item baaur - Brevemente, irá subir os pacotes para o AUR.
\item babuild - Constrói os pacotes.
\item bachroot - Manipula o chroot para testes.
\item baclean - Limpa velhos arquivos .pkg.tar.xz do repositório.
\item baconflict - Irá substiuir o \verb|scripts/conflicts|.
\item bad-files - Acha arquivos ruins nos pacotes de construção.
\item balock - Obtém ou atualiza pacotes travados.
\item banotify - Notifica IRC sobre pushes de pacotes.
\item barelease - Atualiza pacotes no repositórios de pacotes.
\item baright - Imprime a informação de copyright do BlackArch.
\item basign - Assina pacotes.
\item basign-key - Assina as chaves.
\item blackman - Funciona de forma análoga ao pacman builds construindo pacotes do git(não confundir com o Blackman).
\item check-groups - Confere grupos.
\item checkpkgs - Confere erro em pacotes.
\item conflicts - Confere conflitos em pacotes.
\item dbmod - Modifica o banco de dados de pacotes.
\item depth-list - Cria uma lista de dependência.
\item deptree - Cria uma lista de dependência, somente dos pacotes do BlackArch.
\item get-blackarch-deps - Pega uma lista de dependências de pacotes do BlackArch.
\item get-official - Pega atualizações oficiais do BlackArch.
\item list-loose-packages - Lista os pacotes que estão ou não estão em um grupo que dependem de outros pacotes.
\item list-needed - Lista dependências perdidas.
\item list-removed - Lista os pacotes que estão no repositório mas não no git.
\item list-tools - Lista as ferramentas.
\item outdated - Olhar por pacotes desatualizados no repositório comparado ao git.
\item pkgmod - Modifica pacotes de construção.
\item pkgrel - Adiciona pkgrel em um pacote.
\item prep - Limpa arquivos PKGBUILD e procura erros.
\item sitesync - Sincroniza entre os pacotes locais e os pacotes no repositório.
\item size-hunt - Procura pacotes pesados/grandes.
\item source-backup - Realiza backup do codigo fonte dos arquivos.
\end{itemize}

\section{Contribuindo para o repositório}
Essa seção mostra como você pode contribuir com o projeto BlackArch Linux.
Nós aceitamos pull requests de todos os tamanhos, de pequenos consertos até novos pacotes.
\\Para mais ajuda, sugestões ou questões sinta-se livre para nos contatar.\\\\
Todo mundo que queira contribuir será bem vindo.Toda contribuição é bem vinda.

\subsection{Tutorial de requerimento}
Por favor leia este tutorial antes de contribuir:
\begin{itemize}
\item
\href{https://wiki.archlinux.org/index.php/Arch\_Packaging\_Standards)}{Arch
Packaging Standards}
\item \href{https://wiki.archlinux.org/index.php/Creating\_Packages}{Creating
Packages}
\item \href{https://wiki.archlinux.org/index.php/PKGBUILD}{PKGBUILD}
\item \href{https://wiki.archlinux.org/index.php/Makepkg}{Makepkg}
\end{itemize}

\subsection{Passos para contribuir}
Quando submeter suas mudanças para o projeto BlackArchLinux, siga estes passos:
\begin{enumerate}
\item Fork o repositório em
\url{https://github.com/BlackArch/blackarch}
\item Hackei os arquivos necessários, (e.g. PKGBUILD, .patch files, etc).
\item Commit as mudanças.
\item Push suas mudanças.
\item Nos peça para dar merge em suas mudanças, de preferencialmente um pull request.
\end{enumerate}

\subsection{Exemplo}
O exemplo a seguir demonstra como enviar um novo pacotes para o projeto BlackArch.
Nós usamos \href{https://wiki.archlinux.org/index.php/yaourt}{yaourt}
(you can use pacaur as well) para buscar arquivos PKGBUILD de
\textbf{nfsshell} para \href{https://aur.archlinux.org/}{AUR} e modificamos de acordo com a necessidade.

\subsubsection{Buscando PKGBUILD}
Busque o arquivo \textit{PKGBUILD} usando o yaourt ou pacaur:
\begin{lstlisting}
  user@blackarchlinux $ yaourt -G nfsshell
  ==> Download nfsshell sources
  x LICENSE
  x PKGBUILD
  x gcc.patch
  user@blackarchlinux $ cd nfsshell/
\end{lstlisting}

\subsubsection{Limpar o PKGBUILD}
Limpe o arquivo \textit{PKGBUILD} e salve:
\begin{lstlisting}
  user@blackarchlinux nfsshell $ ./blarckarch/scripts/prep PKGBUILD
  cleaning 'PKGBUILD'...
  expanding tabs...
  removing vim modeline...
  removing id comment...
  removing contributor and maintainer comments...
  squeezing extra blank lines...
  removing '|| return'...
  removing leading blank line...
  removing $pkgname...
  removing trailing whitespace...
\end{lstlisting}

\subsubsection{Modificações PKGBUILD}
Modifique o arquivo \textit{PKGBUILD}:
\begin{lstlisting}
  user@blackarchlinux nfsshell $ vi PKGBUILD
\end{lstlisting}

\subsubsection{Construa o pacote}
Construindo o pacote:
\begin{lstlisting}user@blackarchlinux nfsshell $ makepkg -sf
==> Making package: nfsshell 19980519-1 (Mon Dec  2 17:23:51 CET 2013)
==> Checking runtime dependencies...
==> Checking buildtime dependencies...
==> Retrieving sources...
-> Downloading nfsshell.tar.gz...
% Total    % Received % Xferd  Average Speed   Time    Time     Time
CurrentDload  Upload   Total   Spent    Left  Speed100 29213  100 29213    0
0  48150      0 --:--:-- --:--:-- --:--:-- 48206
-> Found gcc.patch
-> Found LICENSE
...
<lots of build process and compiler output here>
...
==> Leaving fakeroot environment.
==> Finished making: nfsshell 19980519-1 (Mon Dec  2 17:23:53 CET 2013)
\end{lstlisting}

\subsubsection{Instalando e testando os pacotes}
Intale e teste os pacotes:
\begin{lstlisting}
  user@blackarchlinux nfsshell $ pacman -U nfsshell-19980519-1-x86_64.pkg.tar.xz
  user@blackarchlinux nfsshell $ nfsshell # test it
\end{lstlisting}

\subsubsection{Adiciona, commit e push o pacote}
\begin{lstlisting}user@blackarchlinux nfsshell $ cd /blackarchlinux/packages
user@blackarchlinux ~/blackarchlinux/packages $ mv ~/nfsshell .
user@blackarchlinux ~/blackarchlinux/packages $ git commit -am nfsshell && git push
\end{lstlisting}

\subsubsection{Crie um pull request}
Create a pull request on \href{https://github.com/}{github.com}
\begin{lstlisting}
  firefox https://github.com/<contributor>/blackarchlinux
\end{lstlisting}

\subsubsection{Adicione remotamente}
Se melhor a se fazer se você trabalha remotamente é realizar o fork e o seu pull request para adicionar no repositório principal.
\begin{lstlisting}
  user@blackarchlinux ~/blackarchlinux $ git remote -v
  origin <the url of your fork> (fetch)
  origin <the url of your fork> (push)
  user@blackarchlinux ~/blackarchlinux $ git remote add upstream https://github.com/blackarch/blackarch
  user@blackarchlinux ~/blackarchlinux $ git remote -v
  origin <the url of your fork> (fetch)
  origin <the url of your fork> (push)
  upstream https://github.com/blackarch/blackarch (fetch)
  upstream https://github.com/blackarch/blackarch (push)
\end{lstlisting}

Por padrão o git deve realizar o push para a origem, mas tenha certeza que seu git
está configurado corretamente.Isto não deve causar problemas ao menos que você tenhas
alguns direitos de commit com não habilitar o push sem o mesmo.

Se você tem a habilidade de commitar, é mais fácil usar o
\textit{git@github.com:blackarch/blackarch.git} mas ai é com você.

\subsection{Requisitações}
\begin{enumerate}
\item Não adicione \textbf{Maintainer} ou \textbf{Contributor} comentarios nos arquivos
\textit{PKGBUILD}. Adicione o nome do mantedor ou contribuinte na seção de
AUTHORS do guia BlackArch.
\item Procure ser consistente, por favor siga o estilo dos outros arquivos \textit{PKGBUILD}
no repositório e use dois espaços para identação.
\end{enumerate}

\subsection{Dicas gerais}
\href{http://wiki.archlinux.org/index.php/Namcap}{namcap} para conferir os pacotes com erros.

%------------------%
%  Chapter 4       %
%------------------%

\chapter{Guia das ferramentas}
Em breve...

\section{Em breve}
Em breve...

%%% APPENDIX %%%
\appendix
\include{latex/appendix-pt-br}

\end{document}

%%% EOF %%%
