%%%%%%%%%%%%%%%%%%%%%%%%%%%%%%%%%%%%%%%%%%%%%%%%%%%%%%%%%%%%%%%%%%%%%%%%%%%%%%%%
%                                                                              %
% BlackArch Linux Guide                                                        %
%                                                                              %
%%%%%%%%%%%%%%%%%%%%%%%%%%%%%%%%%%%%%%%%%%%%%%%%%%%%%%%%%%%%%%%%%%%%%%%%%%%%%%%%

\documentclass[a4paper, oneside, 11pt]{book}

%%% INCLUDES %%%
\renewcommand{\familydefault}{\sfdefault}

\usepackage{array}
\usepackage{color}
\usepackage{enumerate}
\usepackage{fancyhdr}
\usepackage{fancyvrb}
\usepackage{geometry}
\usepackage{graphicx}
\usepackage{html}
\usepackage{hyperref}
\usepackage{ifpdf}
\usepackage{listings}
\usepackage{pstricks}
\usepackage{supertabular}
\usepackage{tocloft}
\usepackage[utf8]{inputenc}
\usepackage[french]{babel}


%%% LAYOUT %%%
\setlength{\parindent}{0em}
\setlength{\parskip}{1.5ex plus0.5ex minus0.5ex}
\geometry{left=2.5cm, textwidth=16cm, top=3cm, textheight=25cm, bottom=3cm}
\widowpenalty=2000
\clubpenalty=1000
\frenchspacing
\sloppy
\pagecolor[HTML]{FFFFFF}
\color[HTML]{333333}
\setcounter{tocdepth}{10}
\setcounter{secnumdepth}{10}

\hypersetup{
  pdftitle={BlackArch Linux, The BlackArch Linux Guide},
  pdfsubject={BlackArch Linux, The BlackArch Linux Guide},
  pdfauthor={BlackArch Linux, BlackArch Linux},
  pdfkeywords={BlackArch Linux, Penetration Testing, Security, Hacking, Linux},
  pdfcenterwindow=true,
  colorlinks=true,
  breaklinks=true,
  linkcolor=blue,
  menucolor=blue,
  urlcolor=blue
}

% syntax highlighting
% all options should be set here document wide
\lstset{
backgroundcolor=\color[HTML]{EEEEEE},
frame=single,
basicstyle=\footnotesize\ttfamily,
float,
deletekeywords={return},
otherkeywords={mkdir, curl, sudo, sha1sum, grep, cut, sort, wget, makepkg,
pacman, blackman, chmod},
keywordstyle=\color{orange},
commentstyle=\color{blue},
stringstyle=\color{red},
language=bash,
showspaces=false,
showtabs=false,
tabsize=2
}

%%% HEADER / FOOTER %%%
\setlength{\headheight}{33pt}
\setlength{\headsep}{33pt}
\lhead{{\includegraphics[width=1cm,height=1cm]{images/logo.png}}}
\rhead{The BlackArch Linux Guide}

%%% CUSTOM MACROS %%%
% for html links
\ifpdf\else
\def\href#1#2{\htmladdnormallink{#2}{#1}}
\fi

%------------------%
%  TITLE PAGE      %
%------------------%
\begin{document}
\pagestyle{empty}
\begin{center}
\begin{figure}[htbp]
\centering
\vspace{0.5cm}
\includegraphics[width=8cm]{images/logo.png}
\label{fig:logo}
\end{figure}
\vspace{0.5cm}
\Huge{\textbf{The BlackArch Linux Guide}}\\
\vspace{1cm}
\Large{\color{blue}https://www.blackarch.org/}\\
\vspace{0.5cm}
\end{center}
\newpage
\tableofcontents
\newpage
\pagestyle{fancy}

%------------------%
%  Chapter 1       %
%------------------%

\chapter{Introduction}

\section{Vue d'ensemble}

Ce guide est divis\'{e} en quatre parties majeures, soient:
\begin{itemize}
\item Introduction - Offre une vue d'ensemble et quelques informations relatives au projet
\item Guide utilisateur - Tous les \'{e}l\'{e}ments n\'{e}cessaires \`{a} la bonne utilisation de BlackArch
\item Guide d\'{e}veloppeur - Comment bien d\'{e}buter afin de contribuer au projet BlackArch
\item Guide outils - Description d\'{e}taill\'{e}e des outils et exemples d'utilisation (En construction)
\end{itemize}

\section{Qu'est-ce que BlackArch Linux?}
\href{http://www.blackarch.org}{BlackArch Linux} est une distribution GNU/Linux bas\'{e}e sur Arch Linux destin\'{e}e aux professionnels et chercheurs en s\'{e}curit\'{e} informatique.

L'ensemble d'outils est distribu\'{e} tel un \href{https://wiki.archlinux.org/index.php/Unofficial\_User\_Repositories} {d\'{e}p\^{o}t non-officiel} d'Arch Linux, vous permettant d'installer les composantes de BlackArch sur une installation d'Arch Linux existante. Les paquets peuvent y \^{e}tre install\'{e}s individuellement ou par cat\'{e}gorie.

Le d\'{e}pot logiciel de Black Arch contient plus de 2600 outils et ce nombre augmente sans cesse. Chacun de ces outils est minutieusement test\'{e} avant d'\^{e}tre ajout\'{e} aux d\'{e}pots, afin d'assurer leur bon fonctionnement.

\section{Histoire de BlackArch Linux}
En construction...

\section{Plateformes support\'{e}es}
En construction...

\section{S'impliquer}

Il est possible de contacter l'\'{e}quipe Black Arch via:

Website: \url{https://www.blackarch.org/}

Mail: \href{mailto:team@blackarch.org}{team@blackarch.org}

IRC: \url{irc://irc.freenode.net/blackarch}

Twitter: \url{https://twitter.com/blackarchlinux}

Github: \url{https://github.com/Blackarch/}

%%%%%%%%%%%%%%%%%%%%%%%%%%%%%%%%%%%%%%%%%%%%%%%%%%%%%%%%%%%%%%%%%%%%%%%%%%%%%%%%

\chapter{Guide d'utilisation}

\section{Installation}
La section suivante illustre comment configurer le d\'{e}p\^{o}t de paquets de Black Arch ainsi que la proc\'{e}dure d'installation de paquets. BlackArch supporte l'installation de fichiers binaires pr\'{e}compil\'{e}s ainsi que l'installation \`{a} l'aide du code source original des paquets.

BlackArch est compatible avec une installation de base d'Arch Linux. Il fait office de d\'{e}p\^{o}t non-officiel. Si vous d\'{e}sirez utiliser une image ISO \`{a} la place, visitez la section \href{http://www.blackarch.org/download.html#iso}{Live ISO}.

\subsection{Installer le d\'{e}p\^{o}t}
Ex\'{e}cuter \href{https://blackarch.org/strap.sh}{strap.sh} en tant que root et suivez les instructions suivantes:

\begin{lstlisting}
   curl -O https://blackarch.org/strap.sh
   sha1sum strap.sh # doit etre egal a 5ea40d49ecd14c2e024deecf90605426db97ea0c
   sudo chmod +x strap.sh
   sudo ./strap.sh
\end{lstlisting}

Mettez ensuite \`{a} jour la liste des paquets en ex\'{e}cutant:
\begin{lstlisting}
  sudo pacman -Syyu
\end{lstlisting}

\subsection{Installation de paquets}
Vous pouvez d\'{e}sormais installer des outils directement du d\'{e}p\^{o}t de BlackArch
\begin{enumerate}
\item Afin d'obtenir la liste de tous les outils disponibles ex\'{e}cutez
\begin{lstlisting}
  pacman -Sgg | grep blackarch | cut -d' ' -f2 | sort -u
\end{lstlisting}

\item Afin d'installer tous les outils, ex\'{e}cutez
\begin{lstlisting}
  pacman -S blackarch
\end{lstlisting}

\item Afin d'installer une cat\'{e}gorie compl\`{e}te d'outils, ex\'{e}cutez
\begin{lstlisting}
  pacman -S blackarch-<category>
\end{lstlisting}

\item Afin d'obtenir la liste des cat\'{e}gories de paquets disponibles, ex\'{e}cutez
\begin{lstlisting}
  pacman -Sg | grep blackarch
\end{lstlisting}

\end{enumerate}

\subsection{Installation des paquets depuis le code source}
Il existe une m\'{e}thode alternative \`{a} l'installation de paquets sous forme binaire. BlackArch vous offre la possibilit\'{e} d'obtenir le code source directement du d\'{e}p\^{o}t officiel du paquet d\'{e}sir\'{e}, et BlackArch le compilera pour vous. Vous trouverez les fichiers PKGBUILD sur \href{https://github.com/BlackArch/blackarch/tree/master/packages}{github}. Afin d'obtenir le d\'{e}p\^{o}t en entier, vous pourrez utiliser l'outil \href{https://github.com/BlackArch/blackman}{blackman}.

\begin{itemize}
\item En premier lieu,vous devez installer blackman. Si vous avez d\'{e}j\`{a} configur\'{e}
le d\'{e}p\^{o}t de paquets de BlackArch sur votre syst\`{e}me, tel qu'indiqu\'{e} au d\'{e}but de
ce document, vous vous pouvez installer blackman \`{a} l'aide de pacman en utilisant:
\begin{lstlisting}
  pacman -S blackman
\end{lstlisting}

\item Vous pouvez installer blackman depuis les sources:
\begin{lstlisting}
  mkdir blackman
  cd blackman
  wget https://raw.github.com/BlackArch/blackarch/master/packages/blackman/PKGBUILD
  # S'assurer de l'integrite du fichier PKGBUILD avec:
  makepkg -s
\end{lstlisting}

\item Vous pouvez \'{e}galement installer blackman depuis les d\'{e}p\^{o}ts AUR
\begin{lstlisting}
  <votre assistant AUR> -S blackman
\end{lstlisting}

\end{itemize}

\subsection{Utilisation de Blackman} Blackman est un outil simple \`{a} utiliser. Cependant, les param\`{e}tres peuvent \^{e}tre diff\'{e}rent de ce qu'offrent des gestionnaires de paquets tel que pacman. Les exemples ci-dessous illustrent les cas d'utilisation de base.

\begin{itemize}
\item T\'{e}l\'{e}charger, compiler et installer un paquet:
\begin{lstlisting}
  sudo blackman -i paquet
\end{lstlisting}

\item T\'{e}l\'{e}charger, compiler et installer un groupe entier:
\begin{lstlisting}
  sudo blackman -g groupe
\end{lstlisting}

\item T\'{e}l\'{e}charger, compiler et installer tous les outils BlackArch:
\begin{lstlisting}
  sudo blackman -a
\end{lstlisting}

\item Lister les cat\'{e}gories disponibles:
\begin{lstlisting}
  blackman -l
\end{lstlisting}

\item Lister les outils d'une cat\'{e}gorie:
\begin{lstlisting}
  blackman -p category
\end{lstlisting}

\end{itemize}

\subsection{Installation d'un LiveCD, netinstall, ou d'Arch Linux}
Vous pouvez installer BlackArch depuis l'une des images ISO fournies (live ou netinstall.
Voir \url{https://www.blackarch.org/download.html#iso}. Voici les \'{e}tapes \`{a} r\'{e}aliser une fois le syst\`{e}me d\'{e}marr\'{e} depuis une ISO.

\begin{itemize}
\item Installer le paquet blackarch-installer:
\begin{lstlisting}
  sudo pacman -S blackarch-installer
\end{lstlisting}

\item Ex\'{e}cuter:
\begin{lstlisting}
  sudo blackarch-install
\end{lstlisting}

\end{itemize}

%%%%%%%%%%%%%%%%%%%%%%%%%%%%%%%%%%%%%%%%%%%%%%%%%%%%%%%%%%%%%%%%%%%%%%%%%%%%%%%%


\chapter{Guide de d\'{e}veloppement}

\section{Le Arch Build System et les d\'{e}p\^{o}ts de paquets}

Les fichiers PKGBUILD sont de simples scripts permettant d'installer et de compiler et installer un programme sur plusieurs environnements diff\'{e}rents. Chaque PKGBUILD indique \`{a} la commande makepkg(1) comment cr\'{e}er un paquet valide qui pourra ensuite \^{e}tre install\'{e}. Les fichiers PKGBUILD sont \'{e}crits en bash.

Pour plus d'information, il est sugg\'{e}r\'{e} de lire ces pages:

\begin{itemize}
\item \href{http://wiki.archlinux.fr/Standard_paquetage}{Arch Wiki FR:
Standard paquetage}
\item \href{http://wiki.archlinux.fr/Makepkg}{Arch Wiki FR: Makepkg}
\item \href{http://wiki.archlinux.fr/PKGBUILD}{Arch Wiki FR: PKGBUILD}
\end{itemize}

\section{Le standard de paquetage de BlackArch}

Par souci de simplicit\'{e}, nos PKGBUILDs sont similaires \`{a} ceux du
Arch User Repository, \`{a} quelques diff\'{e}rences pr\`{e}s. Tous les paquets doivent
\^{e}tre membre du groupe \textit{blackarch}. Il est possible qu'un paquet soit
membre de plusieurs groupes.

\subsection{Groupes}

Afin de permettre aux utilisateurs d'installer certaines cat\'{e}gories de paquets
, ceux-ci ont \'{e}t\'{e}s r\'{e}pertori\'{e}s en plusieurs groupes. Ces derniers
permettent aux utilisateurs d'ex\'{e}cuter la commande
\textit{pacman -S \textless nom\_du\_groupe \textgreater } afin d'installer tous les paquets compris dans ce groupe.

\subsubsection{blackarch}

En th\'{e}orie, le groupe \textit{blackarch} doit contenir tous les paquets qu'offre BlackArch. Tous les paquets doivent \^{e}tre membre de ce groupe, comme expliqu\'{e} pr\'{e}c\'{e}demment.

\subsubsection{blackarch-anti-forensic}

Groupe contenant les paquets utilis\'{e}s pour contrer les tentatives
d'investigation informatique de syst\`{e}mes. Vous y trouverez des paquets offrant
des solutions de chiffrement de donn\'{e}es, st\'{e}ganographie, modification d'attributs de fichiers, et tout autre type de logiciels permettant la modification d'un syst\`{e}me afin de masquer de l'information.

Exemples: luks, TrueCrypt, Timestomp, dd, ropeadope, secure-delete

\subsubsection{blackarch-automation}

Groupe contenant les paquets utilis\'{e}s pour l'automation de plusieurs t\^{a}ches.
Puisque les paquets pr\'{e}sents dans ce groupe varient \'{e}norm\'{e}ment en
fonctionnalit\'{e}s, il vous est sugg\'{e}r\'{e} d'y consulter directement les paquets
afin de vous informer sur leurs capacit\'{e}s.

Exemples: blueranger, tiger, wiffy

\subsubsection{blackarch-backdoor}

Groupe contenant les paquets reli\'{e}s \`{a} l'exploitation ou \`{a} l'ouverture de
backdoors sur des syst\`{e}mes vuln\'{e}rables.

Exemples: backdoor-factory, rrs, weevely

\subsubsection{blackarch-binary}

Groupe contenant les paquets reli\'{e}s aux fichiers binaires. Vous y trouverez entre autres des paquets permettant l'extraction d'archives d'un firmware, des logiciels de
r\'{e}tro-ing\'{e}nierie, logiciels aidant \`{a} l'exploitation de fichiers binaires.

Exemples: binwally, packerid, hex2bin, binwalk

\subsubsection{blackarch-bluetooth}

Groupe contenant les paquets relatifs \`{a} l'exploitation du standard Bluetooth
(802.15.1).

Exemples: ubertooth, tbear, redfang

\subsubsection{blackarch-code-audit}

Groupe contenant les paquets \`{a} fin d'audit de code source. Ces paquets analysent statiquement le code source d'un projet afin d'y d\'{e}celer des vuln\'{e}rabilit\'{e}s.

Exemples: flawfinder, pscan

\subsubsection{blackarch-cracker}

Groupe contenant les paquets utilis\'{e}s afin de tester la robustesse de multiples fonctions cryptographiques.

Exemples: hashcat, john, crunch

\subsubsection{blackarch-crypto}

Groupe contenant tous types de paquets relatifs \`{a} la cryptographie.

Exemples: ciphertest, xortool, sbd

\subsubsection{blackarch-database}

Groupe contenant les paquets relatifs \`{a} l'exploitation de bases de donn\'{e}es.

Exemples: metacoretex, blindsql

\subsubsection{blackarch-debugger}

Groupe contenant les paquets permettant \`{a} l'usager d’investiguer le
fonctionnement interne d'un programme lors de son utilisation.

Exemples: radare2, shellnoob

\subsubsection{blackarch-decompiler}

Groupe contenant les paquets qui tentent d'effectuer de la r\'{e}tro-ing\'{e}nierie sur un programme compil\'{e}.

Exemples: flasm, jd-gui

\subsubsection{blackarch-defensive}

Groupe contenant les paquets utilis\'{e}s afin de prot\'{e}ger un syst\`{e}me des logiciels (malwares) et utilisateurs malveillants.

Exemples: arpon, chkrootkit, sniffjoke

\subsubsection{blackarch-disassembler}

Ce groupe est tr\`{e}s semblable \`{a} \textit{blackarch-decompiler} dans le sens
que ces deux groupes tentent d'effectuer une r\'{e}tro-ing\'{e}nierie g\'{e}n\'{e}rique de
fichiers ex\'{e}cutables, mais les paquets compris dans
\textit{blackarch-disassembler} en extrairont le code assembleur et non un code source sous forme de pseudocode par exemple.

Exemples: inguma, radare2

\subsubsection{blackarch-dos}

Groupe contenant les paquets relatifs aux attaques de type \textit{d\'{e}ni de service}.

Exemples: 42zip, nkiller2

\subsubsection{blackarch-drone}

Groupe contenant les paquets aidant \`{a} l'entretien et l'utilisation de drones.

Exemples: meshdeck, skyjack

\subsubsection{blackarch-exploitation}

Groupe contenant les paquets utilis\'{e}s lors de l'exploitation de multiples
programmes et services.

Exemples: armitage, metasploit, zarp

\subsubsection{blackarch-fingerprint}

Groupe contenant les paquets utilis\'{e}s \`{a} des fins de reconnaissance et prise d'empreinte.

Exemples: dns-map, p0f, httprint

\subsubsection{blackarch-firmware}

Groupe contenant les paquets utilis\'{e}s lors de l'exploitation de firmware.

Exemples: None yet, amend asap.

\subsubsection{blackarch-forensic}

Groupe contenant les paquets utilis\'{e}s lors d'investigation informatique portant sur le recouvrement de donn\'{e}es potentiellement effac\'{e}s.

Exemples: aesfix, nfex, wyd

\subsubsection{blackarch-fuzzer}

Groupe contenant les paquets permettant de fuzzer des logiciels.

Exemples: msf, mdk3, wfuzz

\subsubsection{blackarch-hardware}

Groupe contenant les paquets relatifs \`{a} l'exploitation de mat\'{e}riel physique.

Exemples: arduino, smali

\subsubsection{blackarch-honeypot}

Groupe contenant les paquets agissant comme des \textit{honeypots}, c'est-\`{a}-dire des services s\'{e}curitaires, mais se faisant passer pour vuln\'{e}rables afin d'observer son comportement.

Exemples: artillery, bluepot, wifi-honey

\subsubsection{blackarch-keylogger}

Groupe contenant les paquets utilis\'{e}s lors d'installation de \textit{keylogging}
, soit l'enregistrement de touches clavier sur un syst\`{e}me qui pourrait \^{e}tre
utilis\'{e} par un utilisateur tiers afin d'en soutirer des informations confidentielles.

Exemples: klogger, logkeys, xspy

\subsubsection{blackarch-malware}

Groupe contenant tous types de paquets ayant un lien au \textit{malware}, que ce soit \`{a} l'utilisation de ceux-ci ou \`{a} leur d\'{e}tection.

Exemples: malwaredetect, peepdf, yara

\subsubsection{blackarch-misc}

Groupe contenant les paquets ne semblant pas prendre part dans aucunes autres
cat\'{e}gories.

Exemples: oh-my-zsh-git, winexe, stompy

\subsubsection{blackarch-mobile}

Groupe contenant les paquets relatifs aux activit\'{e}s d'entretien et d'audits
d'applications, de d\'{e}veloppement d'application, et de mat\'{e}riel mobile.

Exemples: android-sdk-platform-tools, android-udev-rules

\subsubsection{blackarch-networking}

Groupe contenant les paquets relatifs au r\'{e}seau.

Exemples: Anything pretty much

\subsubsection{blackarch-nfc}

Groupe contenant les paquets utilisant la technologie \textit{NFC} (near-field
communications).

Exemples: nfcutils

\subsubsection{blackarch-packer}

Groupe contenant les paquets relatifs aux \textit{packers}, soit des logiciels qui cr\'{e}ent des applications embarquant un programme malveillant.

Exemples: packerid

\subsubsection{blackarch-proxy}

Groupe contenant les paquets agissant tel un proxy, redirigeant le trafic internet afin de l'analyser ou de le modifier.

Exemples: burpsuite, ratproxy, sslnuke

\subsubsection{blackarch-recon}

Groupe contenant les paquets servant \`{a} la recherche de services exploitable.

Exemples: canri, dnsrecon, netmask

\subsubsection{blackarch-reversing}

Groupe contenant les paquets de d\'{e}compilation, de d\'{e}sassemblage, et tout autres types de programmes relatifs \`{a} la r\'{e}tro-ing\'{e}nierie.

Exemples: capstone, radare2, zerowine

\subsubsection{blackarch-scanner}

Groupe contenant les paquets utilis\'{e}s afin de scanner multiples syst\`{e}mes permettant d'y d\'{e}celer des vuln\'{e}rabilit\'{e}s actives.

Exemples: scanssh, tiger, zmap

\subsubsection{blackarch-sniffer}

Groupes contenant les paquets aidant \`{a} l'analyse de trafic r\'{e}seau.

Exemples: hexinject, pytactle, xspy

\subsubsection{blackarch-social}

Groupe contenant les paquets relatifs aux attaques sur les sites de r\'{e}seaux
sociaux.

Exemples: jigsaw, websploit

\subsubsection{blackarch-spoof}

Groupe contenant les paquets qui tentent de dissimuler un attaquant, de tel que
la victime ne puisse le percevoir comme un acteur malicieux.

Exemples: arpoison, lans, netcommander

\subsubsection{blackarch-threat-model}

Groupe contenant les paquets utilis\'{e}s pour enregistrer et cr\'{e}er des rapports sur
plusieurs mod\`{e}les de menaces pouvant se produire lors d'un sc\'{e}nario sp\'{e}cifique.

Exemples: magictree

\subsubsection{blackarch-tunnel}

Groupe contenant les paquets utilis\'{e}s afin d'encapsuler le contenu r\'{e}seau via un
tiers parti, afin de dissimuler la provenance initiale de donn\'{e}es.

Exemples: ctunnel, iodine, ptunnel

\subsubsection{blackarch-unpacker}

Groupe contenant les paquets utilis\'{e}s lors de l'extraction de logiciel
malveillants de logiciels \textit{pack\'{e}s} ou obfusqu\'{e}s.

Exemples: js-beautify

\subsubsection{blackarch-voip}

Groupe contenant les paquets permettant de mener des op\'{e}rations sur les protocoles et programmes VoIP.

Exemples: iaxflood, rtp-flood, teardown

\subsubsection{blackarch-webapp}

Groupe contenant les paquets permettant de tester les applications web.

Exemples: metoscan, whatweb, zaproxy

\subsubsection{blackarch-windows}

Groupe contenant les paquets Windows s'ex\'{e}cutant via Wine.

Exemples: 3proxy-win32, pwdump, winexe

\subsubsection{blackarch-wireless}

Groupe contenant les paquets utilis\'{e}s lors d'audits de s\'{e}curit\'{e} sur une
infrastructure sans-fil.

Exemples: airpwn, mdk3, wiffy

\section{Structure du d\'{e}p\^{o}t}

Vous pouvez trouver le d\'{e}pot principal de BlackArch \`{a}
\href{https://github.com/BlackArch/blackarch}
{https://github.com/BlackArch/blackarch}. Il existe aussi plusieurs d\'{e}pots de
code \`{a} \href{https://github.com/BlackArch}{https://github.com/BlackArch}.

\`{a} l'int\'{e}rieur du d\'{e}p\^{o}t principal, vous trouverez les trois dossiers principaux,
tels que :

\begin{itemize}
\item docs - Contient la documentation du projet
\item packages - Contient les fichiers PKGBUILD
\item scripts - Contient de multiples scripts d'automation
\end{itemize}

\subsection{Scripts}

Voici une r\'{e}f\'{e}rence des scripts que vous trouverez dans le dossier
\verb|scripts/| :

\begin{itemize}
\item baaur - Bient\^{o}t, ce script enverra nos paquets au d\'{e}p\^{o}t AUR.
\item babuild - Construit un paquet.
\item bachroot - Pr\'{e}pare un chroot afin d'effectuer des tests.
\item baclean - N\'{e}ttoie les anciens fichiers .pkg.tar.xz du d\'{e}p\^{o}t.
\item baconflict - Bient\^{o}t, ce script remplacera \verb|scripts/conflicts|.
\item bad-files - Trouve des fichiers erron\'{e}s dans les paquets construits.
\item balock - Cr\'{e}e ou termine le lock sur le d\'{e}p\^{o}t de paquets.
\item banotify - Indique IRC des push de paquets.
\item barelease - Envoie les paquets au d\'{e}p\^{o}t.
\item baright - Affiche les informations de copyright de BlackArch.
\item basign - Signe les paquets.
\item basign-key - Signe une cl\'{e}.
\item blackman - Op\`{e}re sensiblement comme pacman mais construit les paquets \`{a}
partir de git. (\`{a} ne pas confondre avec le programme Blackman, \'{e}crit par nzr)
\item check-groups - Valide les groupes.
\item checkpkgs - Valide les paquets.
\item conflicts - Valide les paquets pour des conflits.
\item dbmod - Modifie la base de donn\'{e}e d'un paquet.
\item depth-list - Cr\'{e}e une liste de d\'{e}pendances ordonn\'{e}e par profondeur.
\item deptree - Cr\'{e}e une arbre de d\'{e}pendance, ne listant que les paquets obtenus
via BlackArch.
\item get-blackarch-deps - Affiche une liste des d\'{e}pendances de paquets obtenus
via BlackArch pour un paquet donn\'{e}.
\item get-official - Obtient les paquets officiels.
\item list-loose-packages - Liste les paquets qui ne se trouvent dans aucun
groupe et ne sont pas des d\'{e}pendances pour d'autres paquets.
\item list-needed - Liste les d\'{e}pendances manquantes.
\item list-removed - Liste les paquets qui sont dans le d\'{e}p\^{o}t mais pas dans le
r\'{e}pertoire git.
\item list-tools - Liste les outils.
\item outdated - Cherche les paquets qui ne sont plus \`{a} jour lorsque compar\'{e}s
avec le r\'{e}pertoire git.
\item pkgmod - Modifie un paquet de construction.
\item pkgrel - Incr\'{e}mente le pkgrel d'un paquet.
\item prep - Nettoie un fichier PKGBUILD de ses erreurs de style et autre.
\item sitesync - Synchronise entre une copie locale et sa version distante.
\item size-hunt - Recherche de gros paquets.
\item source-backup - Sauvegarde les fichiers sources d'un paquet.
\end{itemize}

\section{Contribuer au d\'{e}p\^{o}t}
Cette section indique comment contribuer au projet BlackArch Linux. Nous
acceptons les pull requests de toutes tailles, que ce soit de petites
modifications ou un ajout de paquet.\\Pour recevoir de l'aide, n'h\'{e}sitez pas \`{a}
nous contacter.
\\\\
Tout le monde est bienvenu \`{a} contribuer. Toute contribution est appr\'{e}ci\'{e}e.

\subsection{Tutoriels requis.}
Veuillez lire les documents suivants avant de contribuer:

\begin{itemize}
\item \href{http://wiki.archlinux.fr/Standard_paquetage}{Arch Wiki FR:
Standard paquetage}
\item \href{http://wiki.archlinux.fr/Makepkg}{Arch Wiki FR: Makepkg}
\item \href{http://wiki.archlinux.fr/PKGBUILD}{Arch Wiki FR: PKGBUILD}
\end{itemize}

\subsection{\'{e}tapes \`{a} suivre afin de contribuer}
Afin de contribuer au projet BlackArch Linux, suivez ces \'{e}tapes:
\begin{enumerate}
\item Forkez le d\'{e}p\^{o}t sur github \`{a} partir de
\url{https://github.com/BlackArchLinux/blackarchlinux}.
\item Effectuez vos modifications sur les fichiers n\'{e}c\'{e}ssaires (e.g. PKGBUILD,
.patch files, etc).
\item Commitez vos changements.
\item Poussez vos changements.
\item Cr\'{e}ez un pull request vers le d\'{e}p\^{o}t.
\end{enumerate}

\subsection{Exemple}
L'exemple suivant d\'{e}montre la soumissions d'un nouveau paquets au projet
BlackArch. Nous utilisons \href{https://wiki.archlinux.org/index.php/yaourt}{yaourt}
(vous pouvez aussi utiliser pacaur) afin d'obtenir le PKGBUILD existant de
\textbf{nfsshell}, et nous pouvons ensuite l'\'{e}diter, au besoin.

\subsubsection{Obtenir le PKGBUILD}
En utilisant yaourt ou pacaur, obtenir le fichier \textit{PKGBUILD}:
\begin{lstlisting}
  user@blackarchlinux $ yaourt -G nfsshell
  ==> Download nfsshell sources
  x LICENSE
  x PKGBUILD
  x gcc.patch
  user@blackarchlinux $ cd nfsshell/
\end{lstlisting}

\subsubsection{Nettoiement du PKGBUILD}
Nettoyer le fichier \textit{PKGBUILD} \`{a} l'aide de \textit{prep}, afin de sauver
du temps:
\begin{lstlisting}
  user@blackarchlinux nfsshell $ ./blarckarch/scripts/prep PKGBUILD
  cleaning 'PKGBUILD'...
  expanding tabs...
  removing vim modeline...
  removing id comment...
  removing contributor and maintainer comments...
  squeezing extra blank lines...
  removing '|| return'...
  removing leading blank line...
  removing $pkgname...
  removing trailing whitespace...
\end{lstlisting}

\subsubsection{Ajustement du PKGBUILD}
Ajustez le fichier \textit{PKGBUILD}:
\begin{lstlisting}
  user@blackarchlinux nfsshell $ vi PKGBUILD
\end{lstlisting}

\subsubsection{Construire le paquet}
Construire le paquet:
\begin{lstlisting}user@blackarchlinux nfsshell $ makepkg -sf
==> Making package: nfsshell 19980519-1 (Mon Dec  2 17:23:51 CET 2013)
==> Checking runtime dependencies...
==> Checking buildtime dependencies...
==> Retrieving sources...
-> Downloading nfsshell.tar.gz...
% Total    % Received % Xferd  Average Speed   Time    Time     Time
CurrentDload  Upload   Total   Spent    Left  Speed100 29213  100 29213    0
0  48150      0 --:--:-- --:--:-- --:--:-- 48206
-> Found gcc.patch
-> Found LICENSE
...
<lots of build process and compiler output here>
...
==> Leaving fakeroot environment.
==> Finished making: nfsshell 19980519-1 (Mon Dec  2 17:23:53 CET 2013)
\end{lstlisting}

\subsubsection{Installation et test du paquet}
Installation et test du paquet:
\begin{lstlisting}
  user@blackarchlinux nfsshell $ pacman -U nfsshell-19980519-1-x86_64.pkg.tar.xz
  user@blackarchlinux nfsshell $ nfsshell # test it
\end{lstlisting}

\subsubsection{Ajout, commit et push du paquet}
Ajoute, commit et push du paquet:
\begin{lstlisting}user@blackarchlinux nfsshell $ cd /blackarchlinux/packages
user@blackarchlinux ~/blackarchlinux/packages $ mv ~/nfsshell .
user@blackarchlinux ~/blackarchlinux/packages $ git commit -am nfsshell && git push
\end{lstlisting}

\subsubsection{Cr\'{e}ation d'un pull request}
Cr\'{e}ez un pull request de votre fork vers
 \href{https://github.com/BlackArchLinux/blackarchlinux}{le projet officiel}:
\begin{lstlisting}
  firefox https://github.com/<contributor>/blackarchlinux
\end{lstlisting}

\subsubsection{Ajout d'un remote pour l'upstream}
Une fois votre modification faite, il ajouter le projet officiel comme remote, afin d'ajouter les modifications au projet \`{a} votre fork.
Cela vous permettra de synchroniser votre fork avec BlackArch.
\begin{lstlisting}
  user@blackarchlinux ~/blackarchlinux $ git remote -v
  origin <the url of your fork> (fetch)
  origin <the url of your fork> (push)
  user@blackarchlinux ~/blackarchlinux $ git remote add upstream https://github.com/blackarch/blackarch
  user@blackarchlinux ~/blackarchlinux $ git remote -v
  origin <the url of your fork> (fetch)
  origin <the url of your fork> (push)
  upstream https://github.com/blackarch/blackarch (fetch)
  upstream https://github.com/blackarch/blackarch (push)
\end{lstlisting}

Par d\'{e}faut, git devrait push directement \`{a} l'origine, mais assurez-vous que
votre configuration git est correcte. Cela ne sera pas un probl\`{e}me \`{a} moins
d'avoir les droits de commit puisque vous ne pourrez pas push upstream sans
cela.

Si vous avez les droits de commit, vous aurez possiblement plus de succ\`{e}s en
utilisant \textit{git@github.com:blackarch/blackarch.git}. A vous de voir.

\subsection{Requ\^{e}tes}
\begin{enumerate}
\item N'ajoutez pas de commentaire \textbf{Maintainer} ou \textbf{Contributor}
au fichiers \textit{PKGBUILD}. Ajoutez ces commentaires \`{a} la section
\textbf{AUTHORS} du guide BlackArch.
\item Afin de conserver la consistance du projet, veuillez suivre les r\`{e}gles
g\'{e}n\'{e}rales de style des autres fichiers \textit{PKGBUILD} en utilisant un
indentation de deux espaces.
\end{enumerate}

\subsection{Conseils g\'{e}n\'{e}raux}
\href{http://wiki.archlinux.org/index.php/Namcap}{namcap} peut trouver des
erreurs dans vos paquets. Il est recommand\'{e} de l'utiliser avant de soumettre vos
modifications.

\chapter{Guide outils}
En construction...

%%% APPENDIX %%%
\appendix
\include{latex/appendix-fr}

\end{document}

%%% EOF %%%
