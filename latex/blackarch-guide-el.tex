%%%%%%%%%%%%%%%%%%%%%%%%%%%%%%%%%%%%%%%%%%%%%%%%%%%%%%%%%%%%%%%%%%%%%%%%%%%%%%%%
%                                                                              %
% BlackArch Linux Guide                                                        %
%                                                                              %
%%%%%%%%%%%%%%%%%%%%%%%%%%%%%%%%%%%%%%%%%%%%%%%%%%%%%%%%%%%%%%%%%%%%%%%%%%%%%%%%

\documentclass[a4paper, oneside, 11pt]{book}

%%% INCLUDES %%%
\renewcommand{\familydefault}{\sfdefault}

\usepackage{array}
\usepackage{color}
\usepackage{enumerate}
\usepackage{fancyhdr}
\usepackage{fancyvrb}
\usepackage{geometry}
\usepackage{graphicx}
\usepackage{html}
\usepackage{hyperref}
\usepackage{ifpdf}
\usepackage{listings}
\usepackage{pstricks}
\usepackage{supertabular}
\usepackage{tocloft}
\usepackage[utf8]{inputenc}

%%% LAYOUT %%%
\setlength{\parindent}{0em}
\setlength{\parskip}{1.5ex plus0.5ex minus0.5ex}
\geometry{left=2.5cm, textwidth=16cm, top=3cm, textheight=25cm, bottom=3cm}
\widowpenalty=2000
\clubpenalty=1000
\frenchspacing
\sloppy
\pagecolor[HTML]{FFFFFF}
\color[HTML]{333333}
\setcounter{tocdepth}{10}
\setcounter{secnumdepth}{10}

\hypersetup{
  pdftitle={BlackArch Linux, The BlackArch Linux Guide},
  pdfsubject={BlackArch Linux, The BlackArch Linux Guide},
  pdfauthor={BlackArch Linux, BlackArch Linux},
  pdfkeywords={BlackArch Linux, Penetration Testing, Security, Hacking, Linux},
  pdfcenterwindow=true,
  colorlinks=true,
  breaklinks=true,
  linkcolor=blue,
  menucolor=blue,
  urlcolor=blue
}

% syntax highlighting
% all options should be set here document wide
\lstset{
backgroundcolor=\color[HTML]{EEEEEE},
frame=single,
basicstyle=\footnotesize\ttfamily,
float,
deletekeywords={return},
otherkeywords={mkdir, curl, sudo, sha1sum, grep, cut, sort, wget, makepkg,
pacman, blackman, chmod},
keywordstyle=\color{orange},
commentstyle=\color{blue},
stringstyle=\color{red},
language=bash,
showspaces=false,
showtabs=false,
tabsize=2
}

%%% HEADER / FOOTER %%%
\setlength{\headheight}{33pt}
\setlength{\headsep}{33pt}
\lhead{{\includegraphics[width=1cm,height=1cm]{images/logo.png}}}
\rhead{The BlackArch Linux Guide}

%%% CUSTOM MACROS %%%
% for html links
\ifpdf\else
\def\href#1#2{\htmladdnormallink{#2}{#1}}
\fi

%------------------%
%  TITLE PAGE      %
%------------------%
\begin{document}
\pagestyle{empty}
\begin{center}
\begin{figure}[htbp]
\centering
\vspace{0.5cm}
\includegraphics[width=8cm]{images/logo.png}
\label{fig:logo}
\end{figure}
\vspace{0.5cm}
\Huge{\textbf{The BlackArch Linux Guide}}\\
\vspace{1cm}
\Large{\color{blue}https://www.blackarch.org/}\\
\vspace{0.5cm}
\end{center}
\newpage
\tableofcontents
\newpage
\pagestyle{fancy}

%------------------%
%  Chapter 1       %
%------------------%

\chapter{Introduction}

\section{Overview}
Ο οδηγός για BlackArch Linux είναι χωρισμένος σε διάφορα μέρη:
\begin{itemize}
\item Εισαγωγή - Παρέχει μια γενική επισκόπηση, εισαγωγή και επιπρόσθετες χρήσιμες πληροφορίες για το project
\item Οδηγός Χρήστη - Ότιδήποτε χρειάζεται να γνωρίζει ένα τυπικός χρήστης για να αξιοποιήσει τα BlackArch
\item Οδηγός Developer - Πώς να ξεκινήσετε να συνεισφέρετε στο BlackArch
\item Οδηγός Εργαλείων - Μια σε βάθος ανάλυση των εργαλείων μαζί με παραδείγματα (WIP)
\end{itemize}

\section{Τι είναι τα BlackArch Linux?}
Το BlackArch είναι μια πλήρης διανομή Linux για penetration testers και ερευνητές ασφαλείας.
Προέρχεται από \href{https://www.archlinux.org/}{ArchLinux} και οι χρήστες του μπορούν να εγκαταστήσουν τα επι μέρους εργαλεία του BlackArch 
πάνω σε αυτό είτε κάθε εργαλείο μόνο του είτε ομαδικά.

Η εργαλειοθήκη διανέμεται ως Arch Linux
\href{https://wiki.archlinux.org/index.php/Unofficial\_User\_Repositories}
{unofficial user repository} οπότε μπορείτε να εγκατασήσετε το BlackArch πάνω σε μια
υπάρχουσα διανομή Arch Linux. Τα πακέτα μπορούν να εγκατασταθούν το καθένα
μόνο του είτε μαζικά σε ομάδες.

Το συνεχώς αναπτυσσόμενο αποθετήριο περιλαμβάνει  \href{https://www.blackarch.org/tools.html}{2600} εργαλεία.
Όλα τα εργαλεία ελέγχονται πριν προστεθούν για να εξασφαλιστεί η ποιότητα του αποθετηρίου.
% should quickly describe the testing methods/code review procedures etc

\section{History of BlackArch Linux}
Coming soon...

\section{Supported platforms}
Coming soon...

\section{Get involved}
Μπορείτε να επικοινωνήσετε με την ομάδα του BlackArch ακολουθώντας τους παρακάτω τρόπους:

Σελίδα: \url{https://www.blackarch.org/}

Mail: \href{mailto:team@blackarch.org}{team@blackarch.org}

IRC: \url{irc://irc.freenode.net/blackarch}

Twitter: \url{https://twitter.com/blackarchlinux}

Github: \url{https://github.com/Blackarch/}

%------------------%
%  Chapter 2       %
%------------------%


\chapter{User Guide}

\section{Installation}
Ο ακόλουθος τομέας θα σας δείξει πως θα στήσετε το αποθετήριο του BlackArch
και πως θα εγκαταστήσετε πακέτα. Το BlackArch υποστηρίζει και τα 2, μπορείτε και να
εγκαταστήσετε πακέτα απο τα αποθετήρια χρησιμοποιώντας δυαδικά πακέτα αλλά και να
κάνετε compile και εγκατάσταση απο τα εκάστοτε sources.

Το BlackArch είναι συμβατό με όλες τις γνωστές Arch εγκαταστάσεις. Λειτουργεί ως ένα
ανεπίσημο αποθετήριο. Εαν θέλετε να χρησιμοποιείσετε ένα ISO μπορείτε να δείτε εδώ
\href{https://www.blackarch.org/downloads.html#iso}{Live ISO} section.

\subsection{Installing on top of ArchLinux}
Τρέξτε \href{https://blackarch.org/strap.sh}{strap.sh} ως root και ακολουθήστε τις
οδηγίες. Δείτε το ακόλουθο παράδειγμα.
\begin{lstlisting}
   curl -O https://blackarch.org/strap.sh
   sha1sum strap.sh # should match: 5ea40d49ecd14c2e024deecf90605426db97ea0c
   sudo chmod +x strap.sh
   sudo ./strap.sh
\end{lstlisting}

Τώρα κατεβάστε ένα φρέσκο αντίγραφο του master package list και συγχρονίστε τα πακέτα:
\begin{lstlisting}
  sudo pacman -Syyu
\end{lstlisting}


\subsection{Installing packages}
Μπορείτε πλέον να εγκαταστήσετε εργαλεία απο το αποθετήριο του blackarch.
\begin{enumerate}
\item To list all of the available tools, run
\begin{lstlisting}
  pacman -Sgg | grep blackarch | cut -d' ' -f2 | sort -u
\end{lstlisting}

\item Για να εγκαταστήσετε όλα τα εργαλεία, τρέξτε
\begin{lstlisting}
  pacman -S blackarch
\end{lstlisting}

\item Για να εγκαταστήσετε μια κατηγορία εργαλείων, τρέξτε
\begin{lstlisting}
  pacman -S blackarch-<category>
\end{lstlisting}

\item Για να δείτε τις κατηγορίες εργαλείων, τρέξτε
\begin{lstlisting}
  pacman -Sg | grep blackarch
\end{lstlisting}

\end{enumerate}

\subsection{Installing packages from source}
Μέρος μια εναλλακτικής εγκατάστασης είναι να κάνετε build τα πακέτα
BlackArch από source. Μπορείτε να βρείτε τα PKGBUILDs στο
\href{https://github.com/BlackArch/blackarch/tree/master/packages}{github}. 
Για να κάνετε build ολόκληρο το αποθετήριο μπορείτε να χρησιμοποιήσετε το
\href{https://github.com/BlackArch/blackman}{Blackman} εργαλείο.
\begin{itemize}
\item Πρώτα κάνετε την εγκατάσταση του Blackman. Εαν το αποθετήριο του BlackArch
βρίσκεται εγκατεστημένο στο μηχανημά σας, μπορείτε να βάλετε το Blackman:
\begin{lstlisting}
  pacman -S blackman
\end{lstlisting}

\item Μπορείτε να κάνετε build και εγκατάσταση το Blackman:
\begin{lstlisting}
  mkdir blackman
  cd blackman
  wget https://raw.github.com/BlackArch/blackarch/master/packages/blackman/PKGBUILD
  # Make sure the PKGBUILD has not been maliciously tampered with.
  makepkg -s
\end{lstlisting}

\item ή να εγκαταστήσετε το Blackman απο τα AUR:
\begin{lstlisting}
  <whatever AUR helper you use> -S blackman
\end{lstlisting}

\end{itemize}

\subsection{Basic Blackman usage} Το Blackman είναι πολύ απλό στη χρήση του, ωστόσο
οι σημαίες είναι λιγάκι διαφορετικές απο αυτό που θα περιμένατε απο κάτι σαν το pacman.
Η βασική χρήση παρουσιάζεται παρακάτω. 
\begin{itemize}
\item Download, compile and install packages:
\begin{lstlisting}
  sudo blackman -i package
\end{lstlisting}

\item Download, compile and install whole category:
\begin{lstlisting}
  sudo blackman -g group
\end{lstlisting}

\item Download, compile and install all of the BlackArch tools:
\begin{lstlisting}
  sudo blackman -a
\end{lstlisting}

\item To list the blackarch categories:
\begin{lstlisting}
  blackman -l
\end{lstlisting}

\item To list category tools:
\begin{lstlisting}
  blackman -p category
\end{lstlisting}

\end{itemize}

\subsection{Installing from full-, netinstall- ISO or ArchLinux}
Μπορείτε να εγκαταστήσετε τα BlackArch Linux από ένα απο τα full- ή netinstall-ISOs.\\Δείτε
\url{https://www.blackarch.org/download.html#iso}. Τα ακόλουθα βήματα είναι
απαραίτητα μετα απο την εκίννηση του ISO.
\begin{itemize}
\item Install blackarch-installer package:
\begin{lstlisting}
  sudo pacman -S blackarch-installer
\end{lstlisting}

\item Run
\begin{lstlisting}
  sudo blackarch-install
\end{lstlisting}

\end{itemize}

%------------------%
%  Chapter 3       %
%------------------%

\chapter{Developer Guide}

\section{Arch's Build System and Repositories}

Τα αρχεία PKGBUILD είναι build scripts. Το καθένα λέει makepkg(1) πως να δημιουργηθεί ένα
πακέτο. Τα αρχεία PKGBUILD είναι γραμμένα σε Bash.

Για περισσότερες πληροφορίες μπορείτε να δείτε τα παρακάτω:
\begin{itemize}
\item \href{https://wiki.archlinux.org/index.php/Creating_Packages}{Arch Wiki: Creating Packages}
\item \href{https://wiki.archlinux.org/index.php/Makepkg}{Arch Wiki: makepkg}
\item \href{https://wiki.archlinux.org/index.php/PKGBUILD}{Arch Wiki: PKGBUILD}
\item \href{https://wiki.archlinux.org/index.php/Arch_Packaging_Standards}{Arch Wiki: Arch Packaging Standards}
\end{itemize}

\section{Blackarch PKGBUILD standards}
Για χάριν απλότητας, τα PKGBUILDs μας, είναι παρόμοια με αυτά από τα AUR,
με μικρές αλλαγές που παρουσιάζονται παρακάτω. Κάθε πακέτο πρέπει
να ανήκει στο blackarch τουλάχιστον, και επίσης πολλά πακέτα μπορούν να
ανήκουν σε περισσότερες από μια ομάδες.

\subsection{Groups}
Για να μπορούν οι χρήστες να εγκαταστήσουν συγκεκριμένα πακέτα εύκολα και γρήγορα,
τα πακέτα έχουν χωριστεί σε ομάδες. Οι ομάδες επιτρέπουν στους χρήστες απλά να πάνε στο
"pacman -S <group name>" προκειμένου να πάρουν όποια πακέτα.

\subsubsection{blackarch}
Η ομάδα blackarch είναι η βασική ομάδα στην οπόια ανήκουν όλα τα πακέτα. Αυτό επιτρέπει
να μπορεί κάποιος να εγκαταστήσει όλα τα πακέτα εύκολα.

Τι θα έπρεπε να είναι εδώ: Τα πάντα.

\subsubsection{blackarch-anti-forensic}
Τα πακέτα που χρησιμοποιούνται για την καταπολέμηση της παρακολούθησης
(forensic activities), συμπεριλαμβανόμένων και κρυπτογραφίας, στεγανογραφίας και
οτιδήποτε αλλάζει δεδομένα αρχείων. Περιλαμβάνει γενικά εργαλεία που βοηθούν στις
αλλαγές ενός συστήματος για την απόκρυψη πληροφοριών.

Παραδείγματα: luks, TrueCrypt, Timestomp, dd, ropeadope, secure-delete

\subsubsection{blackarch-automation}
Εργαλεία για την αυτοματοποίηση διαδικασιών και ταχύτερη ολοκληρωση.

Παραδείγματα: blueranger, tiger, wiffy

\subsubsection{blackarch-backdoor}
Πακέτα που εκμεταλλεύονται ευπάθειες ή ανοίγουν κερκόπορτες σε ήδη
ευπαθή συστήματα.

Παραδείγματα: backdoor-factory, rrs, weevely

\subsubsection{blackarch-binary}
Πακέτα που επεξεργάζονται δυαδικά αρχεία με οποιοδήποτε τρόπο.

Παραδείγματα: binwally, packerid

\subsubsection{blackarch-bluetooth}
Πακέτα που εκμεταλλεύονται οτιδήποτε αφορά το Bluetooth (802.15.1).

Παραδείγματα: ubertooth, tbear, redfang

\subsubsection{blackarch-code-audit}
Πακέτα που ελέγχουν υπάρχοντα πηγαίο κώδικα για ανάλυση ευπαθειών.

Παραδείγματα: flawfinder, pscan

\subsubsection{blackarch-cracker}
Πακέτα για σπάσιμο κρυπτογραφικών μεθόδων.

Παράδειγματα: hashcat, john, crunch

\subsubsection{blackarch-crypto}
Πακέτα που ασχολούνται με την κρυπτογραφία, με την εξάιρεση το σπάσιμο.

Παραδείγματα: ciphertest, xortool, sbd

\subsubsection{blackarch-database}
Πακέτα που έχουν να κάνουμε με την εκμετάλλευση βάσεων σε
οποιοδήποτε επίπεδο.

Παραδείγματα: metacoretex, blindsql

\subsubsection{blackarch-debugger}
Πακέτα για προγράμματα με βασική λειτουργία την αποσφαλμάτωση.

Παραδείγματα: radare2, shellnoob

\subsubsection{blackarch-decompiler}
Πακέτα που χρησιμοποιούνται για προβολή του πηγαίου κώδικα
ήδη μεταγλωττισμένων(compiled) προγραμμάτων.

Παραδείγματα: flasm, jd-gui

\subsubsection{blackarch-defensive}
Πακέτα για την προστασία του χρήστη απο κακοβουλο λογισμικό
και επιθέσεις.

Παραδείγματα: arpon, chkrootkit, sniffjoke

\subsubsection{blackarch-disassembler}
Πακέτα που δίνουν ως αποτέλεσμα τον κώδικα assembly ενός
προγράμματος.

Παραδείγματα: inguma, radare2

\subsubsection{blackarch-dos}
Πακέτα για DoS (Denial of Service) επιθέσεις.

Παραδείγματα: 42zip, nkiller2

\subsubsection{blackarch-drone}
Πακέτα για την διαχείριση drones

Παραδείγματα: meshdeck, skyjack

\subsubsection{blackarch-exploitation}
Πακέτα που εκμεταλλεύονται ευπάθειες σε άλλα προγράμματα ή υπηρεσίες.

Παραδείγματα: armitage, metasploit, zarp

\subsubsection{blackarch-fingerprint}
Πακέτα που εκμεταλλεύονται εξοπλισμό για την αναγνώριση αποτυπωμάτων.

Παραδείγματα: dns-map, p0f, httprint

\subsubsection{blackarch-firmware}
Πακέτα που εκμεταλλεύονται ευπάθειες σε firmware

Παραδείγματα: None yet, amend asap.

\subsubsection{blackarch-forensic}
Πακέτα που χρησιμοποιούνται για την ανέυρεση πληροφοριών σε φυσικους δίσκους
ή ενσωματωμένες μνήμες.

Παραδείγματα: aesfix, nfex, wyd

\subsubsection{blackarch-fuzzer}
Πακέτα που χρησιμοποιούν την αρχή του fuzz testing principle,
πχ . "βάζουμε" τυχαία δεδομένα και βλέπουμε ποιό θα είναι το αποτέλεσμα.

Παραδείγματα: msf, mdk3, wfuzz

\subsubsection{blackarch-hardware}
Πακέτα που εκμευταλλέυονται με όποιο τρόπο το hardware.

Παραδείγματα: arduino, smali

\subsubsection{blackarch-honeypot}
Πακέτα που δρουν ως "honeypots", πχ. προγράμματα που φαίνονται
να είναι ευάλωτα σε υπηρεσίες και προσελκύουν επίδοξους hackers.

Παραδείγματα: artillery, bluepot, wifi-honey

\subsubsection{blackarch-keylogger}
Πακέτα που καταγράφουν και κατακρατούν πακέτα σε ένα άλλο σύστημα

Παράδειγματα: None yet, amend asap.

\subsubsection{blackarch-malware}
Πακέτα που είτε αναγνωρίζονται ως κακόβουλα είτε
δρουν για την αναγνώριση κακόβουλου λογισμικού.

Παραδείγματα: malwaredetect, peepdf, yara

\subsubsection{blackarch-misc}
Πακέτα που δεν ταιριάζουν σε άλλες κατηγορίες.

Παραδείγματα: oh-my-zsh-git, winexe, stompy

\subsubsection{blackarch-mobile}
Πακέτα που εκμεταλλεύονται πλατφόρμες κινητών.

Παραδείγματα: android-sdk-platform-tools, android-udev-rules

\subsubsection{blackarch-networking}
Πακέτα που περιλαμβάνουν IP networking.

Παραδείγματα: aprtools, dnsdiag, impacket

\subsubsection{blackarch-nfc}
Πακέτα που χρησιμοποιούν nfc (near-field communications).

Παραδείγματα: nfcutils

\subsubsection{blackarch-packer}
Πακέτα που χρησιμοποιούνται ως packers.

\textit{packers είναι προγράμματα που ενσωματώνουν άλλα προγράμματα
σε εκτελέσιμα αρχεία.}

Παράδειγμα: packerid

\subsubsection{blackarch-proxy}
Πακέτα που λειτουργούν ως proxy, δηλαδή ανακατευθύνοντας traffic
μέσω κάποιου άλλου κόμβου στο internet.

Παραδείγματα: burpsuite, ratproxy, sslnuke

\subsubsection{blackarch-recon}
Πακέτα που ψάχνουν για ευπάθειες γενικά.

Παραδείγματα: canri, dnsrecon, netmask

\subsubsection{blackarch-reversing}
Αυτό είναι μια ομάδα συλλογική για decompiler,
disassembler ή οτιδήποτε σχετικό.

Παραδείγματα: capstone, radare2, zerowine

\subsubsection{blackarch-scanner}
Πακέτα που ελέγχουν συστήματα για ευπάθειες.

Παραδείγματα: scanssh, tiger, zmap

\subsubsection{blackarch-sniffer}
Πακέτα που περιλαμβάνουν ανάλυση της κίνησης του δικτύου.

παραδείγματα: hexinject, pytactle, xspy

\subsubsection{blackarch-social}
Πακέτα που περιλαμβάνουν επιθέσεις με στόχο
μεσα κοινωνικής δικτύωσης.

Παραδείγματα: jigsaw, websploit

\subsubsection{blackarch-spoof}
Πακέτα που δρουν ως μάσκες και αποκρύπτουν την
πραγματική ταυτότητα του επιτιθέμενου στο θύμα.

Παραδείγματα: arpoison, lans, netcommander

\subsubsection{blackarch-threat-model}
Πακέτα για την ανάλυση και παρουσίαση του επιπέδου
απειλής σε κάποιο σενάριο.

Παράδειγμα: magictree

\subsubsection{blackarch-tunnel}
Πακέτα που χρησιμοποιούνται για περάσουν την κίνηση
ενός δικτύου μέσω tunnel.

Παραδείγματα: ctunnel, iodine, ptunnel

\subsubsection{blackarch-unpacker}
Πακέτα που χρησιμοποιούνται για την εξαγωγή 
πακεταρισμένου κακόβουλου λογισμικού.

Παράδειγμα: js-beautify

\subsubsection{blackarch-voip}
Πακέτα που λειτουργούν σε πρωτόκολλο VoIP.

Παραδείγματα: iaxflood, rtp-flood, teardown

\subsubsection{blackarch-webapp}
Πακέτα που λειτουργούν σε εφαρμογές προσώπου.

Παραδείγματα: metoscan, whatweb, zaproxy

\subsubsection{blackarch-windows}
Πακέτα για Windows που τρέχουν σε wine.

Παραδείγματα: 3proxy-win32, pwdump, winexe

\subsubsection{blackarch-wireless}
Πακέτα που λειτουργούν σε ασύρματα δίκτυα
σε κάθε επίπεδο.

Παραδείγματα: airpwn, mdk3, wiffy

\section{Repository structure}
Μπορείτε να βρείτε το κεντρικό BlackArch git repo εδώ:
\href{https://github.com/BlackArch/blackarch}{https://github.com/BlackArch/blackarch}.
Υπάρχουν επίσης και κάποια δευτερέυοντα repos εδώ:
\href{https://github.com/BlackArch}{https://github.com/BlackArch}.

Μέσα στο κύριο git repo, υπάρχουν 3 βασικόι κλάδοι:

\begin{itemize}
\item docs - Documentation.
\item packages - PKGBUILD files.
\item scripts - Useful little scripts.
\end{itemize}

\subsection{Scripts}
Εδώ είναι μια αναφορά των scripts στον κλάδο \verb|scripts/| :

\begin{itemize}
\item baaur - Σύντομα αυτό θα ανεβάζει πακέτα στο AUR.
\item babuild - Χτίσιμο πακέτου
\item bachroot - Διαχείρηση chroot για δοκιμή.
\item baclean - Καθαρισμός παλιών αρχείων .pkg.tar.xz από το repo.
\item baconflict - Σύντομα αυτό θα αντικαταστήσει το \verb|scripts/conflicts|.
\item bad-files - Εύρεση κακών αρχείων.
\item balock - Απόκτηση ή απελευθέρωση του κλειδώματος του repo.
\item banotify - Ειδοποίηση IRC για push πακέτων.
\item barelease - Ελευθέρωση πακέτων στο repo package.
\item baright - Τύπωση του BlackArch copyright.
\item basign - Υπογραφή πακέτων.
\item basign-key - Υπογραφή κλειδιού.
\item blackman - Κάτι σαν τον pacman, αλλά φτιάχνεται απο το git (not to be
    confused with nrz's Blackman).
\item check-groups - Έλεγχος ομάδων.
\item checkpkgs - Έλεγχος πακέτων για σφάλματα.
\item conflicts - Έλεγχος για συγκούσεις αρχείων
\item dbmod - Τροποποίηση βάσης κάπου πακέτου.
\item depth-list - Δημιουργία λίστας με βάση το βάθος εξαρτήσεων.
\item deptree - Δημιουργία δέντρου εξαρτήσεων, φαίνονται μόνο τα πακέτα απο BlackArch.
\item get-blackarch-deps - Λίστα με τις εξαρτήσεις απο το BlackArch σε ένα πακέτο.
\item get-official - Πάρε τα επίσημα πακέτα για μια έκδοση.
\item list-loose-packages - Λίστα πακέτων που δεν είναι σε ομάδες και δεν είναι σε εξάρτηση από άλλα πακέτα.
\item list-needed - Λίστα εξαρτήσεων που λείπουν.
\item list-removed - Λίστα πακέτων που είναι στο αποθετήριο αλλά όχι στο git.
\item list-tools - Λίστα εργαλείων.
\item outdated - Έλεγχος πακέτων που είναι παλιά σε σχέση με αυτά που είναι στο git.
\item pkgmod - Τροποποίηση πακέτου Build
\item pkgrel - Αύξηση pkgrel σε ένα πακέτο.
\item prep - Καθαρισμός ενός αρχείου PKGBUILD του στυλ και έλεγχος σφαλμάτων.
\item sitesync - Συγχρονισμός μεταξύ ενός τοπικού αρχείου repo και ενός απομακρυσμένου.
\item size-hunt - Κυνήγι μεγάλων πακετων.
\item source-backup - Αντίγραφο ασφαλείας πηγαίων αρχείων.
\end{itemize}

\section{Contributing to repository}
Αυτή η παράγραφος είναι για να σας βοηθήσει στο πως μπορείτε να συνεισφέρετε στο
project του BlackArch. Δεχόμαστε pull requests όλων των μεγεθών, απο μικρά ορθογραφικά
μέχρι και νέα πακετα.\\ Για βοήθεια, προτάσεις ή ερωτήσεις επικοινωνήστε μαζί μας.
\\\\
Όλοι είναι ελεύθεροι να συνεισφέρουν. Εκτιμούμε όλες τις συνεισφορές.

\subsection{Required tutorials}
Παρακαλούμε διαβάστε τους παρακάτω οδηγούς πριν συνεισφέρετε:
\begin{itemize}
\item
\href{https://wiki.archlinux.org/index.php/Arch\_Packaging\_Standards)}{Arch
Packaging Standards}
\item \href{https://wiki.archlinux.org/index.php/Creating\_Packages}{Creating
Packages}
\item \href{https://wiki.archlinux.org/index.php/PKGBUILD}{PKGBUILD}
\item \href{https://wiki.archlinux.org/index.php/Makepkg}{Makepkg}
\end{itemize}

\subsection{Steps for contributing}
Για να αποστείλετε τις αλλαγές σας ακολουθήστε τα παρακάτω βήματα:
\begin{enumerate}
\item Fork the repository from
\url{https://github.com/BlackArch/blackarch}
\item Hack the necessary files, (e.g. PKGBUILD, .patch files, etc).
\item Commit your changes.
\item Push your changes.
\item Ask us to merge in your changes, preferably through a pull request.
\end{enumerate}

\subsection{Example}
Το ακόλουθο παράδειγμα δείχνει πως στέλνουμε τις αλλαγές ενός πακέτου
στο BlackArch. Χρησιμοποιούμε \href{https://github.com/Jguer/yay}{yay}
(ή το pacaur, είναι το ίδιο)  για να φέρουμε τα υπαρχοντα PKGBUILD για
\textbf{nfsshell}  από \href{https://aur.archlinux.org/}{AUR} και τα
προσαρμόζουμε ανάλογα με τις ανάγκες μας.

\subsubsection{Fetch PKGBUILD}
Φέρνουμε το \textit{PKGBUILD} αρχείο χρησιμοποιόντας yay ή pacaur:
\begin{lstlisting}
  user@blackarchlinux $ yay -G nfsshell
  ==> Download nfsshell sources
  x LICENSE
  x PKGBUILD
  x gcc.patch
  user@blackarchlinux $ cd nfsshell/
\end{lstlisting}

\subsubsection{Clean up PKGBUILD}
Καθαρισμός \textit{PKGBUILD} αρχείου:
\begin{lstlisting}
  user@blackarchlinux nfsshell $ ./blarckarch/scripts/prep PKGBUILD
  cleaning 'PKGBUILD'...
  expanding tabs...
  removing vim modeline...
  removing id comment...
  removing contributor and maintainer comments...
  squeezing extra blank lines...
  removing '|| return'...
  removing leading blank line...
  removing $pkgname...
  removing trailing whitespace...
\end{lstlisting}

\subsubsection{Adjust PKGBUILD}
Προσαρμόζουμε το αρχείο \textit{PKGBUILD} :
\begin{lstlisting}
  user@blackarchlinux nfsshell $ vi PKGBUILD
\end{lstlisting}

\subsubsection{Build the package}
Build the package:
\begin{lstlisting}user@blackarchlinux nfsshell $ makepkg -sf
==> Making package: nfsshell 19980519-1 (Mon Dec  2 17:23:51 CET 2013)
==> Checking runtime dependencies...
==> Checking buildtime dependencies...
==> Retrieving sources...
-> Downloading nfsshell.tar.gz...
% Total    % Received % Xferd  Average Speed   Time    Time     Time
CurrentDload  Upload   Total   Spent    Left  Speed100 29213  100 29213    0
0  48150      0 --:--:-- --:--:-- --:--:-- 48206
-> Found gcc.patch
-> Found LICENSE
...
<lots of build process and compiler output here>
...
==> Leaving fakeroot environment.
==> Finished making: nfsshell 19980519-1 (Mon Dec  2 17:23:53 CET 2013)
\end{lstlisting}

\subsubsection{Install and test the package}
Install and test the package:
\begin{lstlisting}
  user@blackarchlinux nfsshell $ pacman -U nfsshell-19980519-1-x86_64.pkg.tar.xz
  user@blackarchlinux nfsshell $ nfsshell # test it
\end{lstlisting}

\subsubsection{Add, commit and push package}
Add, commit and push the package
\begin{lstlisting}user@blackarchlinux nfsshell $ cd /blackarchlinux/packages
user@blackarchlinux ~/blackarchlinux/packages $ mv ~/nfsshell .
user@blackarchlinux ~/blackarchlinux/packages $ git commit -am nfsshell && git push
\end{lstlisting}

\subsubsection{Create a pull request}
Create a pull request on \href{https://github.com/}{github.com}
\begin{lstlisting}
  firefox https://github.com/<contributor>/blackarchlinux
\end{lstlisting}

\subsubsection{Adding a remote for upstream}
A smart thing to do if you're working upstream and on a fork is to pull your own fork and add the main ba repo as a remote
\begin{lstlisting}
  user@blackarchlinux ~/blackarchlinux $ git remote -v
  origin <the url of your fork> (fetch)
  origin <the url of your fork> (push)
  user@blackarchlinux ~/blackarchlinux $ git remote add upstream https://github.com/blackarch/blackarch
  user@blackarchlinux ~/blackarchlinux $ git remote -v
  origin <the url of your fork> (fetch)
  origin <the url of your fork> (push)
  upstream https://github.com/blackarch/blackarch (fetch)
  upstream https://github.com/blackarch/blackarch (push)
\end{lstlisting}

By default, git should push straight to origin, but make sure your git config is
configured correctly. This won't be an issue unless you have commit rights as
you won't be able to push upstream without them.

If you do have the ability to commit, you might have more success using
\textit{git@github.com:blackarch/blackarch.git} but it's up to you.

\subsection{Requests}
\begin{enumerate}
\item Don't add \textbf{Maintainer} or \textbf{Contributor} comments to
\textit{PKGBUILD} files. Add maintainer and contributor names to the
AUTHORS section of BlackArch guide.
\item For the sake of consistency, please follow the general style of the other
\textit{PKGBUILD} files in the repo and use two-space indentation.
\end{enumerate}

\subsection{General tips}
\href{http://wiki.archlinux.org/index.php/Namcap}{namcap} can check packages for
errors.

%------------------%
%  Chapter 4       %
%------------------%

\chapter{Tools Guide}
Coming soon...

\section{Coming Soon}
Coming soon...

%%% APPENDIX %%%
\appendix
\include{latex/appendix-el}

\end{document}

%%% EOF %%%
