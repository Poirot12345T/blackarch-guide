%%%%%%%%%%%%%%%%%%%%%%%%%%%%%%%%%%%%%%%%%%%%%%%%%%%%%%%%%%%%%%%%%%%%%%%%%%%%%%%
%                                                                              %
% BlackArch Linux Guide                                                        %
%                                                                              %
%%%%%%%%%%%%%%%%%%%%%%%%%%%%%%%%%%%%%%%%%%%%%%%%%%%%%%%%%%%%%%%%%%%%%%%%%%%%%%%%

\documentclass[a4paper, oneside, 11pt]{book}

%%% INCLUDES %%%
\renewcommand{\familydefault}{\sfdefault}

\usepackage{array}
\usepackage{color}
\usepackage{enumerate}
\usepackage{fancyhdr}
\usepackage{fancyvrb}
\usepackage{geometry}
\usepackage{graphicx}
\usepackage{html}
\usepackage{hyperref}
\usepackage{ifpdf}
\usepackage{listings}
\usepackage{pstricks}
\usepackage{supertabular}
\usepackage{tocloft}

\usepackage{fontspec}
\setmainfont{CMU Serif}[Ligatures=TeX]
\setsansfont{CMU Sans Serif}[Ligatures=TeX]
\usepackage[russian]{babel}
\usepackage[utf8]{inputenc}
\usepackage[russian]{babel}
\usepackage[T1]{fontenc}

\renewcommand{\contentsname}{Содержание}
\renewcommand{\chaptername}{Глава}

%%% LAYOUT %%%
\setlength{\parindent}{0em}
\setlength{\parskip}{1.5ex plus0.5ex minus0.5ex}
\geometry{left=2.5cm, textwidth=16cm, top=3cm, textheight=25cm, bottom=3cm}
\widowpenalty=2000
\clubpenalty=1000
\frenchspacing
\sloppy
\pagecolor[HTML]{FFFFFF}
\color[HTML]{333333}
\setcounter{tocdepth}{10}
\setcounter{secnumdepth}{10}

\hypersetup{
  pdftitle={BlackArch Linux, Руководство по BlackArch Linux},
  pdfsubject={BlackArch Linux, Руководство по BlackArch Linux},
  pdfauthor={BlackArch Linux, BlackArch Linux},
  pdfkeywords={BlackArch Linux, Penetration Testing, Security, Hacking, Linux},
  pdfcenterwindow=true,
  colorlinks=true,
  breaklinks=true,
  linkcolor=blue,
  menucolor=blue,
  urlcolor=red
}

% syntax highlighting
% all options should be set here document wide
\lstset{
backgroundcolor=\color[HTML]{EEEEEE},
frame=single,
basicstyle=\footnotesize\ttfamily,
float,
deletekeywords={return},
otherkeywords={mkdir, curl, sudo, sha1sum, grep, cut, sort, wget, makepkg,
pacman, blackman, chmod},
keywordstyle=\color{orange},
commentstyle=\color{blue},
stringstyle=\color{red},
language=bash,
showspaces=false,
showtabs=false,
tabsize=2
}

%%% HEADER / FOOTER %%%
\setlength{\headheight}{33pt}
\setlength{\headsep}{33pt}
\lhead{{\includegraphics[width=1cm,height=1cm]{images/logo.png}}}
\rhead{Руководство по BlackArch Linux}

%%% CUSTOM MACROS %%%
% for html links
\ifpdf\else
\def\href#1#2{\htmladdnormallink{#2}{#1}}
\fi

%------------------%
%  TITLE PAGE      %
%------------------%
\begin{document}
\pagestyle{empty}
\begin{center}
\begin{figure}[htbp]
\centering
\vspace{0.5cm}
\includegraphics[width=8cm]{images/logo.png}
\label{fig:logo}
\end{figure}
\vspace{0.5cm}
\Huge{\textbf{The BlackArch Linux Guide}}\\
\vspace{1cm}
\Large{\color{blue}https://www.blackarch.org/}\\
\vspace{0.5cm}
\end{center}
\newpage
\tableofcontents
\newpage
\pagestyle{fancy}

%------------------%
%  Chapter 1       %
%------------------%

\chapter{Введение}

\section{Обзор}
Руководство BlackArch Linux разделено на несколько частей:
\begin{itemize}
\item Введение - Предоставляет широкий обзор, введение и дополнительную полезную информацию о проекте
\item Руководство Пользователя - Все, что обычный пользователь должен знать, чтобы эффективно использовать BlackArch
\item Руководство Разработчика - Как начать разработку и внесение вклада в BlackArch
\item Руководство по инструментам - Подробные сведения об инструменте по примеру использования (WIP)
\end{itemize}

\section{Что такое BlackArch Linux?}
BlackArch представляет собой полный Linux дистрибутив для тестеров на прониконовение и исследователей безопасности.
Он основан на \href{https://www.archlinux.org/}{ArchLinux} и пользователи могут установить компоненты BlackArch
лично или группами.

Набор инструментов распраняется как
\href{https://wiki.archlinux.org/index.php/Unofficial\_User\_Repositories}
{неофициальный пользовательский репозиторий} Arch Linux, поэтому вы можете установить BlackArch поверх
существующего Arch Linux. Пакеты могут устанавливаться отдельно или категориями.

Постоянно расширяющийся репозиторий в настоящее время включает в себя \href{https://www.blackarch.org/tools.html}{2600} инструментов.
Все инструменты тщательно тестируются перед добавлением в кодовую базу для поддержания качества репозитория.
% should quickly describe the testing methods/code review procedures etc

\section{История BlackArch Linux}
Coming soon...

\section{Поддерживаемые платформы}
Coming soon...

\section{Принять участие}
Вы можете связаться с командой BlackArch, используя следующие возможности:

Website: \url{https://www.blackarch.org/}

Mail: \href{mailto:team@blackarch.org}{team@blackarch.org}

IRC: \url{irc://irc.freenode.net/blackarch}

Twitter: \url{https://twitter.com/blackarchlinux}

Github: \url{https://github.com/Blackarch/}

Matrix: \url{https://matrix.to/#/#BlackArch:matrix.org}

%------------------%
%  Chapter 2       %
%------------------%

\chapter{Руководство Пользователя}

\section{Установка}
В следующих разделах рассказывается, как настроить репозиторий BlackArch и
установить пакеты. BlackArch поддерживает оба варианта, установка из репозитория 
с использованием бинарных пакетов и их компиляция и установка из источников.

BlackArch совместим с обычной установкой Arch. Он выступает в качестве
неофициального пользовательского репозитория. Если вместо этого вы хотите ISO,
см. Раздел \href{https://www.blackarch.org/downloads.html#iso}{Live ISO}.

\subsection{Установка поверх ArchLinux}
Запустите \href{https://blackarch.org/strap.sh}{strap.sh} с правами админа(root)
и следуйте инструкциям. Смотрите следующий пример.
\begin{lstlisting}
   curl -O https://blackarch.org/strap.sh
   sha1sum strap.sh # should match: 5ea40d49ecd14c2e024deecf90605426db97ea0c
   sudo chmod +x strap.sh
   sudo ./strap.sh
\end{lstlisting}

Теперь загрузите свежую копию master package list и выполните синхронизацию пакетов:
\begin{lstlisting}
  sudo pacman -Syyu
\end{lstlisting}


\subsection{Установка пакетов}
Теперь вы можете установить инструменты из репозитория blackarch.
\begin{enumerate}
\item Чтобы просмотреть все доступные инструменты, выполните
\begin{lstlisting}
  pacman -Sgg | grep blackarch | cut -d' ' -f2 | sort -u
\end{lstlisting}

\item Чтобы установить все инструменты, выполните
\begin{lstlisting}
  pacman -S blackarch
\end{lstlisting}

\item Чтобы установить категорию инструментов, выполните
\begin{lstlisting}
  pacman -S blackarch-<category>
\end{lstlisting}

\item Чтобы посмотреть категории blackarch, выполните 
\begin{lstlisting}
  pacman -Sg | grep blackarch
\end{lstlisting}

\end{enumerate}

\subsection{Установка пакетов из исходников}
В рамках альтернативного метода установки вы можете собрать BlackArch
пакеты из исходников. Вы можете найти PKGBUILDы на
\href{https://github.com/BlackArch/blackarch/tree/master/packages}{github}.
Для сборки всего репозитория, вы можете использовать инструмент
\href{https://github.com/BlackArch/blackman}{Blackman}.
\begin{itemize}
\item Во-первых, вам нужно установить Blackman. Если на вашем компьютере настроен
репозиторий пакетов BlackArch, вы можете установить Blackman:
\begin{lstlisting}
  pacman -S blackman
\end{lstlisting}

\item Вы можете собрать и установить Blackman из исходников:
\begin{lstlisting}
  mkdir blackman
  cd blackman
  wget https://raw.github.com/BlackArch/blackarch/master/packages/blackman/PKGBUILD
  # Make sure the PKGBUILD has not been maliciously tampered with.
  makepkg -s
\end{lstlisting}

\item Или вы можете установить Blackman из AUR:
\begin{lstlisting}
  <whatever AUR helper you use> -S blackman
\end{lstlisting}

\end{itemize}

\subsection{Основное использование Blackman} Blackman очень прост в использовании, хотя флаги отличаются от того, 
чего вы обычно ожидаете от pacman. Основное использование приведено ниже.
\begin{itemize}
\item Скачать, скомпилировать и установить пакеты:
\begin{lstlisting}
  sudo blackman -i package
\end{lstlisting}

\item Скачать, скомпилировать и установить целую категорию:
\begin{lstlisting}
  sudo blackman -g group
\end{lstlisting}

\item Скачать, скомпилировать и установить все инструменты BlackArch:
\begin{lstlisting}
  sudo blackman -a
\end{lstlisting}

\item Список blackarch категорий:
\begin{lstlisting}
  blackman -l
\end{lstlisting}

\item Список категорий инструментов:
\begin{lstlisting}
  blackman -p category
\end{lstlisting}

\end{itemize}

\subsection{Установка из live-, netinstall- ISO или ArchLinux}
Вы можете установить BlackArch Linux из одного из наших live- or netinstall- ISOs.\\См.
\url{https://www.blackarch.org/download.html#iso}. После загрузки ISO необходимо выполнить следующие шаги.
\begin{itemize}
\item Установка пакета blackarch-installer:
\begin{lstlisting}
  sudo pacman -S blackarch-installer
\end{lstlisting}

\item Запуск
\begin{lstlisting}
  sudo blackarch-install
\end{lstlisting}

\end{itemize}

%------------------%
%  Chapter 3       %
%------------------%

\chapter{Руководство разработчика}

\section{Система сборки и Репозиториев Arch}

Файлы PKGBUILD - это скрипты сборки. Каждый из них сообщает makepkg (1), как создать пакет.
Файлы PKGBUILD написаны на Bash.

Для получения дополнительной информации прочтите следующее:
\begin{itemize}
\item \href{https://wiki.archlinux.org/index.php/Creating_Packages}{Arch Wiki: Creating Packages}
\item \href{https://wiki.archlinux.org/index.php/Makepkg}{Arch Wiki: makepkg}
\item \href{https://wiki.archlinux.org/index.php/PKGBUILD}{Arch Wiki: PKGBUILD}
\item \href{https://wiki.archlinux.org/index.php/Arch_Packaging_Standards}{Arch Wiki: Arch Packaging Standards}
\end{itemize}

\section{Стандарты Blackarch PKGBUILD}
Ради простоты, наши PKGBUILDы аналогичны характеристикам AUR, с несколькими
небольшими различиями, описанными ниже. Каждый пакет должен как минимум
предналежать к blackarch; многие пакеты могут принадлежать более чем одной группе.

\subsection{Группы}
Чтобы разрешить пользователям устанавливать быстро и легко определенный диапазон пакетов,
пакеты были разделены на группы. Группы позволяют пользователям перейти в\\"pacman -S <group name>"
чтобы получить множество пакетов.

\subsubsection{blackarch}
Группа blackarch - это основная группа в которую входят все пакеты. Это позволяет
пользователям с легкостью устанавливать каждый пакет.

Что должно быть здесь: Все.

\subsubsection{blackarch-anti-forensic}
Пакеты которые используются для противодейстивия судебной деятельности,
включая шифрование, стеганографию и все, что изменяет атрибуты файлов/файла.
Все это включает в себя инструменты для работы с чем угодно, которые вносят
изменения в систему в целях сокрытия информации.

Примеры: luks, TrueCrypt, Timestomp, dd, ropeadope, secure-delete

\subsubsection{blackarch-automation}
Пакеты, используемые для автоматизации рабочих процессов(workflow automation).

Примеры: blueranger, tiger, wiffy

\subsubsection{blackarch-backdoor}
Пакеты, которые используют уязвимости или бэкдоры на уже уязвимых системах.

Примеры: backdoor-factory, rrs, weevely

\subsubsection{blackarch-binary}
Пакеты, которые обрабатывают бинарные файлы в некоторой форме.

Примеры: binwally, packerid

\subsubsection{blackarch-bluetooth}
Пакеты, которые используют любые уязвимости касаемо стандарта Bluetooth (802.15.1).

Примеры: ubertooth, tbear, redfang

\subsubsection{blackarch-code-audit}
Пакеты, проводящие аудит существующего исходного кода для анализа уязвимости.

Примеры: flawfinder, pscan

\subsubsection{blackarch-cracker}
Пакеты, используемые для взлома криптографических функций, т.е. хешей.

Примеры: hashcat, john, crunch

\subsubsection{blackarch-crypto}
Пакеты, работающие с криптографией, за исключением взлома.

Примеры: ciphertest, xortool, sbd

\subsubsection{blackarch-database}
Пакеты, связанные с эксплуатацией базы данных на любом уровне.

Примеры: metacoretex, blindsql

\subsubsection{blackarch-debugger}
Пакеты, которые позволяют пользователю просматривать то, 
что определенная программа "делает" в режиме реального времени.

Примеры: radare2, shellnoob

\subsubsection{blackarch-decompiler}
Пакеты, используемые для просмотра исходного кода уже скомпилированных программ.

Примеры: flasm, jd-gui

\subsubsection{blackarch-defensive}
Пакеты, которые используются для защиты пользователя от вредоносных программ
и от атак других пользователей.

Примеры: arpon, chkrootkit, sniffjoke

\subsubsection{blackarch-disassembler}
Пакеты, преобразующие программу с машинного кода в текст программы на 
языке ассемблера.

Примеры: inguma, radare2, smali

\subsubsection{blackarch-dos}
Пакеты, используемые для DoS-атак, цель которых довести систему до отказа.

Примеры: 42zip, nkiller2

\subsubsection{blackarch-drone}
Пакеты, которые используются для управления физически сконструированными дронами.

Примеры: meshdeck, skyjack

\subsubsection{blackarch-exploitation}
Пакеты, которые используют уязвимости в других программах или службах.

Примеры: armitage, metasploit, zarp

\subsubsection{blackarch-fingerprint}
Пакеты, использующие идентификацию по "отпечаткам пальцев".

Примеры: dns-map, p0f, httprint

\subsubsection{blackarch-firmware}
Пакеты, которые используют уязвимости в прошивке.

Примеры: None yet, amend asap.

\subsubsection{blackarch-forensic}
Пакеты, которые используются для поиска данных на физических дисках 
или встроенной памяти.

Примеры: aesfix, nfex, wyd

\subsubsection{blackarch-fuzzer}
Пакеты, в которых используется принцип fuzz-тестирования (фаззинга),
заключающийся в передаче приложению на вход неправильных, неожиданных 
или случайных данных, чтобы проверить надёжность ПО и компьютерных систем.

Примеры: msf, mdk3, wfuzz

\subsubsection{blackarch-hardware}
Пакеты, которые используют или управляют чем-либо, связанным
с физическим оборудованием.

Примеры: arduino, smali

\subsubsection{blackarch-honeypot}
Пакеты, которые действуют как "приманки", т.е. программы, которые
оказались уязвимыми службами, используемыми для привлечение
хакеров в ловушку.

Примеры: artillery, bluepot, wifi-honey

\subsubsection{blackarch-keylogger}
Пакеты, которые записывают и сохраняют нажатия клавиш на другой системе.

Примеры: None yet, amend asap.

\subsubsection{blackarch-malware}
Пакеты, обнаруживающие любые типы вредоносного ПО.

Примеры: malwaredetect, peepdf, yara

\subsubsection{blackarch-misc}
Пакеты, которые не вписываются ни в какие категории; разнообразные пакеты.

Примеры: oh-my-zsh-git, winexe, stompy

\subsubsection{blackarch-mobile}
Пакеты, которые работают с мобильными платформами.

Примеры: android-sdk-platform-tools, android-udev-rules

\subsubsection{blackarch-networking}
Пакет, который включает в себя IP-networking.

Примеры: Anything pretty much

\subsubsection{blackarch-nfc}
пакеты, которые используют nfc (near-field communications).

Примеры: nfcutils

\subsubsection{blackarch-packer}
Пакеты, которые оперируют с упаковщиками или связаны с ними.

\textit{Упаковщики - это программы, которые внедряют вредоносное ПО в другие исполняемые файлы.}

Примеры: packerid

\subsubsection{blackarch-proxy}
Пакеты, которые действуют как прокси-сервер, т.е.
перенаправляют трафик через другой узел в Интернете.

Примеры: burpsuite, ratproxy, sslnuke

\subsubsection{blackarch-recon}
Packages that actively seeks vulnerable exploits in the
wild. More of an umbrella group for similar packages.

Примеры: canri, dnsrecon, netmask

\subsubsection{blackarch-reversing}
Это группа umbrella для любого декомпилятора,
дизассемблера или любой подобной программы.

Примеры: capstone, radare2, zerowine

\subsubsection{blackarch-scanner}
Пакеты, которые сканируют выбранные системы на наличие уязвимостей.

Примеры: scanssh, tiger, zmap

\subsubsection{blackarch-sniffer}
Пакеты, которые включают в себя анализ сетевого трафика.

Примеры: hexinject, pytactle, xspy

\subsubsection{blackarch-social}
Пакеты, которые в первую очередь атакуют сайты социальных сетей.

Примеры: jigsaw, websploit

\subsubsection{blackarch-spoof}
Пакеты, которые пытаются обмануть атакующего таким образом,
в котором атакующий не появляется жертве в качестве атакующего.

Примеры: arpoison, lans, netcommander

\subsubsection{blackarch-threat-model}
Пакеты, которые будут использоваться для отчетов/записи 
модели угрозы, изложенной в конкретном сценарии.

Примеры: magictree

\subsubsection{blackarch-tunnel}
Пакеты, которые используются для туннелирования 
сетевого трафика в данной сети.

Примеры: ctunnel, iodine, ptunnel

\subsubsection{blackarch-unpacker}
Пакеты, которые используются для извлечения предварительно 
упакованных вредоносных программ из исполняемого файла.

Примеры: js-beautify

\subsubsection{blackarch-voip}
Пакеты, которые оперируют voip программами и протоколами.

Примеры: iaxflood, rtp-flood, teardown

\subsubsection{blackarch-webapp}
Пакеты, которые оперируют internet-facing приложениями.

Примеры: metoscan, whatweb, zaproxy

\subsubsection{blackarch-windows}
Эта группа предназначена для любого родного пакета Windows,
который работает через wine.

Примеры: 3proxy-win32, pwdump, winexe

\subsubsection{blackarch-wireless}
Пакеты, которые оперируют беспроводными сетями на любом уровне.

Примеры: airpwn, mdk3, wiffy

\section{Струкрура репозитория}
Вы можете найти главный git репозиторий BlackArch тут:\\
\href{https://github.com/BlackArch/blackarch}{https://github.com/BlackArch/blackarch}.\\
\\Также имеется несколько вторичных репозиториев:\\
\href{https://github.com/BlackArch}{https://github.com/BlackArch}.

В основном git репозитории есть 3 важных каталога:

\begin{itemize}
\item docs - Документация.
\item packages - PKGBUILD файлы.
\item scripts - Полезные небольшие скрипты.
\end{itemize}

\subsection{Скрипты}
Вот ссылка на скрипты в \verb|scripts/| каталоге:

\begin{itemize}
\item baaur - Soon, this will upload packages to the AUR.
\item babuild - Сборка пакета.
\item bachroot - Управление chroot для тестирования.
\item baclean - Очистить старые .pkg.tar.xz файлы из репозитория пакетов.
\item baconflict - Вскорее, это заменит \verb|scripts/conflicts|.
\item bad-files - Поиск плохих файлов в встроенных пакетах.
\item balock - Obtain or release the package repo lock.
\item banotify - Notify IRC about package pushes.
\item barelease - Выпуск пакетов в релиз к репозиторию пакетов. 
\item baright - Отобразить информацию об авторских правах BlackArch.
\item basign - Подпись пакетов.
\item basign-key - Подпись ключа.
\item blackman - Ведет себя как pacman но собирается из git
    (не путать с nrz's Blackman).
\item check-groups - Проверка групп.
\item checkpkgs - Проверить пакеты на наличие ошибок.
\item conflicts - Проверить наличие конфликтов файлов.
\item dbmod - Изменение базы данных пакета.
\item depth-list - Создать список, отсортированный по глубине зависимостей.
\item deptree - Создать дерево зависимостей, перечисляя пакеты предоставляемые 
    только blackarch.
\item get-blackarch-deps - Получить список зависимостей для пакета.
\item get-official - Получить официальные пакеты для релиза.
\item list-loose-packages - Список пакетов, которые не входят в группы,
    и не зависят от других пакетов.
\item list-needed - Список недостающих зависимостей.
\item list-removed - Список пакетов, которые находятся в репозитории пакетов, но не в git.
\item list-tools - Список инструментов.
\item outdated - Поиск пакетов, которые устарели в репозитории пакетов
    относительно git репозитория.
\item pkgmod - Изменение сборки пакета.
\item pkgrel - Инкремент pkgrel в пакете.
\item prep - Очистка стиля PKGBUILD файла и поиск ошибок.
\item sitesync - Синхронизация между локальной копией репозитория пакетов и удаленной копией.
\item size-hunt - Поиски больших пакетов.
\item source-backup - Резервные копии исходных файлов.
\end{itemize}

\section{Вклад в репозиторий}
В этом разделе показано, как внести вклад в проект BlackArch Linux. Мы
принимаем pull requests всех размеров, от крошечных изменений до новых пакетов.\\За
помощью, предложениями или вопросами вы можете связаться с нами.
\\\\
Все желающие могут внести свой вклад. Все вклады приветствуются.

\subsection{Необходимые туториалы}
Прочтите следующие туториалы, прежде чем вносить свой вклад:
\begin{itemize}
\item
\href{https://wiki.archlinux.org/index.php/Arch\_Packaging\_Standards)}{Arch
Packaging Standards}
\item \href{https://wiki.archlinux.org/index.php/Creating\_Packages}{Creating
Packages}
\item \href{https://wiki.archlinux.org/index.php/PKGBUILD}{PKGBUILD}
\item \href{https://wiki.archlinux.org/index.php/Makepkg}{Makepkg}
\end{itemize}

\subsection{Шаги по содействию}
Чтобы внести изменения в проект BlackArch Linux, выполните следующие действия:
\begin{enumerate}
\item Сделать форк репозитория
\url{https://github.com/BlackArch/blackarch}
\item Hack the necessary files, (e.g. PKGBUILD, .patch files, etc).
\item Закомитьте свои зменения.
\item Отправьте свои изменения в главный репозиторий.
\item Ask us to merge in your changes, предпочтительно через pull request.
\end{enumerate}

\subsection{Пример}
Следующий пример демонстрирует отправку нового пакета в проект BlackArch.
Мы используем \href{https://wiki.archlinux.org/index.php/yaourt}{yaourt}
(вы также можете использовать pacaur) для извлечения уже существующего PKGBUILD
файла для \textbf{nfsshell} из \href{https://aur.archlinux.org/}{AUR} и настройте
его в соответствии с нашими потребностями.

\subsubsection{Извлечение PKGBUILD}
Получить \textit{PKGBUILD} файл используя yaourt или pacaur:
\begin{lstlisting}
  user@blackarchlinux $ yaourt -G nfsshell
  ==> Download nfsshell sources
  x LICENSE
  x PKGBUILD
  x gcc.patch
  user@blackarchlinux $ cd nfsshell/
\end{lstlisting}

\subsubsection{Очистка PKGBUILD}
Очистка \textit{PKGBUILD} файла и сохранение времени:
\begin{lstlisting}
  user@blackarchlinux nfsshell $ ./blackarch/scripts/prep PKGBUILD
  cleaning 'PKGBUILD'...
  expanding tabs...
  removing vim modeline...
  removing id comment...
  removing contributor and maintainer comments...
  squeezing extra blank lines...
  removing '|| return'...
  removing leading blank line...
  removing $pkgname...
  removing trailing whitespace...
\end{lstlisting}

\subsubsection{Настройка PKGBUILD}
Настройка \textit{PKGBUILD} файл:
\begin{lstlisting}
  user@blackarchlinux nfsshell $ vi PKGBUILD
\end{lstlisting}

\subsubsection{Сборка пакета}
Сборка пакета:
\begin{lstlisting}user@blackarchlinux nfsshell $ makepkg -sf
==> Making package: nfsshell 19980519-1 (Mon Dec  2 17:23:51 CET 2013)
==> Checking runtime dependencies...
==> Checking buildtime dependencies...
==> Retrieving sources...
-> Downloading nfsshell.tar.gz...
% Total    % Received % Xferd  Average Speed   Time    Time     Time
CurrentDload  Upload   Total   Spent    Left  Speed100 29213  100 29213    0
0  48150      0 --:--:-- --:--:-- --:--:-- 48206
-> Found gcc.patch
-> Found LICENSE
...
<lots of build process and compiler output here>
...
==> Leaving fakeroot environment.
==> Finished making: nfsshell 19980519-1 (Mon Dec  2 17:23:53 CET 2013)
\end{lstlisting}

\subsubsection{Установка и тестирование пакета}
Установка и тестирование пакета
\begin{lstlisting}
  user@blackarchlinux nfsshell $ pacman -U nfsshell-19980519-1-x86_64.pkg.tar.xz
  user@blackarchlinux nfsshell $ nfsshell # test it
\end{lstlisting}

\subsubsection{Add, commit и push пакета}
Add, commit и push пакета 
\begin{lstlisting}user@blackarchlinux nfsshell $ cd /blackarchlinux/packages
user@blackarchlinux ~/blackarchlinux/packages $ mv ~/nfsshell .
user@blackarchlinux ~/blackarchlinux/packages $ git commit -am nfsshell && git push
\end{lstlisting}

\subsubsection{Создать pull request}
Создайте pull request на \href{https://github.com/}{github.com}
\begin{lstlisting}
  firefox https://github.com/<contributor>/blackarchlinux
\end{lstlisting}

\subsubsection{Adding a remote for upstream}
A smart thing to do if you're working upstream and on a fork is to pull your own fork and add the main ba repo as a remote
\begin{lstlisting}
  user@blackarchlinux ~/blackarchlinux $ git remote -v
  origin <the url of your fork> (fetch)
  origin <the url of your fork> (push)
  user@blackarchlinux ~/blackarchlinux $ git remote add upstream https://github.com/blackarch/blackarch
  user@blackarchlinux ~/blackarchlinux $ git remote -v
  origin <the url of your fork> (fetch)
  origin <the url of your fork> (push)
  upstream https://github.com/blackarch/blackarch (fetch)
  upstream https://github.com/blackarch/blackarch (push)
\end{lstlisting}

By default, git should push straight to origin, but make sure your git config is
configured correctly. This won't be an issue unless you have commit rights as
you won't be able to push upstream without them.

If you do have the ability to commit, you might have more success using
\textit{git@github.com:blackarch/blackarch.git} but it's up to you.

\subsection{Requests}
\begin{enumerate}
\item Don't add \textbf{Maintainer} or \textbf{Contributor} comments to
\textit{PKGBUILD} files. Add maintainer and contributor names to the
AUTHORS section of BlackArch guide.
\item For the sake of consistency, please follow the general style of the other
\textit{PKGBUILD} files in the repo and use two-space indentation.
\end{enumerate}

\subsection{Общие советы}
\href{http://wiki.archlinux.org/index.php/Namcap}{namcap} может проверять пакеты 
на наличие ошибок.

%------------------%
%  Chapter 4       %
%------------------%

\chapter{Руководство по инструментам}
Coming soon...

\section{Coming Soon}
Coming soon...

%%% APPENDIX %%%
\appendix
\include{latex/appendix-ru}

\end{document}

%%% EOF %%%
