%%%%%%%%%%%%%%%%%%%%%%%%%%%%%%%%%%%%%%%%%%%%%%%%%%%%%%%%%%%%%%%%%%%%%%%%%%%%%%%%
%                                                                              %
% BlackArch Linux Guide                                                        %
%                                                                              %
%%%%%%%%%%%%%%%%%%%%%%%%%%%%%%%%%%%%%%%%%%%%%%%%%%%%%%%%%%%%%%%%%%%%%%%%%%%%%%%%

\documentclass[a4paper, oneside, 11pt]{book}

%%% INCLUDES %%%
\renewcommand{\familydefault}{\sfdefault}

\usepackage{array}
\usepackage{color}
\usepackage{enumerate}
\usepackage{fancyhdr}
\usepackage{fancyvrb}
\usepackage{geometry}
\usepackage{graphicx}
\usepackage{html}
\usepackage{hyperref}
\usepackage{ifpdf}
\usepackage{listings}
\usepackage{pstricks}
\usepackage{supertabular}
\usepackage{tocloft}
\usepackage[utf8]{inputenc}

%%% CHINESE %%%
\usepackage{ctex}
\renewcommand{\chaptername}{章节}

%%% LAYOUT %%%
\setlength{\parindent}{0em}
\setlength{\parskip}{1.5ex plus0.5ex minus0.5ex}
\geometry{left=2.5cm, textwidth=16cm, top=3cm, textheight=25cm, bottom=3cm}
\widowpenalty=2000
\clubpenalty=1000
\frenchspacing
\sloppy
\pagecolor[HTML]{FFFFFF}
\color[HTML]{333333}
\setcounter{tocdepth}{10}
\setcounter{secnumdepth}{10}

\hypersetup{
  pdftitle={BlackArch Linux, BlackArch Linux 指南},
  pdfsubject={BlackArch Linux, BlackArch Linux 指南},
  pdfauthor={BlackArch Linux, BlackArch Linux},
  pdfkeywords={BlackArch Linux, Penetration Testing, Security, Hacking, Linux},
  pdfcenterwindow=true,
  colorlinks=true,
  breaklinks=true,
  linkcolor=blue,
  menucolor=blue,
  urlcolor=blue
}

% syntax highlighting
% all options should be set here document wide
\lstset{
backgroundcolor=\color[HTML]{EEEEEE},
frame=single,
basicstyle=\footnotesize\ttfamily,
float,
deletekeywords={return},
otherkeywords={mkdir, curl, sudo, sha1sum, grep, cut, sort, wget, makepkg,
pacman, blackman},
keywordstyle=\color{orange},
commentstyle=\color{blue},
stringstyle=\color{red},
language=bash,
showspaces=false,
showtabs=false,
tabsize=2
}

%%% HEADER / FOOTER %%%
\setlength{\headheight}{33pt}
\setlength{\headsep}{33pt}
\lhead{{\includegraphics[width=1cm,height=1cm]{images/logo.png}}}
\rhead{BlackArch Linux 指南}

%%% CUSTOM MACROS %%%
% for html links
\ifpdf\else
\def\href#1#2{\htmladdnormallink{#2}{#1}}
\fi

%------------------%
%  TITLE PAGE      %
%------------------%
\begin{document}
\pagestyle{empty}
\begin{center}
\begin{figure}[htbp]
\centering
\vspace{0.5cm}
\includegraphics[width=8cm]{images/logo.png}
\label{fig:logo}
\end{figure}
\vspace{0.5cm}
\Huge{\textbf{BlackArch Linux 指南}}\\
\vspace{1cm}
\Large{\color{blue}https://www.blackarch.org/}\\
\vspace{0.5cm}
\end{center}
\newpage
\tableofcontents
\newpage
\pagestyle{fancy}

%------------------%
%  Chapter 1       %
%------------------%

\chapter{引言}

\section{概览}
BlackArch Linux 指南分为以下几个部分:
\begin{itemize}
\item 引言 - 提供概述,简介,以及一些有用项目的信息
\item 用户指南 - 一个标准用户高效使用BlackArch需要了解的
\item 开发者指南 - 如何参与BlackArch的开发以及如何贡献代码
\item 工具指南 - 用样例深度详细的介绍工具的使用(WIP)
\end{itemize}

\section{什么是BlackArch Linux?}
BlackArch 是一个提供给渗透测试以及安全人员的完整的Linux发行版。
它是 \href{https://www.archlinux.org/}{ArchLinux} 的一个分支,用户可以在ArchLinux上安装单个或一组BlackArch组件。

工具集也被打包成 Arch Linux
\href{https://wiki.archlinux.org/index.php/Unofficial\_User\_Repositories}
{非官方用户仓库(Unofficial User Repositories)},所以你可以在一个已有的Arch Linux系统上安装。可以单独安装某个工具或按分类批量安装。

仓库目前包括超过 \href{https://www.blackarch.org/tools.html}{2600} 个工具,仍在不断扩充。
为了保证仓库的质量,所有的工具在添加到代码库之前都经过测试。
% should quickly describe the testing methods/code review procedures etc

\section{BlackArch Linux 的历史}
敬请期待

\section{支持的平台}
敬请期待

\section{参与进来}
你可以通过以下几个途径联系到BlackArch 团队:

网站: \url{https://www.blackarch.org/}

邮箱: \href{mailto:team@blackarch.org}{team@blackarch.org}

IRC: \url{irc://irc.freenode.net/blackarch}

Twitter: \url{https://twitter.com/blackarchlinux}

Github: \url{https://github.com/Blackarch/}

Matrix: \url{https://matrix.to/#/#BlackArch:matrix.org}

%------------------%
%  Chapter 2       %
%------------------%


\chapter{用户指南}

\section{安装}
接下来的几个小节会告诉你如何配置BlackArch仓库以及如何安装包。
BlackArch支持用二进制安装包安装或是从代码安装。

BlackArch 兼容 Arch Linux 常用的安装方式,比如说非官方用户仓库(Unofficial User Repositories)。
如果你想使用ISO,参考
\href{https://www.blackarch.org/downloads.html#iso}{Live ISO}。

\subsection{在ArchLinux上安装}
用root权限运行脚本 \href{https://blackarch.org/strap.sh}{strap.sh},并跟随提示,看下面的例子:
\begin{lstlisting}
   curl -O https://blackarch.org/strap.sh
   sha1sum strap.sh # should match: 5ea40d49ecd14c2e024deecf90605426db97ea0c
   sudo chmod +x strap.sh
   sudo ./strap.sh
\end{lstlisting}

现在同步软件包列表并更新软件包:
\begin{lstlisting}
  sudo pacman -Syyu
\end{lstlisting}


\subsection{安装软件包}
你现在可以安装blackarch 仓库的软件了。
\begin{enumerate}
\item 列出所有可用的工具:
\begin{lstlisting}
  pacman -Sgg | grep blackarch | cut -d' ' -f2 | sort -u
\end{lstlisting}

\item 安装所有的工具:
\begin{lstlisting}
  pacman -S blackarch
\end{lstlisting}

\item 按照分类批量安装工具:
\begin{lstlisting}
  pacman -S blackarch-<category>
\end{lstlisting}

\item 列出 blackarch的所有分类:
\begin{lstlisting}
  pacman -Sg | grep blackarch
\end{lstlisting}

\end{enumerate}

\subsection{从源代码安装软件包}
作为一种替代的安装方式,你可以从源代码构建BlackArch的安装包。
所需的PKGBUILD文件在
\href{https://github.com/BlackArch/blackarch/tree/master/packages}{github}。要编译整个仓库, 可以使用
\href{https://github.com/BlackArch/blackman}{Blackman} 工具。
\begin{itemize}
\item 首先,你需要安装Blackman,如果你已经按照前面的步骤配置好了BlackArch软件包仓库,
可以通过下面的命令安装Blackman:
\begin{lstlisting}
  pacman -S blackman
\end{lstlisting}

\item 你也可以从源码安装Blackman:
\begin{lstlisting}
  mkdir blackman
  cd blackman
  wget https://raw.github.com/BlackArch/blackarch/master/packages/blackman/PKGBUILD
  # Make sure the PKGBUILD has not been maliciously tampered with.
  makepkg -s
\end{lstlisting}

\item 或者从AUR安装Blackman:
\begin{lstlisting}
  <whatever AUR helper you use> -S blackman
\end{lstlisting}

\end{itemize}

\subsection{基础的 Blackman 用法} Blackman非常易于使用,虽然参数和一些pacman之类的标准包管理器不同。
以下是一些基本的用法。
\begin{itemize}
\item 下载,编译并安装软件包:
\begin{lstlisting}
  sudo blackman -i package
\end{lstlisting}

\item 下载,编译并安装整个分类的软件包:
\begin{lstlisting}
  sudo blackman -g group
\end{lstlisting}

\item 下载,编译并安装所有的BlackArch工具:
\begin{lstlisting}
  sudo blackman -a
\end{lstlisting}

\item 列出所有的BlackArch软件包分类:
\begin{lstlisting}
  blackman -l
\end{lstlisting}

\item 列出某个分类包含的软件:
\begin{lstlisting}
  blackman -p category
\end{lstlisting}

\end{itemize}

\subsection{从live-ISO, netinstall-ISO 或 ArchLinux安装}
你可以从 live-ISO 或 netinstall-ISO 安装BlackArch Linux\\查看
\url{https://www.blackarch.org/download.html#iso}。从ISO 启动后执行下面几个步骤:
\begin{itemize}
\item 安装 blackarch-installer:
\begin{lstlisting}
  sudo pacman -S blackarch-installer
\end{lstlisting}

\item 然后运行:
\begin{lstlisting}
  sudo blackarch-install
\end{lstlisting}

\end{itemize}

%------------------%
%  Chapter 3       %
%------------------%

\chapter{开发者指南}

\section{Arch的构建系统和仓库}

PKGBUILD 文件是构建脚本,它告诉 makepkg(1) 如何创建一个安装包,
PKGBUILD 文件用Bash脚本编写。

要获取更多信息,可以参阅以下内容:
\begin{itemize}
\item \href{https://wiki.archlinux.org/index.php/Creating_Packages}{Arch Wiki: Creating Packages}
\item \href{https://wiki.archlinux.org/index.php/Makepkg}{Arch Wiki: makepkg}
\item \href{https://wiki.archlinux.org/index.php/PKGBUILD}{Arch Wiki: PKGBUILD}
\item \href{https://wiki.archlinux.org/index.php/Arch_Packaging_Standards}{Arch Wiki: Arch Packaging Standards}
\end{itemize}

\section{Blackarch 的 PKGBUILD 规范}
为了简单起见,我们的PKGBUILD文件和AUR的非常相似,下面列出了不同的一些地方。
每个软件包必须至少属于blackarch,也会有许多软件包同时属于多个组。

\subsection{组}
为了让用户快速方便的安装一系列软件包,软件包被分成一些组。
组让用户能够简单的通过 "pacman -S <group name>" 来安装一系列软件。

\subsubsection{blackarch}
blackarch 组是所有软件包必须属于的基本组,这让用户可以轻松的安装所有软件包。

所有的软件包都应当属于这个组。

\subsubsection{blackarch-anti-forensic}
用于反取证的软件包,包括加密,隐写和修改文件属性的工具。
包括的所有工具都用于修改系统来隐藏信息。

例如:luks, TrueCrypt, Timestomp, dd, ropeadope, secure-delete

\subsubsection{blackarch-automation}
用于工具或工作流自动化的软件。

例如:blueranger, tiger, wiffy

\subsubsection{blackarch-backdoor}
利用或打开已被安装后门的系统上的后门的软件。

例如:backdoor-factory, rrs, weevely

\subsubsection{blackarch-binary}
操作一些格式的二进制文件的软件。

例如:binwally, packerid

\subsubsection{blackarch-bluetooth}
利用任何有关蓝牙标准(802.15.1)的软件包。

例如:ubertooth, tbear, redfang

\subsubsection{blackarch-code-audit}
通过代码审计来进行漏洞检测的软件包。

例如:flawfinder, pscan

\subsubsection{blackarch-cracker}
用于破解加密算法,如 哈希 的软件包。

例如:hashcat, john, crunch

\subsubsection{blackarch-crypto}
和加密有关的非破解类软件包。

例如:ciphertest, xortool, sbd

\subsubsection{blackarch-database}
用于攻击各种等级数据库的软件包。

例如:metacoretex, blindsql

\subsubsection{blackarch-debugger}
用于帮助用户实时查看特定程序正在执行的操作(调试)。

例如:radare2, shellnoob

\subsubsection{blackarch-decompiler}
用于将程序反编译成源码。

例如:flasm, jd-gui

\subsubsection{blackarch-defensive}
用于保护用户防止恶意软件或被其他用户入侵。

例如:arpon, chkrootkit, sniffjoke

\subsubsection{blackarch-disassembler}
类似于blackarch-decompiler,可能有许多软件同时属于这两个组,但这个组里的软件包生成的是汇编代码而不是源代码。

例如:inguma, radare2

\subsubsection{blackarch-dos}
用于Dos (拒绝服务) 攻击的软件包。

例如:42zip, nkiller2

\subsubsection{blackarch-drone}
用于破解,控制无线控制无人机的软件包。

例如:meshdeck, skyjack

\subsubsection{blackarch-exploitation}
利用其他程序或服务的漏洞的软件包。

例如:armitage, metasploit, zarp

\subsubsection{blackarch-fingerprint}
利用生物指纹识别设备的软件包。

例如:dns-map, p0f, httprint

\subsubsection{blackarch-firmware}
利用固件漏洞的软件包。

例如:None yet, amend asap.

\subsubsection{blackarch-forensic}
用于在磁盘或内存中寻找数据的软件包。

例如:aesfix, nfex, wyd

\subsubsection{blackarch-fuzzer}
用于进行fuzz测试,即 向目标进行随机输入并查看结果的软件包。

例如:msf, mdk3, wfuzz

\subsubsection{blackarch-hardware}
用于攻击或管理物理硬件的软件包。

例如:arduino, smali

\subsubsection{blackarch-honeypot}
用于搭建“蜜罐”(看起来是容易被入侵的服务,吸引黑客入侵)的软件包。

例如:artillery, bluepot, wifi-honey

\subsubsection{blackarch-keylogger}
用于在别的系统上记录和保存按键的软件包。

例如:None yet, amend asap.

\subsubsection{blackarch-malware}
恶意软件或检测恶意软件的软件包。

例如:malwaredetect, peepdf, yara

\subsubsection{blackarch-misc}

不属于其他类型的软件包(杂项)。

例如:oh-my-zsh-git, winexe, stompy

\subsubsection{blackarch-mobile}
用于操控移动平台的软件包。

例如:android-sdk-platform-tools, android-udev-rules

\subsubsection{blackarch-networking}
关于IP网络的软件包。

例如:Anything pretty much

\subsubsection{blackarch-nfc}
使用nfc(近场通信)的软件包。

例如:nfcutils

\subsubsection{blackarch-packer}
用于捆绑软件或有关捆绑的软件包。

\textit{捆绑程序是将恶意软件嵌入其他可执行程序的程序。}

例如:packerid

\subsubsection{blackarch-proxy}
用于搭建代理,即将流量通过另一个节点转发到互联网的软件包。

例如:burpsuite, ratproxy, sslnuke

\subsubsection{blackarch-recon}
用于主动探测目标漏洞的软件包,是由大多相似的包组成的伞式集合。

例如:canri, dnsrecon, netmask

\subsubsection{blackarch-reversing}
反编译器,反汇编器或类似的软件包。

例如:capstone, radare2, zerowine

\subsubsection{blackarch-scanner}
用于扫描目标系统漏洞的软件包。

例如:scanssh, tiger, zmap

\subsubsection{blackarch-sniffer}
有关网络流量分析的软件包。

例如:hexinject, pytactle, xspy

\subsubsection{blackarch-social}
主要攻击社交网站的软件包。

例如:jigsaw, websploit

\subsubsection{blackarch-spoof}
用于欺骗攻击,即让受害者不会发现攻击者的软件包。

例如:arpoison, lans, netcommander

\subsubsection{blackarch-threat-model}
用于在特定场合报告或记录威胁模型的软件包。

例如:magictree

\subsubsection{blackarch-tunnel}
用于在给定网络上进行隧道传输的软件包。

例如:ctunnel, iodine, ptunnel

\subsubsection{blackarch-unpacker}
用于从可执行文件中提取预打包的恶意软件。

例如:js-beautify

\subsubsection{blackarch-voip}
操作voip程序和相关协议的软件包。

例如:iaxflood, rtp-flood, teardown

\subsubsection{blackarch-webapp}
操作互联网应用程序的软件包。

例如:metoscan, whatweb, zaproxy

\subsubsection{blackarch-windows}
一些借助wine运行的原生windows程序。

例如:3proxy-win32, pwdump, winexe

\subsubsection{blackarch-wireless}
用于操作各种级别的无线网络的软件包。

例如:airpwn, mdk3, wiffy

\section{仓库结构}
BlackArch的主git仓库:
\href{https://github.com/BlackArch/blackarch}{https://github.com/BlackArch/blackarch}.
其余仓库:
\href{https://github.com/BlackArch}{https://github.com/BlackArch}.

在主git仓库里有三个重要的目录:

\begin{itemize}
\item docs - 文档。
\item packages - PKGBUILD 文件。
\item scripts - 有用的小脚本。
\end{itemize}

\subsection{脚本}
这里是\verb|scripts/|目录里脚本的介绍:

\begin{itemize}
\item baaur - 不久将用于上传软件包到AUR。
\item babuild - 构建一个软件包。
\item bachroot - 管理一个用于测试的chroot环境。
\item baclean - 清理软件包目录里的旧.pkg.tar.xz文件。
\item baconflict - 不久将替代脚本\verb|scripts/conflicts|。
\item bad-files - 查找已构建的软件包中损坏的文件。
\item balock - 申请或释放软件包仓库锁。
\item banotify - 向IRC通知推送的软件包。
\item barelease - 发布软件包到软件包仓库。
\item baright - 打印BlackArch的版权信息。
\item basign - 给软件包签名。
\item basign-key - 给钥匙签名。
\item blackman - 像是pacman,但是是从git构建(不要和nrz的Blackman混淆)。
\item check-groups - 检查组。
\item checkpkgs - 检查软件包是否有错误。
\item conflicts - 检查文件冲突。
\item dbmod - 修改一个软件包数据库。
\item depth-list - 创建按依赖关系深度排序的列表。
\item deptree - 创建依赖关系树,仅列出blackarch提供的包。
\item get-blackarch-deps - 得到某个包依赖的blackarch包的列表。
\item get-official - 获取官方发布的软件包。
\item list-loose-packages - 列出不属于组且不依赖其他包的软件包。
\item list-needed - 列出丢失的依赖项。
\item list-removed - 列出在软件包仓库但不在git管理中的软件包。
\item outdated - 查找相对与git仓库已经过时的包。
\item pkgmod - 修改一个构建包。
\item pkgrel - 在一个软件包中添加pkgrel。
\item prep - 格式化PKGBUILD 文件并查找错误。
\item sitesync - 在本地的软件包仓库和远程的软件包仓库间进行同步。
\item source-backup - 备份软件包源码文件。
\end{itemize}

\section{贡献代码}
本节介绍如何向BlackArch Linux项目贡献源码,我们接受任何大小的合并请求(pull requests),
从微小的错别字修改到提交一个新的软件包。\\如果需要帮助,建议或有疑问,请随时和我们联系。
\\\\
欢迎任何人的贡献,感谢所有的贡献。

\subsection{必要的教程}
在贡献前请阅读下面的教程。
\begin{itemize}
\item
\href{https://wiki.archlinux.org/index.php/Arch\_Packaging\_Standards)}{Arch
Packaging Standards}
\item \href{https://wiki.archlinux.org/index.php/Creating\_Packages}{Creating
Packages}
\item \href{https://wiki.archlinux.org/index.php/PKGBUILD}{PKGBUILD}
\item \href{https://wiki.archlinux.org/index.php/Makepkg}{Makepkg}
\end{itemize}

\subsection{提交贡献的步骤}
为了提交你的修改到BlackArch Linux项目,请遵循下面几步:
\begin{enumerate}
\item Fork仓库
\url{https://github.com/BlackArch/blackarch}
\item 修改必要的文件,(例如,PKGBUILD, .patch文件等).
\item 提交(commit)你的修改。
\item 推送(push)你的修改。
\item 要求我们合并你的修改,最好是通过合并请求(pull requests)。
\end{enumerate}

\subsection{示例}
下面的示例演示如何向BlackArch项目提交一个新包。我们使用 \href{https://wiki.archlinux.org/index.php/yaourt}{yaourt}
(你也可以使用pacaur) 为\textbf{nfsshell}从\href{https://aur.archlinux.org/}{AUR}获取一个已经存在的PKGBUILD文件,并根据我们的需要来调整它。

\subsubsection{获取PKGBUILD}
使用yaourt或pacaur获取 \textit{PKGBUILD}文件:
\begin{lstlisting}
  user@blackarchlinux $ yaourt -G nfsshell
  ==> Download nfsshell sources
  x LICENSE
  x PKGBUILD
  x gcc.patch
  user@blackarchlinux $ cd nfsshell/
\end{lstlisting}

\subsubsection{格式化PKGBUILD}
格式化\textit{PKGBUILD}文件并缩短执行时间:
\begin{lstlisting}
  user@blackarchlinux nfsshell $ ./blackarch/scripts/prep PKGBUILD
  cleaning 'PKGBUILD'...
  expanding tabs...
  removing vim modeline...
  removing id comment...
  removing contributor and maintainer comments...
  squeezing extra blank lines...
  removing '|| return'...
  removing leading blank line...
  removing $pkgname...
  removing trailing whitespace...
\end{lstlisting}

\subsubsection{调整PKGBUILD}
调整\textit{PKGBUILD}文件:
\begin{lstlisting}
  user@blackarchlinux nfsshell $ vi PKGBUILD
\end{lstlisting}

\subsubsection{构建一个软件包}
构建软件包:
\begin{lstlisting}user@blackarchlinux nfsshell $ makepkg -sf
==> Making package: nfsshell 19980519-1 (Mon Dec  2 17:23:51 CET 2013)
==> Checking runtime dependencies...
==> Checking buildtime dependencies...
==> Retrieving sources...
-> Downloading nfsshell.tar.gz...
% Total    % Received % Xferd  Average Speed   Time    Time     Time
CurrentDload  Upload   Total   Spent    Left  Speed100 29213  100 29213    0
0  48150      0 --:--:-- --:--:-- --:--:-- 48206
-> Found gcc.patch
-> Found LICENSE
...
<lots of build process and compiler output here>
...
==> Leaving fakeroot environment.
==> Finished making: nfsshell 19980519-1 (Mon Dec  2 17:23:53 CET 2013)
\end{lstlisting}

\subsubsection{安装并测试软件包}
安装并测试软件包:
\begin{lstlisting}
  user@blackarchlinux nfsshell $ pacman -U nfsshell-19980519-1-x86_64.pkg.tar.xz
  user@blackarchlinux nfsshell $ nfsshell # test it
\end{lstlisting}

\subsubsection{添加,提交并推送软件包}
添加,提交并推送软件包:
\begin{lstlisting}user@blackarchlinux nfsshell $ cd /blackarchlinux/packages
user@blackarchlinux ~/blackarchlinux/packages $ mv ~/nfsshell .
user@blackarchlinux ~/blackarchlinux/packages $ git commit -am nfsshell && git push
\end{lstlisting}

\subsubsection{创建一个合并请求(pull requests)}
在\href{https://github.com/}{github.com}上创建一个合并请求(pull requests)。
\begin{lstlisting}
  firefox https://github.com/<contributor>/blackarchlinux
\end{lstlisting}

\subsubsection{添加要提交的远程仓库}
如果你拉取(pull)的是自己fork主仓库的仓库,一个聪明的做法是是把主仓库作为远程仓库:
\begin{lstlisting}
  user@blackarchlinux ~/blackarchlinux $ git remote -v
  origin <the url of your fork> (fetch)
  origin <the url of your fork> (push)
  user@blackarchlinux ~/blackarchlinux $ git remote add upstream https://github.com/blackarch/blackarch
  user@blackarchlinux ~/blackarchlinux $ git remote -v
  origin <the url of your fork> (fetch)
  origin <the url of your fork> (push)
  upstream https://github.com/blackarch/blackarch (fetch)
  upstream https://github.com/blackarch/blackarch (push)
\end{lstlisting}

默认情况下,git将直接推送到远程仓库origin,但确保你的git配置正确,如果你有提交权限,这就不是问题,
但如果你没有提交权限,就不能向上游推送。

如果你有提交能力,也许你会使用
\textit{git@github.com:blackarch/blackarch.git},这取决于你。

\subsection{要求}
\begin{enumerate}
\item 不要添加 \textbf{Maintainer} 或 \textbf{Contributor} 注释到
\textit{PKGBUILD} 文件,添加维护者和贡献者的名字到BlackArch指南的AUTHORS章节。
\item 为了保持一致性,请遵循仓库中其他
\textit{PKGBUILD} 文件的通用风格并使用双空格缩进。
\end{enumerate}

\subsection{一般提示}
\href{http://wiki.archlinux.org/index.php/Namcap}{namcap} 可以检查软件包的错误。

%------------------%
%  Chapter 4       %
%------------------%

\chapter{工具指南}
敬请期待

\section{敬请期待}
敬请期待

%%% APPENDIX %%%
\appendix
\include{latex/appendix-zh}

\end{document}

%%% EOF %%%
