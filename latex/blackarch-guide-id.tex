%%%%%%%%%%%%%%%%%%%%%%%%%%%%%%%%%%%%%%%%%%%%%%%%%%%%%%%%%%%%%%%%%%%%%%%%%%%%%%%%
%                                                                              %
% Panduan Blackarch Linux                                                      %
%                                                                              %
%%%%%%%%%%%%%%%%%%%%%%%%%%%%%%%%%%%%%%%%%%%%%%%%%%%%%%%%%%%%%%%%%%%%%%%%%%%%%%%%

\documentclass[a4paper, oneside, 11pt]{book}

%%% INCLUDES %%%
\renewcommand{\familydefault}{\sfdefault}

\usepackage{array}
\usepackage{color}
\usepackage{enumerate}
\usepackage{fancyhdr}
\usepackage{fancyvrb}
\usepackage{geometry}
\usepackage{graphicx}
\usepackage{html}
\usepackage{hyperref}
\usepackage{ifpdf}
\usepackage{listings}
\usepackage{pstricks}
\usepackage{supertabular}
\usepackage{tocloft}
\usepackage[utf8]{inputenc}

%%% LAYOUT %%%
\setlength{\parindent}{0em}
\setlength{\parskip}{1.5ex plus0.5ex minus0.5ex}
\geometry{left=2.5cm, textwidth=16cm, top=3cm, textheight=25cm, bottom=3cm}
\widowpenalty=2000
\clubpenalty=1000
\frenchspacing
\sloppy
\pagecolor[HTML]{FFFFFF}
\color[HTML]{333333}
\setcounter{tocdepth}{10}
\setcounter{secnumdepth}{10}

\hypersetup{
  pdftitle={BlackArch Linux, Panduan BlackArch Linux},
  pdfsubject={BlackArch Linux, Panduan BlackArch Linux},
  pdfauthor={BlackArch Linux, BlackArch Linux},
  pdfkeywords={BlackArch Linux, Penetration Testing, Security, Hacking, Linux},
  pdfcenterwindow=true,
  colorlinks=true,
  breaklinks=true,
  linkcolor=blue,
  menucolor=blue,
  urlcolor=blue
}

% syntax highlighting
% all options should be set here document wide
\lstset{
backgroundcolor=\color[HTML]{EEEEEE},
frame=single,
basicstyle=\footnotesize\ttfamily,
float,
deletekeywords={return},
otherkeywords={mkdir, curl, sudo, sha1sum, grep, cut, sort, wget, makepkg,
pacman, blackman, chmod},
keywordstyle=\color{orange},
commentstyle=\color{blue},
stringstyle=\color{red},
language=bash,
showspaces=false,
showtabs=false,
tabsize=2
}

%%% HEADER / FOOTER %%%
\setlength{\headheight}{33pt}
\setlength{\headsep}{33pt}
\lhead{{\includegraphics[width=1cm,height=1cm]{images/logo.png}}}
\rhead{Panduan BlackArch Linux}

%%% CUSTOM MACROS %%%
% for html links
\ifpdf\else
\def\href#1#2{\htmladdnormallink{#2}{#1}}
\fi

%------------------%
%  JUDUL HALAMAN   %
%------------------%
\begin{document}
\pagestyle{empty}
\begin{center}
\begin{figure}[htbp]
\centering
\vspace{0.5cm}
\includegraphics[width=8cm]{images/logo.png}
\label{fig:logo}
\end{figure}
\vspace{0.5cm}
\Huge{\textbf{Panduan BlackArch Linux}}\\
\vspace{1cm}
\Large{\color{blue}https://www.blackarch.org/}\\
\vspace{0.5cm}
\end{center}
\newpage
\tableofcontents
\newpage
\pagestyle{fancy}

%------------------%
%  Bab 1           %
%------------------%

\chapter{Introduction}

\section{Overview}
Panduan BlackArch Linux dibagi menjadi beberapa bagian:
\begin{itemize}
\item Pendahuluan - Memberikan pandangan luas, pendahuluan dan informasi tambahan seputar proyek
\item Panduan Pengguna - Segala sesuatu yang perlu diketahui oleh pengguna untuk menggunakan BlackArch secara efektif
\item Panduan Pengembang - Panduan cara memulai mengembangkan dan berkontribusi pada BlackArch
\item Panduan Tool - Detail tool mendalam beserta contoh penggunaan(WIP)
\end{itemize}

\section{Apa itu BlackArch Linux?}
BlackArch adalah distro Linux lengkap untuk penetration tester dan security researchers. 
BlackArch berasal dari \href{https://www.archlinux.org/}{ArchLinux} dan pengguna dapat install komponen-komponen BlackArch
secara masing-masing atau dalam kelompok langsung di atasnya.

Toolset didistribusikan sebagai Arch Linux
\href{https://wiki.archlinux.org/index.php/Unofficial\_User\_Repositories}
{unofficial user repository} sehingga Anda dapat menginstal BlackArch di atas
instalasi Arch Linux yang ada. Paket dapat diinstal secara individual atau dengan
kategori.

Repositori yang terus berkembang saat ini mencakup lebih dari \href{https://www.blackarch.org/tools.html}{2600} alat.
Semua alat diuji secara menyeluruh sebelum ditambahkan ke basis kode untuk menjaga kualitas repositori.
% should quickly describe the testing methods/code review procedures etc

\section{Sejarah BlackArch Linux}
Segera...

\section{Platform yang didukung}
Segera...

\section{Terlibat}
Anda dapat menghubungi tim BlackArch menggunakan jalur berikut:

Website: \url{https://www.blackarch.org/}

Mail: \href{mailto:team@blackarch.org}{team@blackarch.org}

IRC: \url{irc://irc.freenode.net/blackarch}

Twitter: \url{https://twitter.com/blackarchlinux}

Github: \url{https://github.com/Blackarch/}

%------------------%
%  Chapter 2       %
%------------------%


\chapter{Panduan Pengguna}

\section{Installasi}
Bagian ini akan menunjukkan kepada Anda cara mempersiapkan repositori BlackArch dan
instalasi paket. BlackArch mendukung keduanya, instalasi dari repositori meggunakan
paket biner serta kompilasi dan instalasi dari source.

BlackArch kompatibel dengan instalasi Arch normal. Itu bertindak sebagai repositori 
pengguna tidak resmi. Jika anda menginginkan ISO, lihat bagian
\href{https://www.blackarch.org/downloads.html#iso}{ISOs}.

\subsection{Instalasi di atas ArchLinux}
Jalankan \href{https://blackarch.org/strap.sh}{strap.sh} sebagai root dan ikuti
perintah-perintah. Lihat conth berikut.
\begin{lstlisting}
   curl -O https://blackarch.org/strap.sh
   sha1sum strap.sh # should match: 5ea40d49ecd14c2e024deecf90605426db97ea0c
   sudo chmod +x strap.sh
   sudo ./strap.sh
\end{lstlisting}

Sekarang unduh salinan baru dari daftar paket master dan sinkronisasi paket:
\begin{lstlisting}
  sudo pacman -Syyu
\end{lstlisting}


\subsection{Instalasi paket}
Anda sekarang dapat menginstal alat dari repositori BlackArch.
\begin{enumerate}
\item Untuk melihat daftar semua alat yang tersedia, jalankan
\begin{lstlisting}
  pacman -Sgg | grep blackarch | cut -d' ' -f2 | sort -u
\end{lstlisting}

\item Untuk instalasi semua alat, jalankan
\begin{lstlisting}
  pacman -S blackarch
\end{lstlisting}

\item Untuk instalasi alat berdasarkan kategori, jalankan
\begin{lstlisting}
  pacman -S blackarch-<category>
\end{lstlisting}

\item Untuk melihat kategori blackarch, jalankan
\begin{lstlisting}
  pacman -Sg | grep blackarch
\end{lstlisting}

\end{enumerate}

\subsection{Menginstal paket dari sumber}
Sebagai bagian dari metode instalasi alternatif, Anda dapat membangun BlackArch
paket dari sumber. Anda dapat menemukan PKGBUILD di
\href{https://github.com/BlackArch/blackarch/tree/master/packages}{github}. Untuk
membangun seluruh repo, Anda dapat menggunakan
\href{https://github.com/BlackArch/blackman}{Blackman} alat.
\begin{itemize}
\item Pertama, Anda harus menginstal Blackman. Jika repositori paket BlackArch
sudah diatur pada mesin Anda, Anda dapat menginstal Blackman:
\begin{lstlisting}
  pacman -S blackman
\end{lstlisting}

\item Anda dapat membangun dan menginstal Blackman dari sumber:
\begin{lstlisting}
  mkdir blackman
  cd blackman
  wget https://raw2.github.com/BlackArch/blackarch/master/packages/blackman/PKGBUILD
  # Pastikan PKGBUILD tidak dirusak dengan maksud jahat.
  makepkg -s
\end{lstlisting}

\item Atau Anda dapat menginstal Blackman dari AUR:
\begin{lstlisting}
  <apa pun penolong AUR yang Anda gunakan> -S blackman
\end{lstlisting}

\end{itemize}

\subsection{Penggunaan dasar Blackman} Blackman sangat mudah digunakan, meskipun benderanya berbeda dari Anda
biasanya diharapkan dari sesuatu seperti pacman. Penggunaan dasar telah diuraikan di bawah ini.
\begin{itemize}
\item Unduh, kompilasi, dan instal paket:
\begin{lstlisting}
  sudo blackman -i paket
\end{lstlisting}

\item Unduh, kompilasi, dan instal seluruh kategori:
\begin{lstlisting}
  sudo blackman -g kelompok
\end{lstlisting}

\item Unduh, kompilasi, dan instal semua alat BlackArch:
\begin{lstlisting}
  sudo blackman -a
\end{lstlisting}

\item Untuk membuat daftar kategori blackarch:
\begin{lstlisting}
  blackman -l
\end{lstlisting}

\item Untuk membuat daftar alat katagori:
\begin{lstlisting}
  blackman -p kategori
\end{lstlisting}

\end{itemize}

\subsection{Menginstal dari full-, netinstall- ISO atau ArchLinux}
Anda dapat menginstal Blackarch Linux dari salah satu live- atau netinstall-ISO kami.\\See
\url{https://www.blackarch.org/download.html#iso}. Langkah-langkah berikut
diperlukan setelah ISO boot.
\begin{itemize}
\item Install blackarch-installer package:
\begin{lstlisting}
  sudo pacman -S blackarch-installer
\end{lstlisting}

\item Run
\begin{lstlisting}
  sudo blackarch-install
\end{lstlisting}

\end{itemize}

%------------------%
%  Chapter 3       %
%------------------%

\chapter{Developer Guide}

\section{Arch's Build System and Repositories}

PKGBUILD files are build scripts. Each one tells makepkg(1) how to create a
package. PKGBUILD files are written in Bash.

For more information, read (or skim through) the following:
\begin{itemize}
\item \href{https://wiki.archlinux.org/index.php/Creating_Packages}{Arch Wiki: Creating Packages}
\item \href{https://wiki.archlinux.org/index.php/Makepkg}{Arch Wiki: makepkg}
\item \href{https://wiki.archlinux.org/index.php/PKGBUILD}{Arch Wiki: PKGBUILD}
\item \href{https://wiki.archlinux.org/index.php/Arch_Packaging_Standards}{Arch Wiki: Arch Packaging Standards}
\end{itemize}

\section{Blackarch PKGBUILD standards}
For the sake of simplicity, our PKGBUILDs are similar to that of the AUR ones,
with a few small differences outlined below. Every package must
belong to blackarch at the minimum, there will also be a lot of crossover with
multiple packages belonging to multiple groups.

\subsection{Groups}
To allow users to install a specific range of packages quickly and easily,
packages have been separated into groups. Groups allow users to simply
go "pacman -S <group name>" in order to pull a lot of packages.

\subsubsection{blackarch}
The blackarch group is the base group that all packages must belong too. This allows
users to install every package with ease.

What should be in here: Everything.

\subsubsection{blackarch-anti-forensic}
Packages that are used for countering forensic activities,
including encryption, steganography, and anything that modifies files/file attributes.
This all includes tools to work with anything in general that makes changes to a system
for the purposes of hiding information.

Examples: luks, TrueCrypt, Timestomp, dd, ropeadope, secure-delete

\subsubsection{blackarch-automation}
Packages that are used for tool or workflow automation.

Examples: blueranger, tiger, wiffy

\subsubsection{blackarch-backdoor}
Packages that exploit or open backdoors on already vulnerable systems.

Examples: backdoor-factory, rrs, weevely

\subsubsection{blackarch-binary}
Packages that operate on binary files in some form.

Examples: binwally, packerid

\subsubsection{blackarch-bluetooth}
Packages that exploit anything concerning the Bluetooth standard (802.15.1).

Examples: ubertooth, tbear, redfang

\subsubsection{blackarch-code-audit}
Packages that audit existing source code for vulnerability analysis.

Examples: flawfinder, pscan

\subsubsection{blackarch-cracker}
Packages used for cracking cryptographic functions, ie hashes.

Examples: hashcat, john, crunch

\subsubsection{blackarch-crypto}
Packages that work with cryptography, with the exception of cracking.

Examples: ciphertest, xortool, sbd

\subsubsection{blackarch-database}
Packages that involve database exploitations on any level.

Examples: metacoretex, blindsql

\subsubsection{blackarch-debugger}
Packages that allow the user to view what a particular program is "doing" in realtime.

Examples: radare2, shellnoob

\subsubsection{blackarch-decompiler}
Packages that attempt to reverse a compiled program into source code.

Examples: flasm, jd-gui

\subsubsection{blackarch-defensive}
Packages that are used to protect a user from malware \& attacks from other users.

Examples: arpon, chkrootkit, sniffjoke

\subsubsection{blackarch-disassembler}
This is similar to blackarch-decompiler, and there will probably be a lot
of programs that fall into both, however these packages produce assembly output
rather than the raw source code.

Examples: inguma, radare2

\subsubsection{blackarch-dos}
Packages that use DoS (Denial of Service) attacks.

Examples: 42zip, nkiller2

\subsubsection{blackarch-drone}
Packages that are used for managing physically engineered
drones.

Examples: meshdeck, skyjack

\subsubsection{blackarch-exploitation}
Packages that takes advantages of exploits in other programs or services.

Examples: armitage, metasploit, zarp

\subsubsection{blackarch-fingerprint}
Packages that exploit fingerprint biometric equipment.

Examples: dns-map, p0f, httprint

\subsubsection{blackarch-firmware}
Packages that exploit vulnerabilities in firmware

Examples: None yet, amend asap.

\subsubsection{blackarch-forensic}
Packages that are used to find data on physical disks or embedded memory.

Examples: aesfix, nfex, wyd

\subsubsection{blackarch-fuzzer}
Packages that use the fuzz testing principle, ie
"throwing" random inputs at the subject to see what happens.

Examples: msf, mdk3, wfuzz

\subsubsection{blackarch-hardware}
Packages that exploit or manage anything to do with
physical hardware.

Examples: arduino, smali

\subsubsection{blackarch-honeypot}
Packages that act as "honeypots", ie programs that appear to
be vulnerable services used to attract hackers into a trap.

Examples: artillery, bluepot, wifi-honey

\subsubsection{blackarch-keylogger}
Packages that record and retain keystrokes on another system.

Examples: None yet, amend asap.

\subsubsection{blackarch-malware}
Packages that count as any type of malicious software or
malware detection.

Examples: malwaredetect, peepdf, yara

\subsubsection{blackarch-misc}
Packages that don't particularly fit into any categories.

Examples: oh-my-zsh-git, winexe, stompy

\subsubsection{blackarch-mobile}
Packages that manipulate mobile platforms.

Examples: android-sdk-platform-tools, android-udev-rules

\subsubsection{blackarch-networking}
Package that involve IP networking.

Examples: Anything pretty much

\subsubsection{blackarch-nfc}
Packages that use nfc (near-field communications).

Examples: nfcutils

\subsubsection{blackarch-packer}
Packages that operate on or invlove packers.

\textit{packers are programs that embed malware within other executables.}

Examples: packerid

\subsubsection{blackarch-proxy}
Packages that acts as a proxy, ie redirecting traffic
through another node on the internet.

Examples: burpsuite, ratproxy, sslnuke

\subsubsection{blackarch-recon}
Packages that actively seeks vulnerable exploits in the
wild. More of an umbrella group for similar packages.

Examples: canri, dnsrecon, netmask

\subsubsection{blackarch-reversing}
This is an umbrella group for any decompiler,
disassembler or any similar program.

Examples: capstone, radare2, zerowine

\subsubsection{blackarch-scanner}
Packages that scan selected systems for vulnerabilities.

Examples: scanssh, tiger, zmap

\subsubsection{blackarch-sniffer}
Packages that involve analyzing network traffic.

Examples: hexinject, pytactle, xspy

\subsubsection{blackarch-social}
Packages that primarily attack social networking sites.

Examples: jigsaw, websploit

\subsubsection{blackarch-spoof}
Packages that attempt to spoof the attacker such, in that
the attacker doesn't show up as an attacker to the victim.

Examples: arpoison, lans, netcommander

\subsubsection{blackarch-threat-model}
Packages that would be used for reporting/recording the
threat model outlined in a particular scenario.

Examples: magictree

\subsubsection{blackarch-tunnel}
Packages that are used to tunnel network traffic on a given
network.

Examples: ctunnel, iodine, ptunnel

\subsubsection{blackarch-unpacker}
Packages that are used to extract pre-packed malware from an
executable.

Examples: js-beautify

\subsubsection{blackarch-voip}
Packages that operate on voip programs and protocols.

Examples: iaxflood, rtp-flood, teardown

\subsubsection{blackarch-webapp}
Packages that operate on internet-facing applications.

Examples: metoscan, whatweb, zaproxy

\subsubsection{blackarch-windows}
This group is for any native Windows package that runs via wine.

Examples: 3proxy-win32, pwdump, winexe

\subsubsection{blackarch-wireless}
Packages that operates on wireless networks on any level.

Examples: airpwn, mdk3, wiffy

\section{Repository structure}
You can find the main BlackArch git repo here:
\href{https://github.com/BlackArch/blackarch}{https://github.com/BlackArch/blackarch}.
There are also several secondary repos here:
\href{https://github.com/BlackArch}{https://github.com/BlackArch}.

Within the main git repo, there are three important directories:

\begin{itemize}
\item docs - Documentation.
\item packages - PKGBUILD files.
\item scripts - Useful little scripts.
\end{itemize}

\subsection{Scripts}
Here is a reference for scripts in the \verb|scripts/| directory:

\begin{itemize}
\item baaur - Soon, this will upload packages to the AUR.
\item babuild - Build a package.
\item bachroot - Manage a chroot for testing.
\item baclean - Clean old .pkg.tar.xz files from the package repo.
\item baconflict - Soon this will replace \verb|scripts/conflicts|.
\item bad-files - Find bad files in built packages.
\item balock - Obtain or release the package repo lock.
\item banotify - Notify IRC about package pushes.
\item barelease - Release packages to the package repo.
\item baright - Print the BlackArch copyright info.
\item basign - Sign packages.
\item basign-key - Sign a key.
\item blackman - This behaves sort of like pacman but builds from git (not to be
    confused with nrz's Blackman).
\item check-groups - Check groups.
\item checkpkgs - Check packages for errors.
\item conflicts - Check for file conflicts.
\item dbmod - Modify a package database.
\item depth-list - Create a list sorted by dependency depth.
\item deptree - Create a dependency tree, listing only blackarch-provided packages.
\item get-blackarch-deps - Get a list of blackarch dependencies for a package.
\item get-official - Get official packages for release.
\item list-loose-packages - List packages that are not in groups and are not
    dependencies for other packages.
\item list-needed - List missing dependencies.
\item list-removed - List packages that are in the package repo but not in git.
\item list-tools - List tools.
\item outdated - Look for packages that are out-dated in the package repo
    relative to the git repo.
\item pkgmod - Modify a build package.
\item pkgrel - Increment pkgrel in a package.
\item prep - Clean up a PKGBUILD file's style and find errors.
\item sitesync - Sync between a local copy of the package repo and a remote copy.
\item size-hunt - Hunt for large packages.
\item source-backup - Backup package source files.
\end{itemize}

\section{Contributing to repository}
This section shows you how to contribute to the BlackArch Linux project. We
accept pull requests of all sizes, from tiny typo fixes to new packages.\\For
help, suggestions, or questions feel free to contact us.
\\\\
Everyone is welcome to contribute. All contributions are appreciated.

\subsection{Required tutorials}
Please read the following tutorials before contributing:
\begin{itemize}
\item
\href{https://wiki.archlinux.org/index.php/Arch\_Packaging\_Standards)}{Arch
Packaging Standards}
\item \href{https://wiki.archlinux.org/index.php/Creating\_Packages}{Creating
Packages}
\item \href{https://wiki.archlinux.org/index.php/PKGBUILD}{PKGBUILD}
\item \href{https://wiki.archlinux.org/index.php/Makepkg}{Makepkg}
\end{itemize}

\subsection{Steps for contributing}
In order to submit your changes to the BlackArchLinux project, follow these
steps:
\begin{enumerate}
\item Fork the repository from
\url{https://github.com/BlackArch/blackarch}
\item Hack the necessary files, (e.g. PKGBUILD, .patch files, etc).
\item Commit your changes.
\item Push your changes.
\item Ask us to merge in your changes, preferably through a pull request.
\end{enumerate}

\subsection{Example}
The following example demonstrates submitting a new package to the BlackArch
project. We use \href{https://wiki.archlinux.org/index.php/yaourt}{yaourt}
(you can use pacaur as well) to fetch a pre-existing PKGBUILD file for
\textbf{nfsshell} from the \href{https://aur.archlinux.org/}{AUR} and adjust it
according to our needs.

\subsubsection{Fetch PKGBUILD}
Fetch the \textit{PKGBUILD} file using yaourt or pacaur:
\begin{lstlisting}
  user@blackarchlinux $ yaourt -G nfsshell
  ==> Download nfsshell sources
  x LICENSE
  x PKGBUILD
  x gcc.patch
  user@blackarchlinux $ cd nfsshell/
\end{lstlisting}

\subsubsection{Clean up PKGBUILD}
Clean up the \textit{PKGBUILD} file and save some time:
\begin{lstlisting}
  user@blackarchlinux nfsshell $ ./blackarch/scripts/prep PKGBUILD
  cleaning 'PKGBUILD'...
  expanding tabs...
  removing vim modeline...
  removing id comment...
  removing contributor and maintainer comments...
  squeezing extra blank lines...
  removing '|| return'...
  removing leading blank line...
  removing $pkgname...
  removing trailing whitespace...
\end{lstlisting}

\subsubsection{Adjust PKGBUILD}
Adjust the \textit{PKGBUILD} file:
\begin{lstlisting}
  user@blackarchlinux nfsshell $ vi PKGBUILD
\end{lstlisting}

\subsubsection{Build the package}
Build the package:
\begin{lstlisting}user@blackarchlinux nfsshell $ makepkg -sf
==> Making package: nfsshell 19980519-1 (Mon Dec  2 17:23:51 CET 2013)
==> Checking runtime dependencies...
==> Checking buildtime dependencies...
==> Retrieving sources...
-> Downloading nfsshell.tar.gz...
% Total    % Received % Xferd  Average Speed   Time    Time     Time
CurrentDload  Upload   Total   Spent    Left  Speed100 29213  100 29213    0
0  48150      0 --:--:-- --:--:-- --:--:-- 48206
-> Found gcc.patch
-> Found LICENSE
...
<lots of build process and compiler output here>
...
==> Leaving fakeroot environment.
==> Finished making: nfsshell 19980519-1 (Mon Dec  2 17:23:53 CET 2013)
\end{lstlisting}

\subsubsection{Install and test the package}
Install and test the package:
\begin{lstlisting}
  user@blackarchlinux nfsshell $ pacman -U nfsshell-19980519-1-x86_64.pkg.tar.xz
  user@blackarchlinux nfsshell $ nfsshell # test it
\end{lstlisting}

\subsubsection{Add, commit and push package}
Add, commit and push the package
\begin{lstlisting}user@blackarchlinux nfsshell $ cd /blackarchlinux/packages
user@blackarchlinux ~/blackarchlinux/packages $ mv ~/nfsshell .
user@blackarchlinux ~/blackarchlinux/packages $ git commit -am nfsshell && git push
\end{lstlisting}

\subsubsection{Create a pull request}
Create a pull request on \href{https://github.com/}{github.com}
\begin{lstlisting}
  firefox https://github.com/<contributor>/blackarchlinux
\end{lstlisting}

\subsubsection{Adding a remote for upstream}
A smart thing to do if you're working upstream and on a fork is to pull your own fork and add the main ba repo as a remote
\begin{lstlisting}
  user@blackarchlinux ~/blackarchlinux $ git remote -v
  origin <the url of your fork> (fetch)
  origin <the url of your fork> (push)
  user@blackarchlinux ~/blackarchlinux $ git remote add upstream https://github.com/blackarch/blackarch
  user@blackarchlinux ~/blackarchlinux $ git remote -v
  origin <the url of your fork> (fetch)
  origin <the url of your fork> (push)
  upstream https://github.com/blackarch/blackarch (fetch)
  upstream https://github.com/blackarch/blackarch (push)
\end{lstlisting}

By default, git should push straight to origin, but make sure your git config is
configured correctly. This won't be an issue unless you have commit rights as
you won't be able to push upstream without them.

If you do have the ability to commit, you might have more success using
\textit{git@github.com:blackarch/blackarch.git} but it's up to you.

\subsection{Requests}
\begin{enumerate}
\item Don't add \textbf{Maintainer} or \textbf{Contributor} comments to
\textit{PKGBUILD} files. Add maintainer and contributor names to the
AUTHORS section of BlackArch guide.
\item For the sake of consistency, please follow the general style of the other
\textit{PKGBUILD} files in the repo and use two-space indentation.
\end{enumerate}

\subsection{General tips}
\href{http://wiki.archlinux.org/index.php/Namcap}{namcap} can check packages for
errors.

%------------------%
%  Chapter 4       %
%------------------%

\chapter{Tools Guide}
Coming soon...

\section{Coming Soon}
Coming soon...

%%% APPENDIX %%%
\appendix
\include{latex/appendix}

\end{document}

%%% EOF %%%
